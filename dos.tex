\documentclass[10pt,twoside]{book}
\usepackage{a4}
\usepackage{times}
%
\setlength{\oddsidemargin}{0cm}
\setlength{\evensidemargin}{0cm}
\setlength{\textwidth}{15.2cm}
\setlength{\footskip}{1.6cm}
%\setlength{\footheight}{0.5cm}
\setlength{\topmargin}{0.0cm}
\setlength{\headheight}{0.5cm}
\setlength{\headsep}{1.5cm}
\setlength{\topskip}{0.5cm}
\setlength{\textheight}{21.8cm}
%
\usepackage{fancyhdr}
%\nonfrenchspacing
\usepackage{latexsym}
\usepackage{t1enc}
\usepackage[latin1]{inputenc}
\usepackage{index}
\pagestyle{headings}
\pagestyle{fancy}
%
\renewcommand{\labelitemi}{\raisebox{0.3ex}{\tiny $\Box$}}
%
%
\renewcommand{\sectionmark}[1]{%
\markboth{#1}{}}
\fancyhead{}
\fancyfoot{}
\fancyfoot[LE]{{\sf \thepage} \hspace{1em} {\sfi Rom Kernel Reference Manual: DOS}}
\fancyfoot[RO]{{\sfi \leftmark} \hspace{1em} {\sf \thepage}}
% Redefine the plain page style
\fancypagestyle{plain}{%
  \fancyhf{}%
  \fancyfoot[LE]{{\sf \thepage} \hspace{1em} {\sfi Rom Kernel Reference Manual: DOS}}
  \fancyfoot[RO]{{\sfi \leftmark} \hspace{1em} {\sf \thepage}}
  \renewcommand{\headrulewidth}{2.0mm}
  \renewcommand{\footrulewidth}{0.0mm}
}
\renewcommand{\headrulewidth}{2.0mm}
\renewcommand{\footrulewidth}{0.0mm}
%
\newcommand{\sfi}{\sffamily\slshape}
\newcommand{\sfb}{\sffamily\bfseries}
\newcommand{\cnot}{\symbol{126}}
\setcounter{secnumdepth}{4}
%
%
\newcounter{rkrmtablecount}
%
\newenvironment{rkrmtable}[1]%
{%
\nopagebreak[5]%
\begin{center}%
\begin{minipage}{15.2cm}%
\begin{center}%
\refstepcounter{rkrmtablecount}
{\sfb Table \arabic{rkrmtablecount}: #1} \\[1ex]
\begin{tabular}{|ll|} \hline
}{%
\end{tabular}%
\end{center}%
\end{minipage}%
\end{center}%
}
%
\newlength{\punchlinewidth}
\setlength{\punchlinewidth}{\textwidth}
\addtolength{\punchlinewidth}{-\rightmargin}
\addtolength{\punchlinewidth}{-\leftmargin}
\addtolength{\punchlinewidth}{-4.2pt}
%
\newlength{\jpgtablewidth}
\setlength{\jpgtablewidth}{\textwidth}
\addtolength{\jpgtablewidth}{-\rightmargin}
\addtolength{\jpgtablewidth}{-\leftmargin}
\addtolength{\jpgtablewidth}{-4.2pt}
%
%
\newcommand{\punchline}[2]{%
\bigskip{\setlength{\arrayrulewidth}{1mm}
\begin{tabular}{|p{\punchlinewidth}}
{\sfi #1} #2
\end{tabular}
}\bigskip}

%
% Insert the page under the given key into the index.
\newcommand{\key}[1]{\index{#1@{\scriptsize{} #1}}}
% The same again, but make this the main entry.
\newcommand{\mkey}[1]{\index{#1@{\scriptsize{} #1}|emph}}
% An entry defining a library function call
\newcommand{\call}[1]{{\bf J2K::#1()\key{#1()}}}
%
\makeindex
%
%
\begin{document}
\thispagestyle{empty}
\pagenumbering{roman}
\begin{center}
\vspace*{\fill}
{\Huge Amiga ROM Kernel Reference Manual\\}
\bigskip
{\Huge DOS\\}
\bigskip
{\huge\sc
Thomas Richter\\}
\bigskip
\end{center}
\bigskip
\vspace*{\fill}
\bigskip
{\small Copyright � 2023 by Thomas Richter, all rights reserved. This
publication is freely distributable under the restrictions stated below, but is
also Copyright � Thomas Richter.

Distribution of the publication by a commercial organization without written
permission from the author to any third party is prohibited if any payment is
made in connection with such distribution, whether directly (as in payment for a
copy of the publication) or indirectly (as in payment for some service related
to the Publication, or payment for some product or service that includes a
copy of the publication ``without charge''; these are only examples, and
not an exhaustive enumeration of prohibited activities).

However, the following methods of distribution involving payment shall not in
and of themselves be a violation of this restriction:
\begin{enumerate}
\item Distributing the Program on a physical data carrier (e.g. CD-ROM,
  DVD, USB-Stick, Disk...) provided that:
  \begin{enumerate}
  \item the Archive is reproduced entirely and verbatim on such data carrier,
    including especially this licence agreement;
  \item the data carrier is made available to the public for a nominal
    fee only, i.e. for a fee that covers the costs of the data carrier,
    and shipment of the data carrier;
  \item a data carrier with the Program installed is made available to the
    author for free except for shipment costs, and
  \item provided further that all information on said data carrier is
    redistributable for non-commercial purposes without charge.
  \end{enumerate}
\end{enumerate}

Redistribution of a modified version of the publication is prohibited in any
way, by any organization, regardless whether commercial or non-commercial.
Everything must be kept together, in original and unmodified form. }

\bigskip

{\small \sc Disclaimer: This publication is provided ``as is'' without any
warranty of any kind, either expressed or implied, including, but not limited
to, the implied warranties of merchantability and fitness for any particular
purpose. Further, the author does not warrant, guarantee, or make any
representation regarding the use of, or the results of the use of, the
information contained herein in term of correctness, accuracy, reliability,
currentness, or otherwise; the entire risk as to its quality and accuracy is
assumed solely by the user. Should the information prove inaccurate, the user
(and not the author) assumes the either cost of all necessary correction. In no
event will the author be liable for direct, indirect, incidental, or
consequential damages resulting from any defect or inaccuracy in this
publication, even if advised of the possibility of such damages. Some laws do
not allow the exclusion or limitation of implied warranties or liabilities for
incidental or consequential damages, so the above limitation or exclusion may
not apply.}

\bigskip

{\small {\it Amiga} is a registered trademark, {\it Amiga-DOS}, {\it Exec} and
{\it Kickstart} are registered trademarks of Amiga Intl. {\it Motorola} is a
registered trademark of Motorola, inc. {\it Unix} is a trademark of
the Open Group.}
\newpage
\vspace*{\fill}
\newpage
\tableofcontents
\pagenumbering{arabic}
%
%
\chapter{Introduction}
\section{Purpose}
The purpose of this manual is to provide a comprehensive documentation
of the AmigaDOS subsystem of the Amiga Operation System. This
subsystem is represented by the \emph{dos.library}, and it provides
services around files, file systems and stream-based input and
output. While the Amiga ROM Kernel Reference Manuals~\cite{rkrmlib}
document major parts of the AmigaOs, they do not include a volume on
AmigaDOS itself. This is due to the history of AmigaDOS which is
nothing but a port of the TRIPOS to the Amiga, and thus its
documentation became available as the AmigaDOS manual\cite{Bantam}
separately. This book itself is, similar to AmigaDOS, based on the
TRIPOS manual which has been augmented and updated to reflect the
changes that were necessary to fit TRIPOS into AmigaOs. Unfortunately,
the book is hard to obtain, and also leaves a lot to deserve.
\smallskip{}

Good third party documentation is available in the form of the Guru
Book\cite{guru}, though this source is out of print and even harder to
obtain. It covers also other aspects of AmigaOs that go beyond
AmigaDOS such that its focus is a bit different than this work.
\smallskip{}

This work attempts to fill this gap by providing a comprehensive and
complete documentation of the AmigaDOS library and its subsystems in
the style of the ROM Kernel Reference Manuals.

\section{Language and Type Setting Conventions}
The words \emph{shall} and \emph{shall not} indicate normative
requirements software shall or shall not follow or in order to satisfy the
interface requirements of AmigaOs. The words \emph{should} and
\emph{should not} indicate best practise and recommendations that are
advisable, but not strictly necessary to satisfy a particular
interface. The word \emph{may} provides a hint to a possible
implementation strategy.
\smallskip{}

The word \emph{must} indicates a logical consequence from existing
requirements or conditions that follows necessarily without
introducing a new restriction, such as in ``if $a$ is 2, $a+a$
\emph{must} be 4''.

\punchline{Worth to remember!}{Important aspects of the text are
  indicated with a bold vertical bar like this.}

Terms are indicated in \emph{italics}, e.g. the \emph{dos.library}
implements interface of \emph{AmigaDOS}. Data structures and
components of source code are printed {\tt in courier} in fixed-width
font, reassembling the output of a terminal, e.g.

\begin{verbatim}
typedef unsigned char UBYTE; /* an 8-bit unsigned integer */
typedef long LONG;           /* a 32-bit integer          */
\end{verbatim}

\chapter{Elementary Data Types}
\section{The dos.library}
\emph{AmigaDOS} as part of the \emph{Amiga Operating System} or short
\emph{AmigaOs} is represented by the ROM-based
\emph{dos.library}. This library is typically opened by the startup
code of most compilers anyhow, and its base pointer is placed into
{\tt DOSBase} by this startup code:
\begin{verbatim}
struct DosLibrary *DOSBase;
\end{verbatim}
Hence, in general, there is no need to open this library manually.
\smallskip{}

The structure \emph{\mkey{DosLibrary}struct DosLibrary} is defined in
{\tt dos/dosextens.h}, but its layout and its members are usually not
required and should rather not be accessed directly. Instead, the
library provides accessor functions to read many objects contained
within it.
\smallskip{}

If you do not link with compiler startup code, the base pointer of the
\emph{dos.library} can be obtained similar to that of any other library:
\begin{verbatim}
#include <proto/exec.h>
#include <proto/dos.h>
#include <exec/libraries.h>
#include <dos/dos.h>

...
if ((DOSBase = (struct DosLibrary *)(OpenLibrary(DOSNAME,47))) {
  ...
  CloseLibrary((struct Library *)DOSBase);
}
\end{verbatim}

Unlike many other operating system, the \emph{dos.library} does not
manage disks or files itself, neither does it provide access to
hardware interface components. It rather implements a \emph{virtual
  file system} which forwards requests to its subsystems, called
\emph{handlers} or \emph{file systems}, see~\ref{sec:handlerovv}.

\section{Booleans}

AmigaDOS uses a somewhat different convention for booleans, i.e. truth
values defined in the file {\tt dos/dos.h}:

\begin{rkrmtable}{DOS Truth Values} \label{table:dosbool}
{\bf Define} & {\bf Value}\\ \hline \hline
DOSFALSE & 0  \\ \hline
DOSTRUE  & -1 \\ \hline
\end{rkrmtable}

Note that the C language instead uses the value $1$ for
{\tt TRUE}. Code that checks for zero or non-zero return codes will
function normally, however code shall not compare to {\tt TRUE} in
boolean tests.

\section{Pointers and BPTRs} \label{sec:bptrs}
AmigaDOS is a descendent of the \emph{TRIPOS system} and as such
originally implemented in the BCPL language. As of Kickstart 2.0,
AmigaDOS was re-implemented in C and assembler, but this
implementation had to preserve the existing interface based on BCPL
conventions.
\smallskip{}

BCPL is a typeless language that structures the memory of its host
system as an array of 32-bit elements enumerated contiguously from
zero up. Rather than pointers, BCPL communicates the position of its
data structures in the form of indices of the first 32-bit element of
such structures. As each 32-bit group is assigned its own index, one
can obtain this index by dividing the byte-address of an element by 4,
or equivalently, by right-shifting the address by two bits. This has
the consequence that (most) data structures passed into and out of the
dos.library shall be aligned to 32-bit boundaries. Similarly, in order
to obtain the byte-address of a BPCL structure, the index is
multiplied by 4, or left-shifted by 2 bits.

\punchline{Not on the Stack!}{Since BPCL structures must have an
  address that is divisible by 4, you should not keep such structures
  on the stack as the average compiler will not ensure long word
  alignment for automatic objects. In the absense of a dedicated
  constructor function such as
  \emph{\key{AllocDosObject()}AllocDosObject()}, a safe strategy is
  use the exec.library memory allocation functions such as
  \emph{\key{AllocMem()}AllocMem()} or
  \emph{\key{AllocVec()}AllocVec()} to obtain memory for holding
  them.}

These indices are called \emph{BCPL pointers} or short \emph{\key{BPTR}BPTR}s,
even though they are not pointers in the sense of the C language, but
rather integer numbers as indices to an array of {\tt LONG}
(i.e. 32-bit) integers. In order to communicate this fact more
clearly, the {\tt dos/dos.h} include file defines the following data
type:

\mkey{BPTR}
\begin{verbatim}
typedef long  BPTR;                 /* Long word pointer */
\end{verbatim}

Conversion from BCPL pointers to conventional C pointers and back are
formed by the following macros, also defined in {\tt dos/dos.h}:

\mkey{BADDR()} \mkey{MKBADDR()}
\begin{verbatim} 
/* Convert BPTR to typical C pointer */
#define BADDR(x)        ((APTR)((ULONG)(x) << 2))
/* Convert address into a BPTR */
#define MKBADDR(x)      (((LONG)(x)) >> 2)
\end{verbatim}

Luckely, in most cases callers of the \emph{dos.library} do not need
to convert from and to BPTRs but can rather use such ``pointers'' as
\emph{opaque values} or \emph{handles} representing some AmigaDOS
objects.
\smallskip{}

It is certainly a burden to always allocate temporary BCPL objects
from the heap, and doing so may also fragment the AmigaOs memory
unnecessarily. However, allocation of automatic objects from the stack
does not ensure long-word aligment in general. To work around this
burden, one can use a trick and instead request from the compiler a
somewhat longer object of automatic lifetime and align the requested
object manually within the memory obtained this way. The following
macro performs this trick:

\begin{verbatim}
#define D_S(type,name) char a_##name[sizeof(type)+3]; \
                       type *name = (type *)((ULONG)(a_##name+3) & ~3UL)
\end{verbatim}

It is used as follows:
\begin{verbatim}
D_S (struct FileInfoBlock, fib); 
\end{verbatim}

At this point, fib is pointer to a properly aligned {\tt struct
FileInfoBlock}, e.g. this is equivalent to
\begin{verbatim}
struct FileInfoBlock _tmp;
struct FileInfoBlock *fib = &tmp;
\end{verbatim}
except that the created pointer is properly aligned and can safely be
passed into the \emph{dos.library}.
\smallskip{}

Similar to the C language, a pointer to a non-existing element is
expressed by the special pointer value $0$. While this is called the
{\tt NULL} pointer in C, it is better to reserve another name for it
in BPCL as its pointers are rather indices. The following convention
is suggested to express an invalid \emph{BPTR}:

\begin{verbatim}
#define ZERO 0L
\end{verbatim}

Clearly, with the above convention, the BCPL {\tt \mkey{ZERO}ZERO} pointer
converts to the C {\tt \mkey{NULL}NULL} pointer and back, even though the two are
conceptionally something different: The first being the index to the
first element of the host memory array, the later the pointer to the
first address.

\section{C strings and BSTRs}
While the C language defines \emph{\mkey{String}strings} as
0-terminated arrays of {\tt char}, and AmigaOs in particular to
0-terminated arrays of {\tt UBYTE}s, that is, unsigned characters, the
BPCL language uses a different convention, namely that of a {\tt
  UBYTE} array whose first element contains the size of the string to
follow. They are not necessarily 0-terminated either. If BCPL strings
are passed into BCPL functions, or are part of BCPL data structures,
then typically in the form of a \emph{\key{BPTR}BPTR} to the 32-bit
element containing the size of the string its 8 most significant
bits. The include file {\tt dos/dos.h} provides its own data type for
such strings:

\mkey{BSTR}
\begin{verbatim}
typedef long  BSTR;                 /* Long word pointer to BCPL string  */
\end{verbatim}

Luckely, functions of the \emph{dos.library} take C strings as
arguments and perform the conversion from C strings to their BCPL
representation as \emph{BSTR}s internally, such that one rarely gets
in contact with this type of strings. They appear as part of some
AmigaDOS structures to be discussed, and as part of the interface
between the \emph{dos.library} and its handlers, e.g. file systems.
However, even though users of the \emph{dos.library} rarely come in
contact with \emph{BSTR}s themselves, the \emph{BCPL} convention has
an important consequence, namely that (most) strings handled by the
\emph{dos.library} cannot be longer than 255 characters as this is the
limit imposed by the BCPL convention.

\punchline{Lengh-Limited Strings}{Remember that most strings that are
  passed into the \emph{dos.library} are internally converted to
  \emph{BSTR}s and thus cannot exceed a length of 255 characters.}

Unfortunately, even as of the latest version of \emph{AmigaDos}, the
\emph{dos.library} is ill-prepared to take longer strings, and will
likely fail or mis-interpret the string passed in. If longer strings
are requried, e.g. as part of a \emph{\key{path}path}, it is
(unfortunately) in the responsibility of the caller to take this path
appart into components and iterate through the components manually,
see also section~\ref{sec:files}.

\section{Files}
Files\key{File} are streams of bytes together with a file pointer that
identifies the next position to be read, or the next byte position to
be filled. Files are explained in more detail in section~\ref{sec:file}.

\section{Locks}
Locks\key{Lock} are access rights to a particular object on a file
system. A locked object cannot be altered by any other
process. Section~\ref{sec:lock} provides more details on locks.

\section{Processes}
AmigaDOS is a multi-tasking system operating on top of the \emph{exec}
kernel~\cite{rkrmlib}. As such, it can operate multiple tasks at once,
where the tasks are assigned to the CPU in a round-robin fashion. A
\emph{Process}\key{Process} is an extension of an AmigaOs
\emph{Task}\mkey{Task} that includes additional state information
relevant to AmigaDOS, such as a \emph{current directory} \emph{Current
  Directory} it operates in, a \emph{default file system}\key{Default
  file system}, a \emph{console}\key{console} it is connected to, and
default input, output and error streams. Processes are explained in
more detail in section~\ref{sec:process}.

\section{Handlers and File Systems} \label{sec:handlerovv}
\emph{Handlers} are special processes that manage files on a volume, or that
input or output data to a physical device. AmigaDOS itself delegates
all operations on files to such handlers. Handlers are introduced in
section~\ref{sec:handler}.
\smallskip{}

\emph{File systems} are special handlers that organize the contents of
data carriers such as hard disks, floppies or CD-Roms in the form of
files and directories, and provides access to such objects through the
\emph{dos.library}. File systems interpret paths (see~\ref{sec:paths})
in order to locate objects such as files and directories on such data
carriers.

\chapter{Files} \label{sec:file}

\section{What are Files?}
\emph{\mkey{File}Files} are streams of sequences of bytes that can be read from
and written to, along with a file pointer that points to the next byte
to be read, or the next byte to be written or overwritten. Files may
have an \emph{End-of-File position}, beyond which the file pointer can
not advance when reading bytes from it.

\section{Interactive vs. non-Interactive Files} \label{sec:interactive}

AmigaDOS knows two types of files: \emph{Interactive} and
\emph{non-interactive} files.
\smallskip{}

\emph{Non-interactive} files are stored on some persistent data carrier;
unless modified by a process\key{Process}, the contents of such non-interactive
files does not change. They also have a defined \emph{file size}. The
file size is the number of bytes between the start of the file and the
end-of-file position, or short \emph{EOF position}\mkey{EOF}. This file size
does not change unless some process\key{Process} writes to the file, which may or
may not be the same process that reads from the file.
\smallskip{}

Examples for non-interactive files are data on a disk, such as a
floppy or a harddisk. Such files have a name, possibly a path within a
hierachical file system, and possibly multiple protection flags that
define which type of actions can be applied to a file; such flags
define whether the file can be read from, written to, and so on.
\smallskip{}

\emph{Interactive} files depend on the interaction of the computer
system with the outside world, and their contents can change due to
such interaction. Interactive files may not define a clear end-of-file
position, and an attempt to read from them or write to them may block
an indefinite amount of time until triggered by an external event.
\smallskip{}

Examples for interactive files are the console\key{Console}, where reading from it
depends on the user entering data in a console window and output
corresponds to printing to the console; or the serial port, where read
requests are satisfied by serial data arriving at the serial port and
written bytes are transmitted out of the port. The parallel port is
another example of an interactive file. Requests to read from it
result in an error condition, while writing prints data on a printer
connected to the port. Writing may block indefinitely if the printer
runs out of paper or is turned off.

\section{Paths and File Names} \label{sec:paths}
Files are identified by \emph{\mkey{Path}paths}, which are strings
from which AmigaDOS locates a process\key{Process} through which
access to the file is managed. Such a process is called the
\emph{\key{Handler}Handler} of the file, or, in case of files of on a
data carrier, also the \emph{\key{File System}File System}.  AmigaDOS
itself does not operate on files directly, but delegates such work to
its handler.
\smallskip{}

A \emph{path} is broken up into two parts: An optional device or
volume name terminated by a colon (``:''), followed by string that
identifies the file within the handler identified by the first part.
\smallskip{}

The first part, if present, is interpreted by AmigaDOS itself. It
relates to the name of a handler (or file system) of the given name,
or a known disk volume\key{Volume}, or a logical volume of the name
within the AmigaDOS \emph{device list}\key{Device List}. These concepts are
presented in further detail in section~\ref{sec:devicelist}.
\smallskip{}

The second, or only part is interpreted by the handler identfied by
the first part.

\subsection{Devices, Volumes and Assigns}
The first part of a path, up to the colon, identifies the device, the
volume or the assign a file is located in.

\subsubsection{Devices} \label{sec:deviceoverview}
A \emph{\mkey{Device name}device name} identifies the handler or file
system directly. Handlers are typically responsible for particular
hardware units within the system, for example for the first floppy
drive, or the second partition of a harddisk. For example, {\tt df0}
is the name of the handler responsible for the first floppy drive,
regardless of which disk is inserted into it.
\smallskip{}

Table~\ref{table:romdevices} lists all devices AmigaDOS mounts itself
even without a boot volume available. They can be assumed present any
time.

\begin{rkrmtable}{System defined devices} \label{table:romdevices}
{\bf Device Name} & {\bf Description}\\ \hline \hline
{\tt DF0}              & First floppy drive\\ \hline
{\tt PRT}              & Printer\\ \hline
{\tt PAR}              & Parallel port\\ \hline
{\tt SER}              & Serial port\\ \hline
{\tt CON}              & Line-interactive console\\ \hline
{\tt RAW}              & Character based console\\ \hline
{\tt PIPE}             & Pipeline between processes\\ \hline
{\tt RAM}              & RAM-based file handler\\ \hline
\end{rkrmtable}

If more than one floppy drives are connected to the system, they are
named {\tt DF1} through {\tt DF3}. If a hard disk is present, then the device
name(s) of the harddisk partitions depend on the contents of Rigid
Disk Block, see~\ref{rkrmdev}. These names can be selected upon
installation of the harddisks, e.g. through \emph{HDToolBox}.  The
general convention is to name them {\tt DH0} and following.
\medskip{}

The following device names have a special meaning and do not belong to
a particular device:

\begin{rkrmtable}{System defined devices} \label{table:specialdevices}
{\bf Name} & {\bf Description} \\ \hline \hline
{\tt *}          & the console of the current process\\ \hline  
{\tt CONSOLE}    & the console of the current process\\ \hline
{\tt NIL}        & the data sink\\ \hline
\end{rkrmtable}

The {\tt NIL}\mkey{NIL} device is a special device without a handler
that is maintained by AmigaDOS itself. Any data written into it
vanishes completely, and any attempt to read data from it results in
an end-of-file\key{EOF} condition.
\smallskip{}

The {\tt *}\mkey{* (file name)}, if used as complete path name without
a trailing colon, is the current console of the process\key{Process}, if such a
console\key{console} exists. Any data output to the file named {\tt *} will be
printed on the console. Any attempt to read from {\tt *} will wait on
the user to input data on the console, and will return such data.

\punchline{Not a wildcard!}{Unlike other operating systems, the
  asterisk {\tt *} is \emph{not} a wildcard under AmigaDOS. It rather
  identifies the current console of a process, or is used as escape
  character in AmigaDOS shell scripts}.

The {\tt CONSOLE}\mkey{CONSOLE (device)} device is the default console
of the process\key{Process}. Unlike {\tt *}, but like any other device name, it
shall be followed by a colon, and an optional job name. Such job names
form \emph{logical consoles} that are used by the shell for job
control purposes.

\punchline{Prefer the stars}{The difference between {\tt *} and {\tt
    CONSOLE} is subtle, and the former should be preferred as it
  identifies the process as part of a particular shell job. An attempt
  to output to {\tt CONSOLE:} may block the current process as it does
  not identify it properly as part of its job, but rather denotes the
  job started when creating the shell. Thus, in case of doubt, use the
  {\tt *} without any colon if you mean the console.}

Additional devices can be loaded into the system by the \emph{Mount}\key{Mount}
command, see section~\ref{sec:handlers}.

\subsubsection{Volumes} \label{sec:volumelist}
A \emph{\mkey{Volume name}volume name} identifies a particular data
carrier within a physical drive. For example, it may identify a
particular floppy disk, regardless of the drive it is inserted it. For
example, the volume name ``Workbench3.2'' relates always to the same
floppy, regardless of whether it is inserted in the first {\tt df0} or
second {\tt df1} drive.

\subsubsection{Assigns} \label{sec:assignlist}
An \emph{\mkey{Assign}assign} or \emph{logical volume} identifies a
subset of a files within a file system under a unique name. Such
assigns are created by the system or by the user helping to identify
portions of the file system containing files that are of particular
relevance for the system. For example, the assign {\tt C} contains all
commands of the boot shell, and the assign {LIBS} contains dynamically
loadable system libraries. Such assigns can be changed or redirected, and by that
the system can be instructed to take system resources from other parts of a file system,
or entirely different file systems.
\medskip{}

Assigns can be of three types: \emph{Regular assigns},
\emph{non-binding assigns} and \emph{late assigns}. \emph{Regular
  assigns} bind to a particular directory on a particular volume. If
the \emph{assign} is accessed, and the original volume the bound
directory is not available, the system will ask to insert this
particular volume, and no other volume will be accepted.
\smallskip{}

\emph{Regular assigns} can also bind to multiple directories at once,
in which case a particular file or directory within such a
\emph{multi-assign} is searched in all directories bound to the
assign. A particular use case for this is the {\tt FONTS} assign,
containing all system-available fonts. Adding another directory to
{\tt FONTS} makes additional fonts available to the system without
loosing the original ones.
\smallskip{}

\emph{Regular assigns} have the drawback that the volume remains known
to the system, and the corresponding volume icon will not vanish from
the workbench. They also require the volume to be present at the time
the assign is created.
\smallskip{}

\emph{Non-binding assigns} avoid these problems by only storing the
symbolic path the assign goes to; whenever the assign is accessed, any
volume of the particular name containing the particular path will
work. However, if this also implies that the target of the assign is
not necessarily consistent, i.e. if the assign is accessed later on,
another volume with different content will be accepted by the system.
\smallskip{}

\emph{Late assigns} are a compromize between \emph{regular assigns}
and \emph{non-binding assigns}. AmigaDOS initially only stores a
target path for the assign, but when the assign is accessed the first
time, the assign is converted to a \emph{regular assign} and thus then
binds to the particular directory of the particular volume that was
inserted at the time of the first access.
\medskip{}

Table~\ref{table:romassigns} lists the assigns made by AmigaDOS
automatically during bootstrap; except for the {\tt SYS} assign, they
all go to a directory of the same name on the boot volume. They are
all \emph{regular assigns}, except for {\tt ENVARC}, which is
\emph{late assign}.

\begin{rkrmtable}{System defined assigns} \label{table:romassigns}
{\bf Assign Name} & {\bf Description}\\ \hline \hline
{\tt C}           & Boot shell commands\\ \hline
{\tt L}           & AmigaDOS handlers and file systems\\ \hline
{\tt S}           & AmigaDOS Scripts \\ \hline
{\tt LIBS}        & AmigaOs libraries \\ \hline
{\tt DEVS}        & AmigaOs hardware drivers \\ \hline
{\tt FONTS}       & AmigaOs fonts \\ \hline
{\tt ENVARC}      & AmigaOs preferences (late)\\ \hline
{\tt SYS}         & The boot volume \\ \hline
\end{rkrmtable}

In addition to the above table, the following assigns are handled by
AmigaDOS internally and are not part of the \emph{device
  list}\key{Device list}, (see section \ref{sec:devicelist}):

\begin{rkrmtable}{System defined assigns} \label{table:specialassigns}
{\bf Assign Name} & {\bf Description}\\ \hline \hline
{\tt PROGDIR}     & Location of the executable\\ \hline
\end{rkrmtable}

Thus, {\tt PROGDIR}\mkey{PROGDIR} is the directory the currently
executed binary was loaded from. Note that {\tt PROGDIR} does not
exist in case an executable file was not loaded from disk, probably
because it was either taken from ROM or was made resident. More on
resident executables is found in section~\ref{sec:commands}.
\smallskip{}

Additional assigns can become necessary for a fully operational
system, though these assigns are created through the
\emph{Startup-sequence}\key{Startup-Sequence}, a particular AmigaDOS
script residing in the S assign which is executed by the boot shell.
Table~\ref{table:startupassigns} lists some of them.

\begin{rkrmtable}{Assigns made during bootstrap} \label{table:startupassigns}
{\bf Assign Name} & {\bf Description}\\ \hline \hline
{\tt ENV}         & Storage for active preferences and global variables\\ \hline
{\tt T}           & Storage for temporary files\\ \hline
{\tt CLIPS}       & Storage for clipboard contents\\ \hline
{\tt KEYMAPS}     & Keymap layouts\\ \hline
{\tt PRINTERS}    & Printer drivers\\ \hline
{\tt REXX}        & ARexx scripts\\ \hline
{\tt LOCALE}      & Catalogs and localization\\ \hline
\end{rkrmtable}

Additional assigns can always be made with the
\emph{Assign}\key{Assign} command, see section~\ref{sec:handlers}.

\subsection{Relative and Absolute Paths}

As introduced in section~\ref{sec:paths}, a path consists either of an
device, volume or assign name followed by a colon followed by a second
part, or the second part alone. If a device, volume or assign name is
present, such a path is said to be an \emph{absolute path} because it
identifies a location within a logical or physical volume.
\smallskip{}

If no first part is present, or if it is empty, i.e. the colon is the
first part of the path, AmigaDOS uses information from the calling
process\key{Process} to identify a suitable handler. Details on this are provided
in section~\ref{sec:process}. Such a path is called a
\emph{relative path}.
\smallskip{}

This second part is forwarded to the handler and is not interpreted by
the \emph{dos.library}. It is then within the responsibility of the
handler to interpret this path and locate a file within the data
carrier it manages, or to configure an interface to the outside world
according to this path.
\smallskip{}

In general, the \emph{dos.library} does not impose a particular syntax
on how this second part looks like. However, several support functions
of AmigaDOS implicitly define conventions file systems should follow
to make these support functions workable and it is therefore advisable
for file system implementors to follow these conventions.

\subsection{Maximum Path Length} \label{sec:pathlimit}

The \emph{dos.library} does not enforce a limit on the size of file or
directory names, except that the total length of a path including all
of its components shall not be larger than 255 characters. This is
because it is converted to a BSTR within the
\emph{dos.library}. How large a component name can be is a matter of
the file system itself. The \emph{Fast File System} includes variants
that limit file names to 30, 56 or 106 characters.
\smallskip{}

File systems typically do not report an error if the maximum file
name is exceeded; instead, the name is clamped to the maximum size
without further notice, which may lead to undesired side effects. For
example, a file system may clip or remove a trailing {\tt .info} from
a workbench icon file name without ever reporting this, resulting in
unexpected side effects. The \emph{icon.library} and
\emph{workbench.library} of AmigaOs take care to avoid such file names
and double check created objects for correct names.

\subsection{Flat vs. Hiearchical File Systems} \ref{sec:flathierarchical}
A flat file system organizes files as a single list of all files
available on a physical data carrier. For large amounts of files, such
a representation is clearly burdensome as files may be hard to find
and hard to identify.
\smallskip{}

For this reason, all file systems provided by AmigaOs are
\emph{hierarchical} and organize files in nested sets of
\emph{directories}\mkey{Directory}, where each directory contains
files or additional directories. The topmost directory of a volume forms
the \emph{root directory}\mkey{Root Directory} of this volume.
\smallskip{}

While AmigaDOS itself does not enforce a particular convention, all
file systems included in AmigaDOS follow the convention that a path
consists of a sequence of zero or more directory names separated by a
forwards slash (``/''), and a final file or directory name.

\subsubsection{Locating Files or Directories} \label{sec:locate}

When attempting to locate a particular file or directory, the
\emph{dos.library} first checks whether an absolute path name is
present. If so, it starts from the root directory on the device,
physical or logical volume identified by the device or volume name and
delegates the interpretation of the path to the handler.
\smallskip{}

Otherwise, it uses the \emph{current directory} of the calling
process\key{Process} to locate a handler responsible for the
interpretation of the path name. If this current directory is {\tt
  ZERO} (see section~\ref{sec:bptrs}), it uses the \emph{default file
  system}\key{Default file system} of the process, which by itself,
defaults to the boot file system.
\smallskip{}

The second part of the path interpretation is up to the file system
identified by the first step and is performed there, outside of the
\emph{dos.library}. If the path name includes a colon (``:''), then locating
a file starts from the root of the inserted volume. This also includes
the special case of an absent device or volume name, though a present
colon, i.e. ``:'' represents the root directory of the volume to which
the current directory belongs.
\medskip{}

The following paragraphs describe a recommended set of operations an
AmigaDOS file system should follow. A path consists of a sequence of
components separated by forward slashes (``/'').
\smallskip{}

To locate a file, the file system should work iteratively through the
path, component by component: A single isolated ``/'' without a
preceeding component indicates the parent directory of the current
directory. The parent directory of the root directory is the root
directory itself.
\smallskip{}

Otherwise, a component followed by ``/'' instructs the handler to enter
the directory of given by the component, and continue searching there.
\smallskip{}

Scanning terminates when the file system reaches the last component. The
file or directory to find is then the given by the last component
reached during the scan.
\smallskip{}

As scanning through directories starts with the current directory and
stops when the end of the path has been reached, the empty string
indicates the current directory.

\punchline{No Dots Here}{Unlike other operating systems, AmigaDOS does
  not use ``.'' and ``..'' to indicate the current directory or the
  parent directory. Rather, the current directory is represented by
  the empty string, and the parent directory is represented by an
  isolated forwards slash without a preceeding component.}

Thus, for example, ``:S'' is a file or directory named ``S'' in the
root directory of the current directory of the process, and
``//Top/Hi'' is a file or directory named ``Hi'' two directories up
from the current directory, in a directory named ``Top''.

\subsection{Case Sensitivity}

The \emph{dos.library} does not define whether file names are
case-sensitive or insensitive, except for the device or volume name
which is case-insensitive. Most if not all AmigaDOS file-systems are
also case-insensitve, or rather should. Some variants of the
\emph{Fast File System} do not handle case-insensitive comparisons
correctly on non-ASCII characters, i.e. ISO-Latin code points whose
most-significant bit is set, see section~\ref{sec:ffs} for
details. These variants should be avoided and the ``international''
variants should be preferred.

\section{Opening Files} \label{sec:open}
To read data from or write data to a file, it first needs to be opened
by the {\tt Open()}\mkey{Open()} function:

\begin{verbatim}
file = Open( name, accessMode )
D0           D1    D2

BPTR Open(STRPTR, LONG)
\end{verbatim}
The {\tt name} argument is the \emph{path} the file to be opened,
which is interpreted according to the rules given in
section~\ref{sec:paths}. The argument {\tt accessMode} identifies how
the file is opened. The function returns a {\tt BPTR} to a
\emph{file handle} on success, or {\tt ZERO} on failure. A secondary return
code can be retrieved from {\tt IoErr()} described in
section~\ref{sec:misc}. It is $0$ on success, or an error code from
{\tt dos/dos.h} in case opening the file failed.
\smallskip{}

The access mode shall be one of the following, defined in {\tt dos/dos.h}:

\begin{rkrmtable}{Access Modes for Opening Files} \label{table:openmodes}
{\bf Access Name} & {\bf Description}\\ \hline \hline
{\tt MODE\_OLDFILE}   & Shared access to existing files\\ \hline
{\tt MODE\_READWRITE} & Shared access to new or existing files\\ \hline
{\tt MODE\_NEWFILE}   & Exclusive access to new files\\ \hline
\end{rkrmtable}

\punchline{Length Limited}{As this function needs to convert the path
  argument from a C string to a BSTR, path names longer than 255
  characters are not supported and results are unpredictable if such
  names are passed into Open(). If such long path names cannot be
  avoided, it is the responsibility of the caller to split the path
  name accordingly and potentially walk through the directories
  manually if necessary. Note that this strategy may not be suitable
  for interactive files or handlers that follow conventions for the
  path name that are different from the conventions described in
  section~\ref{sec:locate}.}

The access mode {\tt MODE\_OLDFILE}\mkey{MODE\_OLDFILE} attempts to find an existing
file. If the file does not exist, the function fails. If the file
exists, it can be read from or written from, though simultaneous
access from multiple processes is possible and does not create an
error condition. If multiple processes write to the same file
simultaneously, the result is undefined and no particular order of the
write operations is imposed.
\smallskip{}

The access mode {\tt MODE\_READWRITE}\mkey{MODE\_READWRITE} first attempts to find an
existing file, but if the file does not exist, it will be created
under the name given by the last component of the path. The function
does not attempt to create directories within the path if they do not
access. Once the file is opened, access to the file is shared, even if
it has been just created. That is, multiple processes may then access
it for reading or writing. If multiple processes write to the file
simultenously, the order in which the writes are served is undefined
and depends on the scheduling of the processes.
\smallskip{}

The access mode {\tt MODE\_NEWFILE}\mkey{MODE\_NEWFILE} creates a new file, potentially
erasing an already existing file of the same name if it already
exists. The function does not attempt to create directories within the
path if they do not exist. Access to the file is exclusive, that is,
any attempt to access the file from a second process fails with an
error condition.

\punchline{No Wildcards}{The Open() function, similar to most
  \emph{dos.library} functions, does not attempt to resolve wild cards. That
  is, any character potentially reassembling a wild card, such as
  ``?'' or ``\#'' will taken as a literal and will be used as part of
  the file name. While these characters are valid, they should be
  avoided as they make such files hard to access from the Shell.}

\section{Closing Files}

The {\tt Close()}\mkey{Close()} function writes all internally
buffered data to disk and makes an exclusively opened file accessible
to other processes again.

\begin{verbatim}
success = Close( file )
D0               D1

BOOL Close(BPTR)
\end{verbatim}

The {\tt file} is a BPTR to {\tt FileHandle} identifying the file. The
return code indicates whether the file system could successfully close
the file and write back any data. If the result code is {\tt DOSFALSE},
an error code can be obtained through {\tt IoErr()} described in
section~\ref{sec:misc}. Otherwise, {\tt IoErr()} will not be altered.
\smallskip{}

Unfortunately, not much can be done if closing a file fails and no
general advise is possible how to handle this situation.
\smallskip{}

Attempting to close the {\tt ZERO} file handle returns success
immediately.

\section{Types of Files and Handlers}

As introduced in \ref{sec:interactive}, AmigaDOS distinguishes between
non-interactive files managed by file systems and interactive files
that interact with the outside world. Typically, \emph{file
  systems}\key{File System} create non-interactive files; all other
\emph{handlers}\key{Handlers} create interactive or non-interactive
files, depending on the nature of the handler.

\subsection{Obtaining the Type from a File}

A file can be either interactive, in which case attempts to read or
write data to the file may block indefinitely, or non-interactive
where the amount of available data is determined by file itself. The
{\tt IsInteractive()} function returns the nature of an already opened
file.

\begin{verbatim}
status = IsInteractive( file )
 D0                      D1

BOOL IsInteractive(BPTR)
\end{verbatim}

The {\tt IsInteractive()}\mkey{IsInteractive()} function returns
{\tt TRUE} in case the \emph{file handle} passed in is interactive, or
{\tt FALSE} in case it corresponds to a non-interactive stream of
bytes, potentially on a file system.
This function cannot fail and does not alter {\tt IoErr()}.

\subsection{Obtaining the Type from a Path}

A \emph{handler}\key{handler} that manages physical data carriers and
allows to access named files on such data carriers is a \emph{file
  system}. The {\tt IsFileSystem()}\key{IsFileSystem()} function
determines the nature of a handler given a path (see~\ref{sec:paths})
to a candidate handler.

\begin{verbatim}
result = IsFileSystem(name)
D0                     D1

BOOL IsFileSystem(STRPTR)
\end{verbatim}
The {\tt name} argument is a path that does not need to identify a
physically existing object. Instead, it is used to identify a handler
that would be responsible to such a hypothetical object regardless
whether it exists or not.
\smallskip{}

It is of advisable to provide a path that identifies the handler
uniquely, i.e. a string that is terminated by a colon (``:''). Otherwise,
the call checks whether the \emph{handler} responsible for the current
directory of the calling process is a file system.
\smallskip{}

The returned result is {\tt DOSTRUE} in case the handler identified by
the path is a file system, and as such allows access to multiple files
on a physical data carrier. Otherwise, it returns {\tt DOSFALSE}.

\section{Unbuffered Input and Output}

The functions described in this section read bytes from or write bytes
to already opened files. These functions are \emph{unbuffered}, that
is, any request goes directly to the handler. Since a request performs
necessarily a task switch from the caller to the handler managing the
file, these functions are inefficient on small amounts of data and
should be avoided. Instead, files should be read or written in larger
chunks, either by buffering data manually, or by using the buffered
I/O functions described in section~\ref{sec:buffered}.

\subsection{Reading Data}
The following function reads data from an opened file by directly
invoking the handler for performing the read:

\begin{verbatim}
actualLength = Read( file, buffer, length )
D0                   D1    D2      D3

LONG Read(BPTR, void *, LONG)
\end{verbatim}

The {\tt Read()}\mkey{Read()} function reads {\tt length} bytes from an opened
file identified by the \emph{file handle} {\tt file} into the buffer pointed
to by {\tt buffer}. The buffer is a standard C pointer, not a BPTR.
\smallskip{}

The return code {\tt actualLength} is the amount of bytes actually
read, or -1 for an error condition. A secondary return
code can be retrieved from {\tt IoErr()} described in
section~\ref{sec:misc}. It is $0$ on success, or an error code from
{\tt dos/dos.h} in case reading failed.
\smallskip{}

The amount of data read may be less data than requested by the {\tt
  length} argument, either because the \emph{EOF} position has been
reached (see section \ref{sec:interactive}) for non-interactive files,
or because the interactive source is depleted. Note that for
interactive files, the function may block indefinitely until data
becomes available.

\subsection{Testing for Availability of Data}

An issue of the {\tt Read}function is that it may block indefinitely
on an interactive file if the user does not enter any data. The {\tt
  WaitForChar()}\mkey{WaitForChar()} tests for the availability of a
character on an interactive file for limited amount of time and
returns if no data becomes available.

\begin{verbatim}
status = WaitForChar( file, timeout )
  D0                    D1    D2

BOOL WaitForChar(BPTR, LONG)
\end{verbatim}

This function waits for a maximum of {\tt timeout} microseconds for
the availability of input on {\tt file}. If data is already available,
or becomes available within this time, the function returns
{\tt DOSTRUE}. Otherwise, the function returns {\tt DOSFALSE}.
\smallskip{}

A secondary return code can be obtained from {\tt IoErr()}. If it is
$0$, the handler was able check the availablility of a byte from the
given file. Otherwise, an error code from {\tt dos/dos.h} indicates
failure of the function.
\smallskip{}

This function requires an interactive file to operate, file systems
will typically not implement this function as they do not block.

\subsection{Writing Data}
The following call writes an array of bytes unbuffered to a file,
interacting directly with the corresponding handler:

\begin{verbatim}
returnedLength =  Write( file, buffer, length )
 D0                       D1    D2      D3

LONG Write (BPTR, void *, LONG)
\end{verbatim}

The Write\mkey{Write} function writes {\tt length} bytes in the buffer
pointed to by {\tt buffer} to the \emph{file handle} given by the {\tt file}
argument. On success, it returns the number of bytes written as {\tt
  returnedLength}, and advances the file pointer of the file by this
amount. Note that this amount of bytes may even be 0 in case the file
cannot absorb any more bytes. On error, $-1$ is returned.
\smallskip{}

A secondary return code can be retrieved from {\tt IoErr()} described
in section~\ref{sec:misc}. It is $0$ on success, or an error code from
{\tt dos/dos.h} in case writing failed.
\smallskip{}

For interactive files, this function may block indefinitely until the
corresponding handler is able to take additional data.

\subsection{Adjusting the File Pointer}

The {\tt Seek()}\mkey{Seek()} adjusts the file pointer of a
non-interactive file such that subsequent reading or writing is
performed from an alternative position of the file.

\begin{verbatim}
oldPosition = Seek( file, position, mode )
 D0                  D1    D2        D3

LONG Seek(BPTR, LONG, LONG)
\end{verbatim}

This function adjusts the file pointer of {\tt file} relative to the
position determined by {\tt mode} by {\tt position} bytes. The value
of {\tt mode} shall be one of the following options, defined in
{\tt dos/dos.h}:

\begin{rkrmtable}{Seek Modes} \label{table:seekmode}
{\bf Mode Name} & {\bf Description}\\ \hline \hline
{\tt OFFSET\_BEGINNING} & Seek relative to the start of the file\\ \hline
{\tt OFFSET\_CURRENT}   & Seek relative to the current file position\\ \hline
{\tt OFFSET\_END}       & Seek relative to the end of the file\\ \hline
\end{rkrmtable}

\punchline{Undefined on Interactive Files}{The {\tt Seek} function
  will typically indicate failure if applied to interactive
  files. Some handlers may assign this function, however, a particular
  meaning. See the handler definition for details.}
  
If {\tt mode} is {\tt OFFSET\_BEGINNING}, then the new file pointer is
placed {\tt position} bytes from the start of the file, i.e. the new
file pointer is equal to {\tt position}.
\smallskip{}

If {\tt mode} is {\tt OFFSET\_CURRENT}, then {\tt position} is added
to the file pointer. That is, the file pointer is advanced if
{\tt position} is positive, or rewinded if {\tt position} is negative.
\smallskip{}

If {\tt mode} is {\tt OFFSET\_END}, then the end-of-file
position\key{EOF} is determined, and {\tt position} is added to this
position. This, in particular, requires that {\tt position} should be
negative.
\smallskip{}

The {\tt Seek()} function returns the file pointer before its
adjustment, or $-1$ in case of an error.
\smallskip{}

A secondary return code can be retrieved from {\tt IoErr()} described
in section~\ref{sec:misc}. It is $0$ on success, or an error code from
{\tt dos/dos.h} in case adjusting the file pointer failed.

\punchline{Not 64bit safe}{Unfortunately, it is not quite clear how
  {\tt Seek} operates on files that are larger than 2GB, and it is
  file system dependent how such files could be handled. {\tt
    OFFSET\_BEGINNING} can probably only reach the first 2GB of a
  larger file as the file system may interpret negative values as an
  attempt to reach a file position upfront the start of the file and
  may return an error. Similarly, {\tt OFFSET\_END} may possibly only
  reach the last 2GB of the file. Any other position within the file
  may be reached by splitting the seek into chunks of at most 2GB and
  perform multiple {\tt OFFSET\_CURRENT} seeks. However, whether such a
  strategy suceeds is pretty much file system dependent. Note in
  particular that the return code of the function does not allow to
  distinguish between a file pointer just below the 4GB barrier and an
  error condition. A zero result code of {\tt IoErr()} should be then
  used to learn whether a result of -1 indicates a file position of
  {\tt 0xffffffff} instead. Most AmigaDOS file systems may not be able
  to handle files larger than 2GB.}

Even though {\tt Seek()} is an unbuffered function, it is aware of a
buffer and implicitly flushes the file system internal buffer. That
is, it can be safely used by buffered and unbuffered functions.

\subsection{Setting the Size of a File}

The {\tt SetFileSize()}\mkey{SetFileSize()} function truncates or
extends the size of an opened file to a given size. Not all handlers
support this function.

\begin{verbatim}
newsize = SetFileSize(fh, offset, mode)
D0                    D1    D2     D3

LONG SetFileSize(BPTR, LONG, LONG)
\end{verbatim}
This function extends or truncates the size of the file identified by
the \emph{file handle} {\tt fh}; the target size is determined by the
current file pointer, {\tt offset} and the {\tt mode}. Interpretation
of {\tt mode} and {\tt offset} is similar to {\tt Seek()}, except that
the end-of-file position\key{EOF} of the file is adjusted, and not the
file pointer.
\medskip{}

The {\tt mode} shall be selected from to
table~\ref{table:seekmode}. In particular, it is interpreted as
follows:
\smallskip{}

If {\tt mode} is {\tt OFFSET\_BEGINNING}, then the file size is set to
the value of {\tt offset}, irrespectible of the current file pointer.
\smallskip{}

If {\tt mode} is {\tt OFFSET\_CURRENT}, then the new end-of-file position
is set {\tt offset} bytes relative to the current file pointer. That
is, the file is truncated if {\tt offset} is negative, and extended if
{\tt offset} is positive.
\smallskip{}

If {\tt mode} is {\tt OFFSET\_END}, the new file size is given by the
current file size plus {\tt offset}. That is, the file is extended by
{\tt offset} bytes if positive, or truncated otherwise. The value of the
current file pointer is irrelevant and ignored.
\smallskip{}

If the current file pointer of any \emph{file handle} opened on the
same file is, after a potential truncation, beyond the new
end-of-file, it is clamped to the end-of-file. They remain unchanged
otherwise.
\smallskip{}

If the file is enlarged, the values within the file beyond the
previous end-of-file position are undetermined.
\smallskip{}

The return value {\tt newsize} is the size of the file after the
adjustment, i.e. the position of the end-of-file location.

\punchline{Not 64bit safe}{Similar to {\tt Seek()},
  {\tt SetFileSize()} cannot be assumed to work properly if the (old or
  new) file size is larger than 2GB. What exactly happens if an
  attempt is made to adjust the file by more than 2GB depends on
  the file system performing the operation. A possible strategy to
  adjust the file size to a value above 2GB is to first seek to the
  closest position, potentially using multiple seeks of maximal size,
  and then perform one or multiple calls to {\tt SetFileSize()} with
  the {\tt mode} set to {\tt OFFSET\_CURRENT}. However, whether this
  strategy succeeds is file system dependent.}

\section{Buffered Input and Output} \label{sec:bufferedio}

AmigaDOS also offers buffered input and output functions that stores
data in an intermediate buffer. AmigaDOS then transfers data only in
larger chunks between the buffer and the handler, minimizing the task
switching overhead and offering better performance if data is to be
read or written in smaller units.

\punchline{Performance Improved}{While buffered I/O functions of
  AmigaOs 3.1.4 and below were designed around single-byte functions
  and thus caused massive overhead in the buffered functions described
  in this section, the functions in this section were redesigned in
  AmigaOs 3.2 and now offer significantly better
  performance. Unfortunately, the default buffer size AmigaDOS uses is
  quite small and should be significantly increased by
  {\tt SetVBuf()}\key{SetVBuf()}. A suggested buffer size is 4096 bytes
  which corresponds to a disk block of modern hard drives.}

\subsection{Buffered Read}

The {\tt FRead()}\mkey{FRead()} function reads multiple equally-sized
records from a file through a buffer, and returns the number of
records retrieved.

\begin{verbatim}
count = FRead(fh, buf, blocklen, blocks)
 D0           D1  D2     D3        D4

LONG FRead(BPTR, STRPTR, ULONG, ULONG)
\end{verbatim}

This function reads {\tt blocks} records each of {\tt blocklen} bytes
from the file {\tt fh} into the buffer {\tt buf}. It returns the
number of complete records retrieved from the file. If the file runs
out of data, the last record may be incomplete.
\smallskip{}

From AmigaOs 3.2 onwards, {\tt FRead()} first attempts to satisfy the
request from the file handle internal buffer, but if the number of
remaining bytes is larger than the buffer size, the handler will be
invoked directly for ``bursting'' the data into the target buffer,
bypassing the file buffer.
\smallskip{}

This function does not modify {\tt IoErr()}\key{IoErr()} in case the
request can be satisfied completely from the file handle buffer. It
neither returns $-1$ in case of an error. Callers should instead use
{\tt SetIoErr(0)} to clear the error state before calling this
function, and then use {\tt IoErr()} to learn if any error occured if
the number of records read is smaller than the number of records
requested.

\subsection{Buffered Write}

The {\tt FWrite()}\mkey{FWrite()} function writes multiple equally-sized
records to a file through a buffer, and returns the number of
records it could write.

\begin{verbatim}
count = FWrite(fh, buf, blocklen, blocks)
D0             D1  D2     D3        D4

LONG FWrite(BPTR, STRPTR, ULONG, ULONG)
\end{verbatim}

This function write {\tt blocks} records each of {\tt blocklen} bytes
from the buffer {\tt buf} to the file {\tt fh}. It returns the number
of complete records written to the file. On an error, the last record
written may be incomplete.
\smallskip{}

From AmigaOs 3.2 onwards, {\tt FWrite()} first checks whether the file
handle internal buffer is partially filled. If so, the file handle
interal buffer is filled from {\tt buf}. If any bytes remain to be
written, and the number of bytes is larger than the internal buffer
size, the handler will be invoked to write the data in a single block,
bypassing the buffer. Otherwise, the data will be copied to the
internal buffer.
\smallskip{}

This function does not modify {\tt IoErr()}\key{IoErr()} in case the
request can be satisfied completely by using the file handle
buffer. It neither returns $-1$ in case of an error. Callers should
instead use {\tt SetIoErr(0)} to clear the error state before calling
this function, and then use {\tt IoErr()} to learn if any error
occured if the number of records written is smaller than the number of
records passed in.

\subsection{Adjusting the Buffer}

The {\tt SetVBuf()}\mkey{SetVBuf()} function allows to adjust the
internal buffer size for buffered input/output functions such as {\tt
  FRead()}\key{FRead()} or {\tt FWrite()}\key{FWrite()}. It also sets
the buffer mode. The default buffer size is 204 characters, which is
too low for many applications.

\begin{verbatim}
error = SetVBuf(fh, buff, type, size)
D0              D1   D2    D3    D4

LONG SetVBuf(BPTR, STRPTR, LONG, LONG)
\end{verbatim}

This function sets the internal buffer of the \emph{file handle}
{\tt fh} to {\tt size} bytes. Sizes smaller than 204 characters will be
rounded up to 204. If {\tt buff} is non-{\tt NULL}, it is a pointer to
a user-provided buffer that will be used for buffering. This buffer
shall be aligned to a 32-bit boundary. A user provided buffer will not
be released when the file is closed.
\smallskip{}

Otherwise, if {\tt buff} is {\tt NULL} AmigaDOS will allocate the
buffer for you, and will also release it when the file is closed.
\smallskip{}

The {\tt type} argument identifies the type of buffering according to
Table~\ref{table:buffmode}; the modes there are defined in the include
file {\tt dos/stdio.h}.

\begin{rkrmtable}{Buffer Modes} \label{table:buffmode}
{\bf Buffer Name} & {\bf Description}\\ \hline \hline
{\tt BUF\_LINE} & Buffer up to end of line\\ \hline
{\tt BUF\_FULL} & Buffer everything\\ \hline
{\tt BUF\_NONE} & No buffering\\ \hline
\end{rkrmtable}

The buffer mode {\tt BUF\_LINE} automatically flushes the buffer when
writing a line feed (0x0a), carriage return (0x0c) or ASCII NUL (0x00)
character to the buffer, and the target file is
interactive. Otherwise, the characters remain in the buffer until it
either overflows or is flushed manually, see {\tt Flush()}\key{Flush()}.
\smallskip{}

The buffer mode {\tt BUF\_FULL} buffers all characters until the
buffer either overflows or is flushed.
\smallskip{}

The buffer mode {\tt BUF\_NONE} effectively disables the buffer and
writes all characters to the target file immediately.
\smallskip{}

On reading, {\tt BUF\_LINE} and {\tt BUF\_FULL} are equivalent and
fill the entire buffer from the file; {\tt BUF\_NONE} disables
buffering.
\medskip{}

The function returns non-zero on success, or $0$ on error. Error
conditions are either out-of-memory, an invalid buffer mode or an
invalid file handle. Unfortunately, {\tt IoErr()} is only set on an
out-of-memory condition and remains otherwise unchanged.

\subsection{Synchronize the File to the Buffer}

The {\tt Flush()}\mkey{Flush()} function flushes the internal buffer
of a \emph{file handle} and synchronizes the file pointer to the
buffer position.

\begin{verbatim}
success = Flush(fh)
D0              D1

LONG Flush(BPTR)
\end{verbatim}

Synchronizes the file pointer to the buffer, that is, if bytes were
written to the buffer, writes out buffer content to file. If bytes
were read from the file and non-read files remained in the buffer,
such bytes are dropped and the function attempts to seek back to the
position of the last read byte. This can fail for interactive files.
\smallskip{}

The return code is currently always {\tt DOSTRUE} and thus cannot be
used as an indication of error, even if not all bytes could be
written, or if seeking failed. If error detection is desired, the
caller should first use {\tt SetIoErr(0)}\key{SetIoErr()} to erase an
error condition, then call flush, and then use {\tt IoErr()} to
check whether an error occurred.

\punchline{Flush when switching between reading and writing}{The {\tt
    Flush()} function shall be called when switching from writing to a
  file to reading from the same file, or vice versa. The internal
  buffer logic is unfortunately not capable to handle this case
  correctly. Also, {\tt Flush()} shall be called when switching from
  buffered to unbuffered input/output.}

\subsection{Write a Character Buffered to a File}

The {\tt FPutC()}\mkey{FPutC()} function writes a single character to
a file, using the \emph{file handle} internal buffer.

\begin{verbatim}
char = FPutC(fh, char)
D0           D1   D2

LONG FPutC(BPTR, LONG)
\end{verbatim}

This function writes the single character {\tt char} to the \emph{file
  handle} {\tt fh}.  Depending on the buffer mode, the character and
the type of file, the character may go to the buffer first, or may
cause the buffer to be emptied. See {\tt SetVBuf()}\key{SetVBuf()} for
details on buffer modes and conditions for implicit buffer flushes.
\smallskip{}

It returns the character written, or {\tt ENDSTREAMCH} on an
error. The latter constant is defined in {\tt dos/stdio.h} and equals
to $-1$.
\smallskip{}

This function does not touch {\tt IoErr()} if the character only goes
into the internal buffer. 

\subsection{Write a String Buffered to a File}

The {\tt FPuts()}\mkey{FPuts()} function writes a {\tt NUL}-terminated
string to a file, using the \emph{file handle} internal buffer.

\begin{verbatim}
error = FPuts(fh, str)
D0            D1  D2

LONG FPuts(BPTR, STRPTR)
\end{verbatim}

This function writes the NUL-terminated (C-style) string {\tt str} to
the \emph{file handle} {\tt fh}. The terminating {\tt NUL} character
is not written.
\smallskip{}

Depending on the buffer mode, the string will first go into the
buffer, or may be written out immediately. See
{\tt SetVBuf()}\key{SetVBuf()} for details on buffer modes and conditions
for implicit buffer flushes.
\smallskip{}

This function returns $0$ on success, or {\tt ENDSTREAMCH} on an
error. The latter constant is defined in {\tt dos/stdio.h} and equals
to $-1$.

\subsection{Read a Character from a File}

The {\tt FGetC()}\mkey{FGetC()} function reads a single character from
a file through the internal buffer of the \emph{file handle}.

\begin{verbatim}
char = FGetC(fh)
D0           D1

LONG FGetC(BPTR)
\end{verbatim}

This function attempts to read a single character from the \emph{file
  handle} {\tt fh} using the buffer of the handle. If characters are
present in the buffer, the request is satisfied from the buffer first,
then the function attempts to refill the buffer from the file and
tries again.
\smallskip{}

The function returns the character read, or {\tt ENDSTREAMCH} on an
end-of-file condition or an error. The latter constant is defined in
{\tt dos/stdio.h} and equals to $-1$.
\smallskip{}

To distingish between the error and the
end-of-file case, the caller should first reset the error condition
with {\tt SetIoErr(0)}\key{SetIoErr()}, and then check {\tt IoErr()}
when the function returns with {\tt ENDSTREAMCH}.

\subsection{Read a Line from a File}

The {\tt FGets()}\mkey{FGets()} function reads a newline-terminated
string from a file, using the \emph{file handle} internal buffer.

\begin{verbatim}
buffer = FGets(fh, buf, len)
D0             D1  D2   D3

STRPTR FGets(BPTR, STRPTR, ULONG)
\end{verbatim}

This function reads a line from the \emph{file handle} into the buffer
pointed to by {\tt buf}, capable of holding {\tt len} characters.
\smallskip{}

Reading terminates either if {\tt len}-1 characters have been read, filling
up the buffer completely; or a line-feed character is found, which is
copied into the buffer; or if an end-of-file condition or an error
condition is encountered. In either event, the string is {\tt NUL}
terminated.
\smallskip{}

The function returns {\tt NULL} in case not even a single character
could be read. Otherwise, the function returns the buffer passed in.
\smallskip{}

To distinguish between the error and end-of-file condition, the caller
should first use {\tt SetIoErr(0)}\key{SetIoErr()}, and then test
{\tt IoErr()} in case the function returns {\tt NULL}.

\subsection{Revert a Single Byte Read}

The {\tt UnGetC()} function reverts a single byte read from a
stream and makes this byte available for reading again.

\begin{verbatim}
value = UnGetC(fh, ch)
D0             D1  D2

LONG UnGetC(BPTR, LONG)
\end{verbatim}

The character {\tt ch} is pushed back into the \emph{file handle} {\tt fh}
such that the next attempt to read a character from {\tt fh}
returns {\tt ch}. If {\tt ch} is $-1$, the last character read will be
pushed back. If the last read operation indicated an error or end-of-file condition,
{\tt UnGetC(fh,-1)} pushes an end-of-file condition back.
\smallskip{}

This function returns non-zero on success or $0$ if the character
could not be pushed back. At most a single character can be pushed
back after each read operation, an attempt to push back more
characters can fail.

\section{File Handle Documentation}

So far, the \emph{file handle} has been used as an opaque value bare
any meaning. However, the {\tt BPTR}, once converted to a regular
pointer, is a pointer to {\tt struct FileHandle}\mkey{FileHandle}:

\begin{verbatim}
BPTR file = Open("S:Startup-Sequence,MODE_OLDFILE);
struct FileHandle *fh = BADDR(file);
\end{verbatim}

In the following sections, this structure and its functions are documented.

\subsection{The struct FileHandle}

When opening a file via {\tt Open()}, the \emph{file handle} is allocated by
the \emph{dos.library} by going through {\tt AllocDosObject()}\key{AllocDosObject()},
and then forwarded to the file system or handler for second-level initialization. It is
documented in {\tt dos/dosextens.h} as replicated here:
\mkey{FileHandle}
\begin{verbatim}
struct FileHandle {
   struct Message *fh_Link;	 
   struct MsgPort *fh_Port;
   struct MsgPort *fh_Type;
   BPTR fh_Buf;       
   LONG fh_Pos;
   LONG fh_End;
   LONG fh_Funcs;
#define fh_Func1 fh_Funcs
   LONG fh_Func2;
   LONG fh_Func3;
   LONG fh_Args;
#define fh_Arg1 fh_Args
   LONG fh_Arg2;
};
\end{verbatim}
{\tt fh\_Link} is actually not a pointer, but an AmigaDOS internal
value that shall not be interpreted or touched, and of which one
cannot make productive use.
\smallskip{}

{\tt fh\_Port} is similarly not a pointer, but a {\tt LONG}. If it is
non-zero, the file is interactive, otherwise it is a file system.
{\tt IsInteractive()}\key{IsInteractive()} makes use of this
member. The file system or handler shall initialize this value when
opening a file and shall initialize it according to the nature of the
handler.
\smallskip{}

{\tt fh\_Type} points to the {\tt MsgPort} of the handler or file
system that implements all input and output
operations. Section~\ref{sec:handler} provides additional information
on how handlers and file systems work. If this pointer is {\tt NULL},
no handler is associated to the file handle. This is also the value
AmigaDOS will deposit here when opening a file to the {\tt NIL:}
(pseudo-)device. Attempting to {\tt Read()}\key{Read()} from this
handle results in an end-of-file situation, and calling {\tt Write()}
on such a handle does nothing, ignoring any data written.
\smallskip{}

{\tt fh\_Buf} is a {\tt BPTR} to the file handle internal buffer all
buffered I/O function documented in this section use.
\smallskip{}

{\tt fh\_Pos} is the next read or write position within this buffer.
\smallskip{}

{\tt fh\_End} is the size of the buffer in bytes.
\smallskip{}

{\tt fh\_Func1} is a function pointer that is called whenever the
buffer is to be filled through the handler. Users shall not call this
function itself, and the function prototype is intentionally not
documented.
\smallskip{}

{\tt fh\_Func2} is a second function pointer that is called whenever
the buffer is full and is to be written by the handler. Users shall
not call this function itself, and the function prototype is
intentially undocumented.
\smallskip{}

{\tt fh\_Func3} is a final function pointer that is called whenever
the file handle is closed. This function then potentially writes the
buffer content out when dirty, releases the buffer if it is
system-allocated, and finally forwards the close request to the
handler.
\smallskip{}

{\tt fh\_Arg1} is a file-system internal value the handler or file
system uses to identify the file. The interpretation of this value is
to the file system or handler, and the \emph{dos.library} does not
attempt to interpret it. The handler deposits the file identification
here when opening a file, and the \emph{dos.library} forwards it to the
handler on {\tt Read()} and {\tt Write()}. See
section~\ref{sec:handler} for details.
\smallskip{}

{\tt fh\_Arg2} is currently unused.

\subsection{String Streams}
It is sometimes useful to provide programs with (temporary) input not
coming from a file system or handler directly, even though the program
uses a file interface to access it. One solution to this problem is to
deposit the input data on the {\tt RAM} disk, then opening this file
and providing it as input to such a program. The drawback of this
approach is that additional tests are necessary to ensure that the
file name is unique, and to avoid that other than the intended program
accesses it.
\smallskip{}

AmigaDOS uses the technique documented here itself, for example to
provide the command to be executed by the {\tt Run}\key{Run}
command. There, the string stream contains the command to be run in
background, which is then provided as input file to the shell. The
{\tt System()}\key{System()} function of the \emph{dos.library} makes
use of the same trick to feed the command to be executed as input
file. Thus, even though the shell can only execute commands from a
file, AmigaDOS can generate \emph{file handles} that do not correspond
to a handler, but to a string in memory containing the commands.
\smallskip{}

The shell itself is using the same technique to pass arguments to the
commands it executes; it deposits the command arguments in the file
handle buffer of the input stream where {\tt ReadArgs()} collects them.
\medskip{}

The idea is to allocate a {\tt struct FileHandle} and initialize its
buffer to contain the string within the file. For this {\tt fh->Buf}
needs to point to the buffer containing the string, and {\tt fh->End}
needs to be its size. The function pointers in the \emph{file handle}
remain 0 such as to avoid that the \emph{dos.library} reads, writes or
flushes the buffer. The {\tt FileHandle} shall be allocated by {\tt
  AllocVec()} as the \emph{dos.library} releases the handle through
{\tt FreeVec()}.

The following program demonstrates this technique:
\begin{verbatim}
#include <exec/memory.h>
#include <dos/dos.h>
#include <dos/stdio.h>
#include <string.h>
#include <proto/dos.h>
#include <proto/exec.h>

int main(int argc,char **argv)
{
  const char *test = "Hello World!\n";
  const int len = strlen(test)+1;
  struct FileHandle *fh;
  BPTR file;

  fh   = AllocVec(sizeof(struct FileHandle) + len,MEMF_PUBLIC|MEMF_CLEAR);
  if (fh) {
    UBYTE *c = (UBYTE *)(fh + 1);
    file     = MKBADDR(fh);
    memcpy(c,test,len);
    fh->fh_Buf = MKBADDR(c);
    fh->fh_End = len;
    {
      BPTR out = Output();
      LONG ch;
      while((ch = FGetC(file)) >= 0) {
        FPutC(out,ch);
      }
    }
    Close(file);
  }
  return 0;
}
\end{verbatim}
Here the buffer is allocated along with the file handle, and thus
released along with it. Setting {\tt MEMF\_PUBLIC} is of utter
importance as it clears all function pointers, and in particular the
{\tt fh\_Link} field to zero; the latter is an indication to the
\emph{dos.library} that this structure was not allocated through
itself.

\subsection{An FSkip() Implementation}

Unlike most unbuffered functions, {\tt Seek()}\key{Seek} can be safely
mixed with buffered input and output functions. However, this function
is not very efficient, and seeking should be avoided if buffer
manipulation is sufficient. Buffer manipulation has the advantage that
small amounts of bytes can be skipped easily without going through the
file system; skipping over larger amounts of bytes can be performed by
a single function without requiring to read bytes.
\smallskip{}

The following function implements an {\tt FSkip()}\mkey{FSkip()}
function that selects the most viable option and is more efficient
that {\tt Seek()} for buffered reads.
\begin{verbatim}
LONG FSkip(BPTR file,LONG skip)
{
  LONG res;
  struct FileHandle *fh = BADDR(file);
  if (fh->fh_Pos >= 0 && fh->fh_End > 0 && fh->fh_Func3) {
    LONG newpos = fh->fh_Pos + skip;
    if (newpos >= 0 && newpos < fh->fh_End) {
      fh->fh_Pos = newpos;
      return DOSTRUE;
    }
  }
  skip += fh->fh_Pos - fh->fh_End;
  fh->fh_Pos = -1;
  fh->fh_End = -1;
  if (Seek(fh,skip,OFFSET_CURRENT) != -1)
   return DOSTRUE;
  return DOSFALSE;
}
\end{verbatim}

The first if-condition checks whether the buffer is actually
present. Then, the new buffer position is computed. If it is within
the buffer, the new buffer position is installed as the work is done.
\smallskip{}

Otherwise, the skip distance is adjusted by the buffer
position. Initializing the buffer size and position to $-1$ ensures
that the following {\tt Seek()} does not attempt to call {\tt Flush()}
internally.
\smallskip{}

There is one particular catch, namely that the {\tt file}
needs to be initialized for reading immediately after opening the
file, or the buffer will not be in the right state for the trick:
\begin{verbatim}
BPTR file = Open(filename,MODE_OLDFILE);
UnGetC(file,-1); /* initialize buffer */
\end{verbatim}
This is only necessary if the first access to the file is an
{\tt FSkip()}.

\section{Formatted Output}

The functions in this section print strings formatted to a file. Both
files use the internal buffer of the \emph{file handle}.

\subsection{Print Formatted String using C-Syntax}

The {\tt VFPrintf()}\mkey{VFPrintf()} function prints multiple
datatypes using a format string that closely reessembles the syntax of
the C syntax. {\tt FPrintf()}\mkey{FPrintf()} is based on the same
entry point of the \emph{dos.library}, though the prototype for the C
language is different and thus arguments are expected directly as function
arguments instead of requiring them to be collected in an array upfront.

\begin{verbatim}
count = VFPrintf(fh, fmt, argv)
D0               D1  D2    D3

LONG VFPrintf(BPTR, STRPTR, LONG *)

count = FPrintf(fh, fmt, ...)

LONG FPrintf(BPTR, STRPTR, ...)
\end{verbatim}

This function uses the {\tt fmt} string to format an array of
arguments pointed to by {\tt argv} and outputs the result to the file
{\tt fh}. The syntax of the format string is identical to that of the
exec function {\tt RawDoFmt()}, and shares its problems. In
particular, format strings indicating integer arguments such as
{\tt \%d} and {\tt \%u} assume 16bit integers, independent of the integer
model of the compiler. On compilers working with a 32bit integer
models, the format modifier {\tt l} should be used, e.g. {\tt \%ld}
for signed and {\tt \%lu} for unsigned integers.
\smallskip{}

As {\tt RawDoFmt()}\key{RawDoFmt()} is also patched by the
\emph{locale.library}, additional syntax elements from the
{\tt FormatString()} function of this library become available for
{\tt VFPrintf()} and {\tt FPrintf()}.
\smallskip{}

The result {\tt count} delivers the number of characters written to
the file, or $-1$ for an error. In the latter case,
{\tt IoErr}\key{IoErr()} provides an error code.

\subsection{BCPL Style Formatted Print to a File}

The {\tt VFWritef()}\mkey{VFwrite()} function formats several
arguments according to a format string similar to
{\tt VFPrintf()}\key{VFprintf()}, but uses the formatting syntax of the
BCPL language. The main purpose of this function is to offer formatted
output for legacy BCPL programs where this function appears as an
entry of the BCPL \emph{Global Vector}. New code should not use this
function but rather depend on {\tt VFPrintf()} which also gets enhanced
by the \emph{locale.library}.
\smallskip{}

The {\tt FWritef()}\mkey{FWritef()} uses the same entry point of the
\emph{dos.library}, though the compiler prototype imposes a different
calling syntax where the objects to be formatted are directly
delivered as function arguments rather requiring the caller to collect
them in an array upfront.

\begin{verbatim}
count = VFWritef(fh, fmt, argv)
D0               D1  D2    D3

LONG VFWritef(BPTR, STRPTR, LONG *)

count = FWritef(fh, fmt, ...)

LONG FWritef(BPTR, STRPTR, ...)
\end{verbatim}

This function formats the arguments from the array pointed to by
{\tt argv} according to the format string in {\tt fmt} and writes the
output to the file {\tt fh}. The format string follows the syntax of
the BCPL language. The following format identifiers are supported:

\begin{itemize}
\item[{\tt \%S }] Write a {\tt NUL} terminated string from the array to the output.
\item[{\tt \%Tx}] Writes a {\tt NUL} terminated string left justified
  in a field whose width is given by the character {\tt x}. The length
  indicator is always a single character; a digit from {\tt 0} to
  {\tt 9} indicates the field widths from $0$ to $9$ directly. Characters
  {\tt A} to {\tt Z} indicate field widths from $10$ onwards.
\item[{\tt \%C }] Writes a single character whose ISO-Latin-1 code is given
  as a 32-bit integer on the {\tt argv} array.
\item[{\tt \%Ox}] Writes an integer in octal to the output where
  {\tt x} indicates the maximal field width. The field width is a single
  character that is encoded similarly to the {\tt \%T} format string.
\item[{\tt \%Xx}] Writes an integer in hexadecimal to the output in a field
  that is at most {\tt x} characters long. {\tt x} is a single character and
  encodes the width similar to that {\tt \%T} format string.
\item[{\tt \%Ix}] Writes a (signed) integer in decimal to the output
  in field that is at most {\tt x} characters long. The field length
  is again indicated by a single character.
\item[{\tt \%N }] Writes a (signed) integer in decimal to the output
  without any length limitation.
\item[{\tt \%Ux}] Writes an unsigned integer in decimal to the output,
  limiting the field length to at most {\tt x} characters, where
  {\tt x} is encoded in a single character.
\item[{\tt \%\$ }] Ignores the next argument, i.e. skips over it.
\end{itemize}

This function is \emph{not} patched by the \emph{locale.library} and
therefore is not localized or enhanced.
\smallskip{}

While the same function can also be found in the BPCL
\emph{Global Vector}, it there takes BSTRs instead of regular C strings for the
format string and arguments of the {\tt \%S} and {\tt \%T} formats.
  
\chapter{Locks} \label{sec:locks}

\emph{Locks}\mkey{Lock} are access rights to objects, such as files or
directories, on a file system. Once an object has been locked, it can
no longer be deleted, or in case of files, it can no longer altered
either. Depending on the file system, locks may also prevent other
forms of changes of the object.
\medskip{}

Locks come in two types: \emph{Exclusive}\mkey{Lock (Exclusive)} and
\emph{shared locks}. Only a single exclusive lock can exist on a file
system object at a time, and no other locks on an exclusively locked
object can exist. An attempt to lock an exclusively locked object
results in failure, and attempting to exclusively lock an object that
is already shared locked will also fail.
\smallskip{}

Multiple \emph{shared locks}\mkey{Lock (Shared)} can be kept on the
same object at the same time, though once a shared lock has been
obtained, any attempt to lock the same object exclusively fails.
\smallskip{}

One particular use case of \emph{locks} is to serve as an identifier
of a particular directory or file on a file system. Since paths are
limited to 255 characters, see~\ref{sec:paths}, locks are the
preferred method of indicating a position within a file system. Even
though paths are length limited, there is no restriction on the depth
within the directory structure of a file system. The {\tt ZERO} lock
identifies the boot volume, also known as {\tt SYS:}, see also
section~\ref{sec:assignlist}.
\smallskip{}

\emph{Locks} are also the building stone of files; in fact, every file
is internally represented by a lock on the corresponding object, even
if the file system does not expose this lock to the caller.
\smallskip{}

As long as at least a single lock is held of an object on a particular
volume, the file system will keep the volume within the
\emph{device list} of the \emph{dos.library}, see section~\ref{sec:devicelist}.
This has, for example, the consequence that the workbench will
continue to show an icon representing the volume in its window.

\section{Obtaining and Releasing Locks}

\emph{Locks} can be obtained either explictly from a path, or can be
derived from another lock or file. As locks block altering accesses to
an object of a file system, locks need to be released as early as
possible to allow other accesses to the locked object.

\subsection{Obtaining a Lock from a Path} \label{sec:lock}.

The {\tt Lock()}\key{Lock()} function obtains a lock on an object
given a path to the object. The path can be either absolute, or
relative (see section~\ref{sec:paths}) to the current directory of the
calling process.

\begin{verbatim}
lock  = Lock( name, accessMode )
D0            D1        D2

BPTR Lock(STRPTR, LONG)
\end{verbatim}

This function locks the object identified by {\tt name}, which is the
path to the object. The type of the lock is identified by {\tt
  accessMode}. This mode shall be one of the two following modes,
defined in {\tt dos/dos.h}:

\begin{rkrmtable}{Lock Access Modes} \label{table:lockmodes}
{\bf Access Mode} & {\bf Description}\\ \hline \hline
{\tt SHARED\_LOCK} & Lock allowing shared access from multiple sources\\ \hline
{\tt ACCESS\_READ} & Synonym of the above, identical to {\tt SHARED\_LOCK} \\ \hline
{\tt EXCLUSIVE\_LOCK} & Exclusive lock, only allowing a single lock on the object \\ \hline
{\tt ACCESS\_WRITE} & Synonym of the above, identical to {\tt EXCLUSIVE\_LOCK} \\ \hline
\end{rkrmtable}

The access mode {\tt SHARED\_LOCK}\mkey{SHARED\_LOCK} or {\tt
  ACCESS\_READ}\mkey{ACCESS\_READ} allows multiple shared locks on the
same object. This type of lock should be preferred. The access mode
{\tt EXCLUSIVE\_LOCK}\mkey{EXCLUSIVE\_LOCK} or
{\tt ACCESS\_WRITE}\mkey{ACCESS\_READ} only allows a single, exclusive
lock on the same object.
\smallskip{}

The return code {\tt lock} identifies the lock. It is non-{\tt ZERO}
(see~\ref{sec:bptrs}) on success, or {\tt ZERO} on failure. In either
case, {\tt IoErr()}\key{IoErr()} is set to $0$ indicating success, or
an error code on failure.
\smallskip{}

\punchline{No Wildcards Here!}{Note that this function does not attempt to
  resolve wild cards, similar to {\tt Open()}. All characters in the
  path are literals.}

\subsection{Duplicating a Lock}

The {\tt DupLock()}\mkey{DupLock()} function replicates a given
\emph{lock}, returning a copy of the \emph{lock} given as
argument. This requires that the original \emph{lock} is a
\emph{shared lock}, and it returns a \emph{shared lock} if successful.

\begin{verbatim}
lock = DupLock( lock )
D0              D1

BPTR DupLock(BPTR)
\end{verbatim}

This function copies the (shared) lock passed in as {\tt lock} and
returns a copy of it in {\tt lock}. In case of error, it returns {\tt
  ZERO}, and then {\tt IoErr()}\key{IoErr()} returns an error code
identifying the error. On success, {\tt IoErr()} is reset. It is not
possible to copy an \emph{exclusive lock}.

\subsection{Obtaining the Parent of an Object}

The {\tt ParentDir()}\mkey{ParentDir()} function obtains a
\emph{shared lock} on the directory containing the locked object
passed in. For directories, this is the parent directory, for files,
this is the directory containing the file.

\begin{verbatim}
newlock = ParentDir( lock )
D0                   D1

BPTR ParentDir(BPTR)
\end{verbatim}

The {\tt lock} argument identifies the object whose parent is to be
found; the function returns a \emph{lock} on the directory containing
the object. If such parent does not exist, or an error occurs, the
function returns {\tt ZERO}. The former case applies to the topmost
directory of a file system, or the {\tt ZERO} lock itself.
\smallskip{}

To distinguish the two cases, the caller should check the {\tt
  IoErr()}\key{IoErr()} function; if this function returns $0$, then
no error occurred and the passed in object is topmost and no parent
exists. If it returns a non-zero error code, then the file system
failed to identify the parent directory.

\subsection{Creating a Directory}

The {\tt CreateDir()} object creates a new empty directory whose name
is given by the last component of the path passed in. It does not
create any intermediate directories between the first component of the
path and its last component, such directories need potentially be
created manually by multiple calls to this function.

\begin{verbatim}
lock = CreateDir( name )
 D0                D1

BPTR CreateDir(STRPTR)
\end{verbatim}

The {\tt name} argument is the path to the new directory to be
created; that is, the directory given by the last component of the
path (see section~\ref{sec:paths}) will be created. If successful,
the function returns an \emph{exclusive lock} in {\tt lock}, otherwise
it returns {\tt ZERO}.
\smallskip{}

In either case, {\tt IoErr()}\key{IoErr()} is set to either an error
code, or to $0$ in case the function succeeds.
\smallskip{}

Note that not all file systems support directories, i.e. flat file
systems (see section~\ref{sec:flathierarchical}) do not.

\subsection{Releasing a Lock}

Once you are done with a \emph{lock} and no part of your program is
using it anymore, you should release it to allow other processes or
functions to access or modify the locked object. Note that setting the
{\tt CurrentDir()}\key{CurrentDir()} to a particular lock implies
usage of the lock, i.e. the lock installed as {\tt CurrentDir()} shall
not be unlocked.

\begin{verbatim}
UnLock( lock )
         D1

void UnLock(BPTR)
\end{verbatim}

This function releases the \emph{lock} passed in as {\tt lock}
argument. Passing {\tt ZERO} as a lock is fine and performs no
activity.

\subsection{Changing the Type of a Lock}

Once a \emph{lock} has been granted, it is possible to change the
nature of the lock, either from
{\tt EXCLUSIVE\_LOCK}\key{EXCLUSIVE\_LOCK} to
{\tt SHARED\_LOCK}\key{SHARED\_LOCK}, or --- if this is the only \emph{lock} on
the object --- vice versa.

\begin{verbatim}
success = ChangeMode(type, object, newmode)
D0                    D1     D2      D3

BOOL ChangeMode(ULONG, BPTR, ULONG)
\end{verbatim}

This function changes the access mode of {\tt object} whose type is
identified by {\tt type} to the access mode {\tt newmode}. The
relation between {\tt type} and the nature of the object shall be as
in table~\ref{table::changemode}, where the types are defined
in {\tt dos/dos.h}:

\begin{rkrmtable}{Object Types for ChangeMode()} \label{table::changemode}
{\bf {\tt \bf type}} & {\bf {\tt \bf object} Type}\\ \hline \hline
{\tt CHANGE\_LOCK} & {\tt object} shall be a \emph{lock}\\ \hline
{\tt CHANGE\_FH} & {\tt object} shall be a \emph{file handle}\\ \hline
\end{rkrmtable}

The argument {\tt newmode} shall be one of the modes indicated in
Table~\ref{table:lockmodes}, i.e. {\tt SHARED\_LOCK}\key{SHARED\_LOCK}
to make either the file or the lock accessible for shared access, and
{\tt EXCLUSIVE\_LOCK}\key{SHARED\_LOCK} for exclusive access.

On success, the function returns a non-zero result code, and
{\tt IoErr()} is set to $0$. Otherwise, the function returns $0$ and sets
{\tt IoErr()}\key{IoErr()} to an appropriate error code.
\smallskip{}

Unfortunately, this function may not work reliable for \emph{file
  handles} under all versions of AmigaDOS. In particular, the
\emph{RAM-Handler} does not interpret {\tt newmode} correctly for {\tt
  CHANGE\_FH}.

\subsection{Comparing two Locks}

The {\tt SameLock()}\mkey{SameLock()} function compares two locks and
returns information whether they are identical, or at least correspond
to objects on the same volume.

\begin{verbatim}
value = SameLock(lock1, lock2)
 D0                D1     D2

LONG SameLock(BPTR, BPTR)
\end{verbatim}

This function compares {\tt lock1} with {\tt lock2}. The return code,
all of them defined in {\tt dos/dos.h}, can be one of the following:

\begin{rkrmtable}{Lock Comparison Return Code} \label{table:lockcompare}
{\bf Return Code} & {\bf Description}\\ \hline \hline
{\tt SAME\_LOCK} & Both locks are on the same object\\ \hline
{\tt SAME\_VOLUME} & Locks are on different objects, but on the same volume\\ \hline
{\tt LOCK\_DIFFERENT} & Locks are on different volumes\\ \hline
\end{rkrmtable}

This function does not set {\tt IoErr()}\key{IoErr()} consistently,
and callers cannot depend on its value. Furthermore, the function does
not compare a {\tt ZERO} lock with lock on the boot volume, e.g {\tt
  SYS:} as identical. It is recommended not to pass in the {\tt ZERO}
lock for either {\tt lock1} or {\tt lock2}.

\section{Locks and Files}

Each \emph{file handle} is associated to a lock to the file that has
been opened. The type of the \emph{lock} depends on the access mode
the file has been opened with, table~\ref{table:modelock} for how lock
types and access modes relate.

\begin{rkrmtable}{Lock and File Access Modes} \label{table:modelock}
{\bf Access Mode} & {\bf Lock Type}\\ \hline \hline
{\tt MODE\_OLDFILE} & {\tt SHARED\_LOCK} \\ \hline
{\tt MODE\_READWRITE} & {\tt SHARED\_LOCK} \\ \hline
{\tt MODE\_NEWFILE} & {\tt EXCLUSIVE\_LOCK} \\ \hline
\end{rkrmtable}

The association of {\tt MODE\_READWRITE} to {\tt SHARED\_LOCK} is
unfortunate, and due to a defect in the \emph{RAM-Handler}
implementation in AmigaDOS 2.0 which was then later copied into the
\emph{Fast File System} implementation. Exclusive access to a file
without deleting its contents can, however, be established through the
{\tt OpenFromLock()}\key{OpenFromLock()} function passing in an
\emph{exclusive lock} to the function as argument.

\subsection{Duplicate the Implicit Lock of a File}

The {\tt DupLockFromFH()}\mkey{DupLockFromFH()} function performs a copy of a lock implicit
to a \emph{file handle} of an openend file. For this to succeed, the
file must be opened in the mode {\tt MODE\_OLDFILE}\key{MODE\_OLDFILE}
or {\tt MODE\_READWRITE}\key{MODE\_READWRITE}. Files openend with {\tt
  MODE\_NEWFILE}\key{MODE\_NEWFILE} are based on an implicit \emph{exclusive
lock} that cannot be copied.

\begin{verbatim}
lock = DupLockFromFH(fh)
D0                   D1

BPTR DupLockFromFH(BPTR)
\end{verbatim}

This function returns a copy of the lock the \emph{file handle} {\tt
  fh} is based on and returns it in {\tt lock}. In case of failure,
{\tt ZERO} is returned. In either case, {\tt IoErr()} is set to either
$0$ in case of succes, or an error code on failure.

\subsection{Obtaining the Directory a File is Located in}

The {\tt ParentOfFH()}\mkey{ParentOfFH()} function obtains a
\emph{shared lock} on the parent directory of the file associated to
the \emph{file handle} passed in. That is, it is roughly equivalent to
first obtaining a lock on the file through {\tt DupLockFromFile()}
\key{DupLockFromFile()}, and then calling {\tt ParentDir()}\key{ParentDir()}
on it, except that this function also applies to files opened in
the {\tt MODE\_NEWFILE}\key{MODE\_NEWFILE} mode.

\begin{verbatim}
lock = ParentOfFH(fh)
 D0               D1

BPTR ParentOfFH(BPTR)
\end{verbatim}

This function returns in {\tt lock} a shared lock on the directory
containing the file opened through the {\tt fh} \emph{file handle}. It
returns {\tt ZERO} on failure. In either case, {\tt IoErr()}
\key{IoErr()} is set, namely to $0$ in case of success or to an error
code on failure.

\subsection{Opening a File from a Lock}

The {\tt OpenFromLock()} function uses a \emph{lock} and opens the
locked file, returning a \emph{file handle}. If the lock is associated
to a directory, the function fails. The \emph{lock} passed in is then
absorbed into the \emph{file handle} and shall not be unlocked. It
will be released by the file system upon closing the file.

\begin{verbatim}
fh = OpenFromLock(lock)
D0                 D1

BPTR OpenFromLock(BPTR)
\end{verbatim}

This function attempts to open the object locked by {\tt lock} as
file, and creates the \emph{file handle} {\tt fh} from it. It fails in
case the {\tt lock} argument belongs to a directory and not a file.
\smallskip{}

In case of success, the \emph{lock} becomes an implicit part of the
\emph{file handle} and shall not be unlocked by the caller anymore.
In case of failure, the function returns {\tt ZERO} and the
\emph{lock} remains available to the caller, and also needs to be
unlocked at a later time. In either case, {\tt IoErr()} is set, to an
error code in case of failure, or $0$ on success.
\smallskip{}

This function allows to open files in exclusive mode without deleting
its contents. For that, obtain an \emph{exclusive lock} on the file to
be opened, and then call {\tt OpenFromLock()} as second step.

\subsection{The struct FileLock} \label{sec:filelock}

\emph{Locks} have been so far been opaque identifiers; in fact, they
are \emph{BPTR}s to a {\tt struct FileLock} that is defined in {\tt
  dos/dosextens.h}.

\begin{verbatim}
#include <dos/dosextens.h>
lock = Lock("S:Startup-Sequence",SHARED_LOCK);
struct FileLock *flock = BADDR(lock);
\end{verbatim}

While this structure is defined there, it is not allocated by the
\emph{dos.library} but by the \emph{file system} itself. The file
system may therefore allocate a structure that is somewhat larger and
can have additional members that are not shown here. 

\mkey{FileLock}
\begin{verbatim}
struct FileLock {
    BPTR                fl_Link;        /* bcpl pointer to next lock */
    LONG                fl_Key;         /* disk block number */
    LONG                fl_Access;      /* exclusive or shared */
    struct MsgPort *    fl_Task;        /* handler task's port */
    BPTR                fl_Volume;      /* bptr to DLT_VOLUME DosList entry */
};
\end{verbatim}

Most of the members of this structure are of no practical value, and
they should not be interpreted in any way. What is listed here is the
information callers can depend upon.
\smallskip{}

The {\tt fl\_Link} member has no practical value for users; the
\emph{file system} can use it to keep multiple links on object on the
same volume in a list. This is particularly important if the volume is
ejected from its drive and another file system needs to take over the
\emph{locks} if the volume is later inserted into another drive.
\smallskip{}

The {\tt fl\_Key} member can be used by the file system to identify the
object that has been locked. It may not necessarily be an integer, but
can be any data type, potentially a pointer to some internal
management object. It shall not be interpreted in any particular way.
\smallskip{}

The {\tt fl\_Access} member keeps the type of the lock. It is either {\tt SHARED\_LOCK}\key{SHARED\_LOCK}
or {\tt EXCLUSIVE\_LOCK}\key{EXCLUSIVE\_LOCK}.
\smallskip{}

The {\tt fl\_Task} member points to the message port of the file
system for processing requests on the lock. Any activity on the lock
goes through this port.
\smallskip{}

The {\tt fl\_Volume} is a \emph{BPTR} to the \emph{volume node} on the
\emph{Device list}\key{Device list}. The \emph{volume node} identifies
the volume the locked object is located
on. Section~\ref{sec:devicelist} provides further information on this
list and its entries.

\chapter{Working with Directories}

As objects on a file system can be identified by a name, these names
need to be stored somewhere on the data carrier. This object is called
a \emph{directory}. While a flat file system only contains a single,
topmost directory which then contains all files, a directory of a
hierarchical file system can contain other directories, thus creating
a \emph{tree} of nested objects, see also
section~\ref{sec:flathierarchical}.
\smallskip{}

AmigaDOS provides functions to list the directory contents, to move
objects in the file system hierarchy or change their name, and to
access adjust their metadata, such as comments, protection bits, or
creation dates.
\smallskip{}

AmigaDOS also supports \emph{links}, that is, entries in the file
system that point to some other object in the same, or some other file
system. Therefore, links circumvent the hierarchy otherwise imposed by
the tree structure of the file system.

\section{Examining Objects on File Systems} \label{sec:fib}

Given a lock on a file or a directory, further information on such an
object can be requested by the {\tt Examine()}\key{Examine()} function
of the \emph{dos.library}. To read multiple directory entries at once
and minimizing the calling overhead, {\tt ExAll()}\key{ExAll()}
provides an advantage that is, however, harder to use, but also
provides options to filter entries.

\punchline{May go away while you look!}{As AmigaDOS is a multitasking operating
system, the directory may change under your feed while scanning; in
particular, entries you received through the above functions may not
be up to date, may have been deleted already when the above functions
return, or new entries may have been added the current scan will not
reach. While a \emph{Lock} on a directory prevents that this directory
goes away, it does \emph{not} prevent other processes to add or remove
objects to this directory, so beware.}

While {\tt ExAll()} seems to provide an advantage by reading multiple
directory entries in one go, the AmigaOS ROM file system does usually
not profit from this feature, at least not unless a directory cache is
used. The latter has, however, other drawbacks and should be avoided
for different reasons, see section~\ref{sec:ffs}.
Actually, {\tt ExAll()} is (even more) complex to implement, and it is
probably not surprising that multiple file systems have issues. The
{\tt dos.library} provides an {\tt ExAll()} implementation for those
file systems that do not implement it themselves, but even this
(ROM-based) implementation had issues in the past. Therefore,
{\tt ExAll()} has probably less to offer than it seems.
\smallskip{}

{\tt Examine()} and {\tt ExNext()} fill a {\tt FileInfoBlock}\mkey{FileInfoBlock}
structure that collects information on an examined object in a directory.
It is defined in {\tt dos/dos.h} and reads as follows:
\begin{verbatim}
struct FileInfoBlock {
   LONG   fib_DiskKey;
   LONG   fib_DirEntryType;  /* Type of Directory. If < 0, then a plain file.
                              * If > 0 a directory */
   char   fib_FileName[108]; /* Null terminated. Max 30 chars used for now */
   LONG   fib_Protection;    /* bit mask of protection, rwxd are 3-0.      */
   LONG   fib_EntryType;
   LONG   fib_Size;          /* Number of bytes in file */
   LONG   fib_NumBlocks;     /* Number of blocks in file */
   struct DateStamp fib_Date;/* Date file last changed */
   char   fib_Comment[80];  /* Null terminated comment associated with file */

   /* Note: the following fields are not supported by all filesystems.  */
   /* They should be initialized to 0 sending an ACTION_EXAMINE packet. */
   /* When Examine() is called, these are set to 0 for you.             */
   /* AllocDosObject() also initializes them to 0.                      */
   UWORD  fib_OwnerUID;         /* owner's UID */
   UWORD  fib_OwnerGID;         /* owner's GID */
   char   fib_Reserved[32];
}; /* FileInfoBlock */

\end{verbatim}
The meaning of the members of this structure are as follows:
\medskip{}

{\tt fib\_DiskKey} is a file system internal identifier of the
object. It shall not be used, and programs shall not make any
assumptions on its meaning.
\smallskip{}

{\tt fib\_DirEntryType} identifies the type of an object. Object types
are defined in {\tt dos/dosextens.h}, replicated in
table~\ref{table:direntrytypes}:

\begin{rkrmtable}{Directory Entry Types} \label{table:direntrytypes}
{\bf Value of {\tt \bf fib\_DirEntryType}} & {\bf Description}\\ \hline \hline
{\tt ST\_SOFTLINK} & Object is a soft link to another object \\ \hline
{\tt ST\_LINKDIR} & Object is a hard link to a directory \\ \hline
{\tt ST\_LINKFILE} & Object is a hard link to a file \\ \hline
\end{rkrmtable}
All other types $>0$ indicate directories, and all other types $<0$
indicate files. Section~\ref{sec:link} provides more details on soft
links and hard links.
\smallskip{}

{\tt fib\_FileName} is the name of the object as {\tt NUL} terminated
string.
\smallskip{}

{\tt fib\_Protection} are the protection bits of the object. It
defines which operations can be performed on it. The following
protection bits are currently defined in {\tt dos/dos.h}:
\begin{rkrmtable}{Protection Bits} \label{table:protectionbits}
{\bf Protection Bits} & {\bf Description}\\ \hline \hline
{\tt FIBB\_DELETE} & If this bit is $0$, the object can be deleted.\\ \hline
{\tt FIBB\_EXECUTE} & If this bit is $0$, the file is an executable binary.\\ \hline
{\tt FIBB\_WRITE} & If this bit is $0$, the file can be written to.\\ \hline
{\tt FIBB\_READ} & If this bit is $0$, the file content can be read. \\ \hline
{\tt FIBB\_ARCHIVE} & This bit is set to $0$ on every write access. \\ \hline
{\tt FIBB\_PURE} & If $1$, the executable is reentrant and can be made resident. \\ \hline
{\tt FIBB\_SCRIPT} & If $1$, the file is a script. \\ \hline
{\tt FIBB\_HOLD} & If $1$, the executable is made resident on first execution. \\ \hline
\end{rkrmtable}
The flags {\tt FIBB\_DELETE} to {\tt FIBB\_READ} are shown inverted in
the output of most tools, i.e. they are shown active if the
corresponding flag is $0$, i.e. a particular protection function is
\emph{not} active.
\smallskip{}

The {\tt FIBB\_EXECUTE} flag is only interpreted by the \emph{Shell}
(see section~\ref{sec.shell}); if the bit is $1$, the \emph{Shell}
refuses to load the file as command.
\smallskip{}

The {\tt FIBB\_ARCHIVE} flag is typically used by archival
software. Such software will set this flag upon archiving the flag,
whereas the file system will reset the flag when writing to a file, or
when creating new files. The archiving software is thus able to learn
which files had been altered since the last backup.
\smallskip{}

The {\tt FIBB\_PURE} flag inidicates an additional property of
executable binaries; if the flag is set, the binaries do not alter
their segments and their code can be loaded in \emph{RAM} and stay
there to be executed from multple processes in parallel. This avoids
loading the binary multiple times. The \emph{Shell} command
{\tt resident} can load such binaries into \emph{RAM} for future usage.
\smallskip{}

The {\tt FIBB\_SCRIPT} flag indicates whether a file is a \emph{Shell}
or an \emph{ARexx} script. If this flag is set, and the script is
given as command to the \emph{Shell}, it will forward this file to a
suitable script interpreter, such as \emph{ARexx} or {\tt Execute}.
\smallskip{}

The {\tt FIBB\_HOLD} flag indicates whether a command shall be made
resident upon loading it the first time. If the flag is $1$, and the
shell loads the file as executable binary, and the {\tt FIBB\_PURE}
bit is also set, the file is kept in \emph{RAM} and stays there for
future execution.
\medskip{}

The {\tt fib\_EntryType} member shall not be used; it can be identical
to the {\tt fib\_DirEntryType}, but its use is not documented.
\smallskip{}

The {\tt fib\_Size} member indicates the size of the file in bytes. It
should have probably be defined as an unsigned type. Its value is
undefined for directories.
\smallskip{}

The {\tt fib\_NumBlocks} member indicates now many blocks a file
occupies on the storage medium, if such a concept applies. Disks and
harddisk organize their storage into blocks of equal size, and the
file system manages these blocks to store data on the medium. The
number of blocks can be meaningless for directories.
\smallskip{}

The {\tt fib\_Date} member indicates when the file was changed last;
depending on the file system, the date may also indicate when the last
modification was made for a directory, such as creating or deleting a
file within. Which operations exactly trigger a change of a directory
is file system dependent. The {\tt DateStamp}\key{DateStamp} structure is specified
in section~\ref{sec:datestamp}.
\smallskip{}

The {\tt fib\_Comment} member contains a {\tt NUL} terminated string
to a comment on the file. Not all file systems support comments. The
comment has no particular meaning, it is only shown by some
\emph{Shell} commands or utilities and can be set by the user.
\smallskip{}

The {\tt fib\_OwnerUID} and {\tt fib\_OwnerGID} are filled in by some
multi-user aware file systems. The AmigaDOS ROM file systems do not
support these fields, and no provision is made to moderate access to a
particular file according to an owner or its group. The two concepts
are alien to AmigaDOS itself.
\smallskip{}

The {\tt fib\_Reserved} field is currently unused and shall not be
accessed.

\subsection{Retrieving Information on an Directory Entry}

The {\tt Examine()}\mkey{Examine()} function retrieves information on the object
identified by a \emph{lock} and fills a {\tt FileInfoBlock} from it.

\begin{verbatim}
success = Examine( lock, FileInfoBlock )
D0                  D1        D2

BOOL Examine(BPTR,struct FileInfoBlock *)
\end{verbatim}

This function fills out the {\tt FileInfoBlock}\key{FileInfoBlock} providing information
on the object identified by {\tt lock}. The structure is discussed in
section~\ref{sec:fib} in more detail. The function returns non-zero in
case of success, and $0$ for failure. In either case, {\tt IoErr()} is filled,
by $0$ on success, on an error code on failure.

\punchline{Keep it Aligned!}{As with most BCPL structures, the {\tt
    FileInfoBlock} shall be aligned to a long-word boundary. For that
  reason, it should be allocated from the heap. Section~\ref{sec:bptrs}
  provides some additional hints on how to allocate such structures.}

\subsection{Retrieving Information from a File Handle}

While {\tt Examine()}\key{Examine()} retrieves information a locked
object, {\tt ExamineFH()}\mkey{ExamineFH()} retrieves the same
information from a \emph{file handle}, or rather from the \emph{lock}
implicit to the handle.

\begin{verbatim}
success = ExamineFH(fh, fib)
D0                  D1  D2

BOOL ExamineFH(BPTR, struct FileInfoBlock *)
\end{verbatim}

This function examines the object accessed through the
\emph{file handle} {\tt fh}, and returns the information in the
\emph{FileInfoBlock}\key{FileInfoBlock}\key{FileInfoBlock}. Note that
the file content and thus its change can be changed any time, and thus
the information returned by this function may not be fully up-to-date,
see also the general information in section~\ref{sec:fib}.
\smallskip{}

This function returns non-zero in case of success, or $0$ on error. In
either case, {\tt IoErr()}\key{IoErr()} is set, namely to $0$ on
success and to an error code otherwise.
\smallskip{}

As for {\tt Examine()}\key{Examine()}, the \emph{FileInfoBlock} shall
be aligned to a 4-byte boundary.

\subsection{Scanning through a Directory Step by Step}

The {\tt ExNext()}\mkey{ExNext()} function iterates through entries of
a directory, retrieving information on one object after another
contained in this directory. For scanning through a directory, first
{\tt Lock()}\key{Lock()} the directory itself. Then use {\tt
  Examine()}\key{Examine()} on the \emph{lock}. This provides
information on the directory itself.
\smallskip{}

To learn about the objects in the directory, iteratively call {\tt
  ExNext()} on the same {\tt lock} and on the same {\tt FileInfoBlock}
until the function returns {\tt DOSFALSE}. Each iteration provides
then information on the subsequent element in the directory of the
{\tt lock}.

\begin{verbatim}
success = ExNext( lock, FileInfoBlock )
D0                 D1        D2

BOOL ExNext(BPTR, struct FileInfoBlock *)
\end{verbatim}

This call returns information on the subsequent entry of a directory
identified by {\tt lock} and deposits this information in the {\tt
  FileInfoBlock}\key{FileInfoBlock} described in~\ref{sec:fib}. The
{\tt lock} shall be a \emph{lock} on a directory, in particular.
\smallskip{}

On success, {\tt ExNext()} returns non-zero. If there is no further
element in the scanned directory, or on an error, it returns {\tt
  DOSFALSE}. In either event, {\tt IoErr()}\key{IoErr()} is set,
namely to $0$ in case of success, or to an error code otherwise.
\smallskip{}

At the end of the directory, the function returns {\tt DOSFALSE}, and
the error code as obtained from {\tt IoErr()} is set to {\tt
  ERROR\_NO\_MORE\_ENTRIES}.

\punchline{Same Lock, Same FIB}{To iterate through a directory, a lock
  to the same directory as passed into {\tt Examine()}\key{Examine()}
  shall be used. Actually, the same \emph{lock} should be used, and
  the same \emph{FileInfoBlock} should be used. As important state
  information is associated to the \emph{lock} and
  \emph{FileInfoBlock}, {\tt UnLock()}ing the original \emph{lock} and
  obtaining a new \emph{lock} on the same directory looses this
  information; using a different \emph{FileInfoBlock} also looses this
  state information, requiring the \emph{file system} to rebuild this
  state information, which is not only complex, but also slows down
  scanning the directory. In particular, you shall \emph{not} use the
  same \emph{FileInfoBlock} you used for scanning one directory for
  scanning a second, different directory as this can confuse the
  \emph{file system}. Also, as for {\tt Examine()}\key{Examine()}, the
       {\tt FileInfoBlock} shall be aligned to a long-word boundary.}

\subsection{Examine Multiple Entries at once}

While scanning a directory with {\tt ExNext()}\key{ExNext()} requires
one interaction with the \emph{file system} for each entry and is
therefore potentially slow, {\tt ExAll()}\mkey{ExAll()} retrieves as
many entries as possible in one go. Whether a particular file system
can take advantage of such a block transfer is a matter of its orginal
organization, however.

\begin{verbatim}
continue = ExAll(lock, buffer, size, type, control)
D0               D1     D2     D3    D4     D5

BOOL ExAll(BPTR,STRPTR,LONG,LONG,struct ExAllControl *)
\end{verbatim}

This function examines as many directory entries belonging to the
directory identified by {\tt lock} as fit into the buffer {\tt buffer}
of {\tt size} bytes. This buffer is filled by a linked list of {\tt
  ExAllData} structures, see below for their layout. {\tt type}
determines which elements of {\tt ExAllData} is filled.
\smallskip{}

The {\tt lock} shall belong to a directory for this function to
succeed. It shall not be {\tt ZERO}.
\smallskip{}

To start a directory scan with {\tt ExAll()}, first allocate a {\tt
  ExAllControl}\mkey{ExAllControl} structure through {\tt
  AllocDosObject()}\key{AllocDosObject()}, see~\ref{sec:allocdosobject}.
This structure looks as follows:

\mkey{ExAllControl}
\begin{verbatim}
struct ExAllControl {
        ULONG   eac_Entries;     /* number of entries returned in buffer      */
        ULONG   eac_LastKey;     /* Don't touch inbetween linked ExAll calls! */
        UBYTE  *eac_MatchString; /* wildcard string for pattern match or NULL */
        struct Hook *eac_MatchFunc; /* optional private wildcard function     */
};
\end{verbatim}

{\tt eac\_Entries} is provided by the \emph{file system} upon
returning from {\tt ExAll} and then contains the number of entries
that fit into the {\tt buffer}. Note that this number may well be $0$,
which does not need to indicate termination of the scan. Callers shall
instead check the return code of {\tt ExAll()} to learn on whether
scanning may continue or not.
\smallskip{}

{\tt eac\_LastKey} is a \emph{file system} internal identifier of the
current state of the directory scanner. This member shall not be
interpreted nor modified in any way.
\smallskip{}

{\tt eac\_MatchString} filters the directory entry names, and returns
only those that match the wild card pointed to by this member. This
entry shall be either {\tt NULL}, or a pre-parsed pattern as generated
by {\tt ParsePatternNoCase()}\key{ParsePatternNoCase()}.
\smallskip{}

{\tt eac\_MatchFunc} is a even more flexible option to filter
directory entries. It shall be either {\tt NULL} or point to a {\tt
  struct Hook} as defined in {\tt utility/hooks.h}. If set, then for
each directory entry the hook function {\tt h\_Entry} is called as
follows:

\begin{verbatim}
match = (hook->h_Entry)(struct Hook *hook, LONG *datap,
d0                                    a0          a2

                        struct ExAllData *buf )
                                          a1
\end{verbatim}

that is, register {\tt a0} points to the called hook, register {\tt
  a1} to the data buffer to be filled, which is part of the {\tt
  buffer} supplied by the caller of {\tt ExAll()} and which is already
filled in.  Register {\tt a2} points to a {\tt LONG}, which is a copy
of the {\tt type} argument supplied to {\tt ExAll()}. If the hook
function returns non-zero, a match is assumed and the directory entry
remains in the output buffer. Otherwise, the data is discarded.
\smallskip{}

{\tt eac\_MatchFunc} and {\tt eac\_MatchString} shall not be filled in
simultaneously, only one of the two shall be non-{\tt NULL}. If both
members are {\tt NULL}, all entries match.
\medskip{}

The {\tt buffer} supplied to {\tt ExAll()} is filled by a singly
linked list of {\tt ExAllData} structures that look as follows:
\mkey{ExAllData}
\begin{verbatim}
struct ExAllData {
        struct ExAllData *ed_Next;
        UBYTE  *ed_Name;
        LONG    ed_Type;
        ULONG   ed_Size;
        ULONG   ed_Prot;
        ULONG   ed_Days;
        ULONG   ed_Mins;
        ULONG   ed_Ticks;
        UBYTE  *ed_Comment;     /* strings will be after last used field */
        UWORD   ed_OwnerUID;    /* new for V39 */
        UWORD   ed_OwnerGID;
};
\end{verbatim}

The members of this structure are as follows:
\smallskip{}

{\tt ed\_Next} points to the next {\tt ExAllData} structure within
{\tt buffer}, or {\tt NULL} for the last structure filled in.
\smallskip{}

{\tt ed\_Name} points to the file name of a directory entry, and
supplies the same name as {\tt fib\_FileName} as in the {\tt FileInfoBlock}.
\smallskip{}

{\tt ed\_Type} identifies the type of the entry. It identifies
directory entries according to table~\ref{table:direntrytypes} and
corresponds to {\tt fib\_DirEntryType}.
\smallskip{}

{\tt ed\_Size} is the size of the directory element for files. It is
undefined for directories. It corresponds to {\tt fib\_Size}.
\smallskip{}

{\tt ed\_Prot} collects the protection bits of the directory entry
according to table~\ref{table:protectionbits} and by that corresponds
to {\tt fib\_Protection}.
\smallskip{}

{\tt ed\_Days}, {\tt ed\_Mins} and {\tt ed\_Ticks} identifies the date
of the last change to the directory element. It corresponds to {\tt fib\_Date}.
Section~\ref{sec:setfiledate} defines these elements more rigorously.
\smallskip{}

{\tt ed\_Comment} points to a potential comment on the directory entry
and corresponds to {\tt fib\_Comment}.
\smallskip{}

{\tt ed\_ed\_OwnerUID} and {\tt ed\_OwnerGID} contain potential user
and group IDs if the file system is able to provide such
information. All the AmigaDOS native file systems do not.
\medskip{}

Which members of the {\tt ExAllData} structure are filled in is
selected by the {\tt type} argument. It shall be selected according to
table~\ref{table:exalltypes}, whose elements are defined in
{\tt dos/exall.h}:

\begin{rkrmtable}{Type Values} \label{table:exalltypes}
{\bf Type} & {\bf Filled Members}\\ \hline \hline
{\tt ED\_NAME} & Fill only {\tt ed\_Next} and {\tt ed\_Name}\\ \hline
{\tt ED\_TYPE} & Fill all members up to {\tt ed\_Type} \\ \hline
{\tt ED\_SIZE} & Fill all members up to {\tt ed\_Size} \\ \hline
{\tt ED\_PROTECTION} & Fill all members up to {\tt ed\_Prot} \\ \hline
{\tt ED\_DATE} & Fill all members up to {\tt ed\_Ticks}, i.e. up to the date \\ \hline
{\tt ED\_COMMENT} & Fill all members up to {\tt ed\_Comment} \\ \hline
{\tt ED\_OWNER} & Fill all members up to {\tt ed\_OwnerGID} \\ \hline
\end{rkrmtable}

The return code {\tt continue} is non-zero in case the directory
contents was too large to fit into the supplied {\tt buffer}
completely. In such a case, either {\tt ExAll()} shall be called again
to read additional entries, or {\tt ExAllEnd()}\key{ExAllEnd()} shall
be called to terminate the call and release all internal state information.
\smallskip{}

If {\tt ExAll()} is called again, the {\tt lock} shall be identical to
the {\tt lock} passed into the first call, and not only a copy on the
same directory as for the first call.
\smallskip{}

The return code {\tt continue} is {\tt DOSFALSE} in case the scan
result fit entirely into {\tt buffer} or in case an error occured.
\smallskip{}

Regardless of the return code, {\tt IoErr()} is set to $0$ in case
{\tt continue} is non-zero, or to an error code otherwise. If the
error code is {\tt ERROR\_NO\_MORE\_ENTRIES}, then {\tt ExAll()}
terminated because all entries have been read and scanning the
directory completed. In this case, {\tt ExAllEnd()} should not be
called.
\medskip{}

Not all file systems --- actually, none delivered with AmigaOs ---
support {\tt ED\_OWNER}. If {\tt continue} is {\tt DOSFALSE} and
{\tt IoErr()} is {\tt ERROR\_BAD\_NUMBER}, try to reduce {\tt type} and
call {\tt ExAll()} again.
\smallskip{}

Some file systems do not implement {\tt ExAll()} themselves; in such a
case, the \emph{dos.library} provides a fall-back implementation
keeping {\tt ExAll()} workable regardless of the completeness of the
target \emph{file system}.

\subsection{Aborting a Directory Scan}

To abort an {\tt ExAll()}\key{ExAll()} scan through a directory,
{\tt ExAllEnd()}\mkey{ExAllEnd()} shall be called to explicitly release
all state information associated to the scan. This is unlike an
item-by-item scan through {\tt ExNext()}\key{ExNext()} which does not
require explicit termination.

\begin{verbatim}
ExAllEnd(lock, buffer, size, type, control)
          D1     D2     D3    D4     D5

void ExAllEnd(BPTR,STRPTR,LONG,LONG,struct ExAllControl *)
\end{verbatim}

This function aborts an {\tt ExAll()} driven directory scan before it
terminates due to an error or due to the end of the directory,
i.e. whenever {\tt ExAll()} returns with a non-zero result code which
would indicate that the function should be called again.
\smallskip{}

{\tt ExAll()} may also be the fastest way to terminate a directory
scan once it is running, for example on network file systems where the
scan may proceed offline on a separate server. The arguments to
{\tt ExAllEnd()} shall be exactly those supplied to
{\tt ExAll()}\key{ExAll()} which it is supposed to terminate. Note in
particular that the {\tt lock} shall be identical to the \emph{lock}
passed into {\tt ExAll()}, and not just a \emph{lock} to the same
object.

\section{Modifying Directory Entries}

While the functions in section~\ref{sec:fib} read directory entries,
the functions listed here modify the directory and its entries.

\subsection{Deleting Objects on the File System}

The {\tt DeleteFile()}\mkey{DeleteFile()} function removes --- despite
its name --- not only files, but also directories and links from a
directory. For this to succeed, the object need to allow deletion
through its protection bits (see section~\ref{sec:fib}), and no
\emph{locks} are held on the object (see section~\ref{sec:locks}).  To
be able to delete a directory, this directory needs to be empty in
addition.

\begin{verbatim}
success = DeleteFile( name )
 D0                    D1

BOOL DeleteFile(STRPTR)
\end{verbatim}

This function deletes the object given by the last component of the
path passed in as {\tt name}. It returns non-zero in case of success,
or $0$ in case of error. In either case, {\tt IoErr()}\key{IoErr()} is
set, namely $0$ on success or an error code in case of failure.

\subsection{Rename or Relocate an Object}

The {\tt Rename()}\mkey{Rename()} function changes the name of an
object, or even relocates it from one directory to another.

\begin{verbatim}
success = Rename( oldName, newName )
  D0                D1       D2

BOOL Rename(STRPTR, STRPTR)
\end{verbatim}

This function renames and optionally relocates an object between
directories. The {\tt oldName} is the current path to the object, and
its last component is the current name of the object to relocate and
rename; {\tt newName} is the target path and its last component the
target name of the object. The target directory may be different from
the directory the object is currently located in, and the target name
may be different from the current name. However, current path and
target path shall be on the same volume, and the target directory
shall not already contain an object of the target name; otherwise,
current and target path may be either relative or absolute paths.
\smallskip{}

A third condition is that if the object to relocate is a directory,
then the target path shall not be a position within the object to
relocate, i.e. you cannot move a directory into itself.
\smallskip{}

This function returns a boolean success indicator. It is non-zero on
success, or $0$ on error. In either case, {\tt IoErr()}\key{IoErr()}
is set, to $0$ on success, or to an error code otherwise.

\subsection{Set the File Comment}

The {\tt SetComment()}\mkey{SetComment()} function sets the comment of
an directory entry, provided the \emph{file system} supports comments.

\begin{verbatim}
success = SetComment( name, comment )
D0                    D1    D2

BOOL SetComment(STRPTR, STRPTR)
\end{verbatim}

This function sets the comment of the \emph{file system} object whose
path is given by {\tt name} to {\tt comment}. It depends on the file
system whether or how long comments can grow. The maximum comment
length AmigaDOS supports is 79 characters, due to the available space
in the {\tt FileInfoBlock}\key{FileInfoBlock} structure.
\smallskip{}

This function returns non-zero on success and $0$ on error. In either
case, the function sets {\tt IoErr()}\key{IoErr()} to $0$ on success
or to an error code otherwise.

\subsection{Set the Modification Date} \label{sec:setfiledate}

The {\tt SetFileDate()}\mkey{SetFileDate()} function sets the
modification date of an object of a \emph{file system}. Despite its
name, the function can also set the modification date of directories
and links if the file system supports them.

\begin{verbatim}
success = SetFileDate(name, date)
D0                     D1    D2

BOOL SetFileDate(STRPTR, struct DateStamp *)
\end{verbatim}

This function adjusts the modification date of the \emph{file system}
object identified by path as given by {\tt name} to {\tt date}. The
{\tt DateStamp}\key{DateStamp} structure is specified in
section~\ref{sec:datestamp}.
\smallskip{}

This function returns $0$ on error or non-zero on success. In either
case, {\tt IoErr()} is set, either to $0$ on success or to an error
code otherwise.
\smallskip{}

Note that not all file systems may be able to set the date precisely
to ticks, e.g. {\tt FAT} has only a precision of 2 seconds. Some file
systems may refuse to set the modification date if an object is
exclusively locked, this is unfortunately not handled consistently.

\subsection{Set User and Group ID}

The {\tt SetOwner()}\mkey{SetOwner()} function sets the user and group
ID of an object within a \emph{file system}. Both are concatenated to
a 32-bit ID value. While this function seems to imply that the file
system or AmigaDOS seems to offer some multi-user capability, this is
not the case. User and group ID are purely metadata that is returned
by the functions discussed in section~\ref{sec:fib}, they usually
ignore them. AmigaDOS has no concept of the current user of a
\emph{file system} and thus cannot decide whether a user is
priviledged to access an object on a file system. In fact, all ROM
based file systems delivered with AmigaDOS do not support setting the
user or group ID.

\begin{verbatim}
success = SetOwner( name, owner_info )
D0                   D1       D2

BOOL SetOwner (STRPTR, LONG)
\end{verbatim}

This function sets the user and group ID of the \emph{file system}
object identified by the path in {\tt name} to the value {\tt
  owner\_info}. How exactly the {\tt owner\_info} is encoded is
\emph{file system} specific. Typically, the owner is encoded in the
topmost 16 bits, and the group in the least significant 16 bits.
\smallskip{}

This function returns a boolean success indicator which is non-zero on
success and $0$ on error. This function always sets {\tt IoErr()},
either to $0$ on success or to an error code otherwise.

\section{Working with Paths}

The \emph{dos.library} contains a couple of support functions that
help working with paths, see also section~\ref{sec:paths}. What is
different from the remaining functions is that the paths are not
interpreted by the file system, but rather by the \emph{dos.library}
itself. This has several consequences: First, there is no 255
character limit as the path is never communicated into the \emph{file
  system} as it was stated in section~\ref{sec:pathlimit}. Second, as
the paths are constructed or interpreted by the library and not the
\emph{file system}, the syntax of the path is also that imposed by the
library.
\smallskip{}

That is, for these functions to work, the separator between component
must be the forwards slash ('{\tt /}') and the parent directory must
be indicated by an isolated single forward slash without a component
upfront. This implies, in particular, that the involved file systems
follow the conventions of AmigaDOS.

\subsection{Find the Path From a Lock}

The {\tt NameFromLock()}\mkey{NameFromLock()} function constructs a
path to the locked object, i.e. if the constructed path is used to
create a lock, it will refer to the same object.

\begin{verbatim}
success = NameFromLock(lock, buffer, len)
D0                      D1     D2    D3

BOOL NameFromLock(BPTR, STRPTR, LONG)
\end{verbatim}

This function constructs in {\tt buffer} an absolute path that
identifies the object locked by {\tt lock}. At most {\tt len} bytes
will be filled into {\tt buffer}, including {\tt NUL} termination of the
string. The created string is always {\tt NUL}-terminated, even if the
buffer is too short. However, in such a case the function returns~$0$,
and {\tt IoErr()}\key{IoErr()} is set to {\tt ERROR\_LINE\_TOO\_LONG}.
\smallskip{}

If the path cannot be constructed due to an error, {\tt success} is
also set to $0$ and {\tt IoErr()} is set to an error code. However, on
success, {\tt IoErr()} is not set consistently and cannot be depended
upon. Possible cases of failure are that the volume the locked object
is located on is currently not inserted in which case it will be
requested. The {\tt ZERO} lock is correctly interpreted, and resolves
into the string {\tt SYS:}. The {\tt lock} remains valid after the
call.

\subsection{Append a Component to a Path}

The {\tt AddPart()}\mkey{AddPart()} adds an absolute or relative path
to an existing path; the resulting path is constructed as if the input
path is a directory, and the attached (second) path identifies an object
relative to this given directory. The function handles special cases
such as the colon ('{\tt :}') and one or multiple leading slashes
('{\tt /}') correctly and are interpreted according to the rules
explained in section~\ref{sec:paths}: The colon identifies the root of
the volume, and a leading slash the parent directory, upon which the
trailing component of the input path is removed.

\begin{verbatim}
success = AddPart( dirname, filename, size )
D0                   D1        D2      D3

BOOL AddPart( STRPTR, STRPTR, ULONG )
\end{verbatim}

This function attaches to the existing path in {\tt dirname} another
path in {\tt filename}. The constructed path will overwrite the buffer
in {\tt dirname}, which is able to hold {\tt size} bytes, including a
terminating {\tt NUL} byte.
\smallskip{}

If the required buffer for the constructed path, including
termination, is larger than {\tt size} bytes, then the function
returns $0$ and {\tt IoErr()} is set to {\tt ERROR\_LINE\_TOO\_LONG},
and the input buffers are not altered. Otherwise, the function returns
non-zero, and {\tt IoErr()} is not altered.
\smallskip{}

This function does not interact with a \emph{file system} and does not
check whether the paths passed in correspond to accessible
objects. The output path is constructed purely based on the AmigaDOS
syntax of paths.

\subsection{Find the last Component of a Path}

The {\tt FilePart()}\mkey{FilePart()} function finds the last
component of a path; the function name is a bit misleading since the
last component does not necessarily correspond to a file, but could
also correspond to a directory once identified by a \emph{file
  system}. If there is only a single component in the path passed in,
this component is returned. If the path passed in terminates with at
least two slashes ('{\tt /}') indicating that the last component is at
least one level above, a pointer to the terminating slash is returned.

\begin{verbatim}
        fileptr = FilePart( path )
        D0                   D1

        STRPTR FilePart( STRPTR )
\end{verbatim}

This function returns in {\tt fileptr} a pointer to the last component
of the path passed in as {\tt path}, or a pointer to '{\tt /}' in case
the input path terminates with at least two slashes.
\smallskip{}

This function cannot fail, and does not touch {\tt IoErr()}.

\subsection{Find End of Next-to-Last Component in a Path}

The {\tt PathPart()}\mkey{PathPart()} identifies the end of the
next-to-last component in a path. That is, if a {\tt NUL} is injected
at the pointer returned by this function, the resulting string
starting at the passed in buffer corresponds to a path that
corresponds to the directory containing the last component of the
path. If the passed in path consists only of a single component, the
returned pointer is identical to the pointer passed in.

\begin{verbatim}
fileptr = PathPart( path )
D0                   D1

STRPTR PathPart( STRPTR )
\end{verbatim}

This function returns in {\tt fileptr} a pointer to the end of the
next-to-last component of the {\tt path} passed in. This function
cannot fail and does not alter {\tt IoErr()}.
\smallskip{}

The only difference between this function and {\tt
  FileParth()}\key{FilePart()} is that the latter advanced over a
potential trailing slash. That is, if the last character of the input
path of {\tt PathPart()} would be a slash, then {\tt PathPart()} would
return a pointer to this slash, but {\tt FilePart()} would advance
beyond this slash. That is, the ``file part'' of a path that
explicitly indicates a directory is empty, though the ``path part'' is
the same path without the trailing slash.

\section{Links} \label{sec:links}

\emph{Links} are tools to escape the tree-like hierarchy of
directories, sub-directories and files. A \emph{link} mirrors one
object of a file system to another location such that if the object is
changed using the path of one location, the changes are reflected in
another location. Put differently, creating a link is like copying an
object except that copy and original are always in sync. The storage
for the payload data of a file is only required once, the link just
points to the same data as the original directory entry. The same goes
for links between directories: Whenever a new entry is made in one
directory, the change also appears in the other.
\smallskip{}

AmigaDOS supports two (or, actually, three) types of links:
\emph{Hard-links} and \emph{Soft-links}. The \emph{RAM-Handler}
supports a third type that will be discussed below. \emph{Hard-links}
establish the relation between two \emph{file-system} objects on the
same volume at the level of the file system. That is, whenever a link
is accessed, the file system resolves the link, transparent to its
user.  While for the Amiga \emph{Fast File System} and the
\emph{RAM-Handler} a \emph{hard-link} is a distinct directory entry
type, some file systems do not distinguish between the original object
and a \emph{hard-link} to it. For such file systems, the same payload
data is just referenced by two directory entries. If the larget of a
link is deleted on the \emph{Fast File System} or the
\emph{RAM-Handler}, and (at least one) link to the object still
exists, then (one of) the link(s) takes over and becomes the object
itself. For other file systems, only a file system internal reference
counter is decreased, and the payload data is removed only if this
counter becomes zero.
\smallskip{}

\emph{Soft-Links} work differently and can also be established between
two different \emph{file systems}, or between two different
volumes. Here, the \emph{soft-link} is a type of its own that contains
the path of the referenced object. If such a \emph{soft-link} is
accessed, an error code is reported by the \emph{file-system} and it
is then up to a higher layer such as the \emph{dos.library} or an
application program to read the link destination, and use it to create
a path from the original path and the link destination. The access is
then (hopefully) retried under the updated path. As this object may
also be a \emph{soft-link}, this process can continue; in worst case,
indefinitely if one link refers to another in a circular way. To avoid
this situation, the \emph{dos.library} follows at most 15 links.
\smallskip{}

The \emph{dos.library} supports \emph{Soft-Links} through the
functions listed in Table~\ref{table:softenabled}:

\begin{rkrmtable}{Softlink aware functions} \label{table:softenabled}
{\bf Function} & {\bf Purpose}\\ \hline \hline
{\tt Open()} & Open a file\\ \hline
{\tt Lock()} & Obtain access rights to an object\\ \hline
{\tt CreateDir()} & Create a directory\\ \hline
{\tt SetProtection()} & Modify protection bits\\ \hline
{\tt SetFileDate()} & Set the modification date of a file\\ \hline
{\tt DeleteFile()} & Delete an object on a file system\\ \hline
{\tt SetComment()} & Modify object comment\\ \hline
{\tt MakeLink()} & Create a link to an object\\ \hline
{\tt SetOwner()} & Set User and Group ID\\ \hline
\end{rkrmtable}

All of the above functions take a path of its first argument. If the
path consists of multiple components, i.e. identifies an object in a
nested directory, and one of the intermediate components are, in
fact, \emph{soft-links}, the \emph{dos.library} will automatically
resolve such an intermediate link and construct internally the true
path to the link destination. Whether a soft-link at the last
component is resolved is typically \emph{file system} and function
dependent. For example, {\tt Open()} will always resolve
\emph{soft-links}, but {\tt Lock()} or {\tt SetProtection()} may not
and may instead affect the link, not the target object. {\tt
  DeleteFile()} will never resolve a link at the final component of
the path, and will therefore delete the link, not the object linked
to.
\smallskip{}

Note that {\tt Rename()}\key{Rename()} is currently not on the list
supporting softlinks as part of the path to the object to be renamed, or as
part of the target path.
\smallskip{}

If the target of a \emph{Soft-Link} is deleted (and not the link
itself), a link pointing to it becomes invalid, even though remains in the
\emph{file system}. Any attempt to resolve the link then, obviously,
fails. AmigaDOS does not attempt to identify such invalid links. The
same cannot happen for \emph{hard-links}.
\medskip{}

Finally, the \emph{RAM-Handler} supports a special type of
\emph{hard-links} that goes across volumes. These \emph{external
  links} copy the linked object on a read-access into the RAM disk,
i.e. the \emph{RAM-Handler} implements a \emph{copy on access}. This
feature is used for the {\tt ENV:} assign containing all active system
settings. This assign points to a directory in the RAM disk which
itself is externally linked to {\tt ENVARC:}. Thus, whenever a program
attempts to access its settings --- such as the preferences programs
--- the \emph{RAM-Handler} automatically copies the data from {\tt
  ENVARC:} to {\tt ENV:}, avoiding a manual copy and also saving RAM
space for settings that are currently not accessed and thus unused.
\medskip{}

The {\tt FileInfoBlock} introduced in section~\ref{sec:fib} identifies
links through the {\tt fib\_DirEntryType} member. As seen from
table~\ref{table:direntrytypes}, \emph{hard-links} to files are
indicated by {\tt ST\_LINKFILE} and \emph{hard-links} to directories
by {\tt ST\_LINKDIR}. Note, however, that not all file systems are
able to distinguish \emph{hard-links} from regular directory entries,
so this feature cannot be dependened upon. In particular,
\emph{external links} of the \emph{RAM-Handler} cannot be identified
by any particular value of the {\tt fib\_DirEntryType}.
\smallskip{}

Table~\ref{table:direntrytypes} also provides the {\tt
  fib\_DirEntryType} for \emph{soft-links}, namely {\tt
  ST\_SOFTLINK}. As the target of a \emph{soft-link} may not under
control of the \emph{file system}, it cannot know whether the link
target is a file or a directory (or maybe another link), and therefore
a single type is sufficient to identify them.

\subsection{Creating Links}

The {\tt MakeLink()}\mkey{MakeLink()} function creates a
\emph{hard-link} or a \emph{soft-link} to an existing object on a
\emph{file system}.

\begin{verbatim}
success = MakeLink( name, dest, soft )
D0                   D1    D2    D3

BOOL MakeLink( STRPTR, LONG, LONG )
\end{verbatim}

This function creates a new link at the path {\tt name} of the type
given by {\tt soft}. The destination the link points to is given by
{\tt dest}.
\smallskip{}

If {\tt soft} is {\tt FALSE}, {\tt dest} is a \emph{lock} represented
by {\tt BPTR}. For most \emph{file systems}, {\tt dest} shall be on
the same volume as the one identified by the path in {\tt name}. The
currently only exception is the \emph{RAM-Handler} for which the
destination \emph{lock} may be on a different volume. In such a case,
an \emph{external link} is created. While the target object will be
created, it may look initially like an empty file or an empty
directory, depending on the type of the link destination. Its contents
is copied, potentially recursively creating directories, by copying
the contents of the link destination into the link, or to a file or
directory within the link. Thus, the link becomes a mirror of the link
destination whenever an object within the link or the link itself is
accessed.
\smallskip{}

If {\tt soft} is non-zero, {\tt dest} is a {\tt const UBYTE *} that
shall be casted to a {\tt LONG}. Then, this function creates a
\emph{soft-link} that is relative to the path of the link, i.e.
{\tt name}. For details on \emph{soft-link} resolution, see
section~\ref{sec:readlink}.
\smallskip{}

This function returns in {\tt success} non-zero if creation of the
lock succeeded, or $0$ in case of failure. In either case,
{\tt IoErr()} is set to an error code on failure, or $0$ on success.

\subsection{Resolving Soft-Links}

The {\tt ReadLink()}\mkey{ReadLink()} function locates the destination
of a \emph{soft-link} and constructs from the path and directory of
the link a new path that identifies the target of the link. A typical
use case for this function is if a \emph{dos.library} function returns
with the error {\tt ERROR\_IS\_SOFT\_LINK}, indicating that the
\emph{file system} needs help from a higher layer to grant access to
the object. You then typically retry the access to the object with the
path constructed by this function. Note well that this path may be
that of yet another \emph{soft-link}, requiring recursive resolution
of the link. To avoid endless recursion, this loop should be aborted
after a maximum number of attempts, then generating an error such as
{\tt ERROR\_TOO\_MANY\_LEVELS}. A suggested maximum level of nested
\emph{soft-links}, also used by the \emph{dos.library}, is 15 links.
\smallskip{}

Note, however, that such steps would not be necessary for the
functions listed in table~\ref{table:softenabled} as they already
perform such steps internally.

\begin{verbatim}
success = ReadLink( port, lock, path, buffer, size)
D0                   D1    D2    D3     D4     D5

BOOL ReadLink( struct MsgPort *, BPTR, STRPTR, STRPTR, ULONG)
\end{verbatim}

This function creates in {\tt buffer} of {\tt size} bytes a path to
the target of a \emph{soft-link} contained in the input {\tt path}
relative to the directory represented by {\tt lock}. Typically, {\tt
  path} is the path given to some object you attempted to access, and
{\tt lock} is the \emph{lock} as given by the current
directory\key{CurrentDir()} to which the path is relative. The output
path constructed in {\tt buffer} is then an updated path relative to
the same directory, i.e. relative to {\tt lock}.
\smallskip{}

The {\tt port} is the message port of the file system that is queried
to resolve the \emph{soft-link}; this port should be obtained from
{\tt GetDeviceProc()}\key{GetDeviceProc()}, see
section~\ref{sec:getdevproc}. For relative paths, this port is
identical to the one in the {\tt fl\_Task} member of the {\tt FileLock}
structure representing {\tt lock}, see section~\ref{sec:filelock}.
\smallskip{}

If {\tt size} is too small to hold the adjusted path, the function
returns $0$ and sets {\tt IoErr()} to {\tt ERROR\_LINE\_TOO\_LONG}.
\smallskip{}

The function returns non-zero in case of success, or $0$ in case of
error. In either case, {\tt IoErr()}\key{IoErr()} is set to ether $0$
on succes, or an error code otherwise.

\chapter{Administration of Volumes, Devices and Assigns} \label{sec:devicelist}

The \emph{dos.library} is just a layer of AmigaDOS that provides a
common API for input/output operations; these operations are not
implemented by the library itself, but forwarded to
\emph{file systems} or \emph{handlers}. This forwarding is based on the
exec \emph{message} and \emph{message port} system, and to this
end, the {\tt FileLock} structure and the {\tt FileHandle} structure
contain a pointer to a {\tt MsgPort}.
\smallskip{}

However, the \emph{dos.library} also needs to obtain this port from
somewhere; for relative paths (see section~\ref{sec:paths}), the
current directory\key{CurrentDir()} (see section~\ref{sec:currentdir})
provides it. For absolute paths, i.e. paths that contain a colon
('{\tt :}'), the string upfront the colon identifies handler, directly
or indirectly. If this string is empty, i.e. the path starts with a
colon, it is again the handler of the current directory that is
contacted, but otherwise, the dos searches the \emph{device list} to
find a suitable \emph{message port}. This algorithm is also available
as a function, namely {\tt GetDeviceProc()}\key{GetDeviceProc()},
which is documented in section~\ref{sec:getdeviceproc}.
\smallskip{}

Internally, the \emph{dos.library} keeps the relation between such
names and the corresponding ports in the {\tt DosList} structure. Such
a structure is also created when \emph{mounting} a handler,
i.e. advertizing the handler to the system, or when creating an
\emph{Assign}, see section~\ref{sec:assignlist}, or when inserting a
disk into a drive, thus making a particular \emph{volume} available to
the system (see also~\ref{sec:volumelist}). Only the names from
table~\ref{table:specialdevices} in~\ref{sec:deviceoverview} are
special cases and hard-coded into the~\emph{dos.library} without
requiring an entry in the \emph{device list} in the form of a
{\tt DosList} structure.
\smallskip{}

This structure, defined in {\tt dos/dosextens.h} reads as follows:

\mkey{DosList}
\begin{verbatim}
struct DosList {
    BPTR                dol_Next;        /* bptr to next device on list */
    LONG                dol_Type;        /* see DLT below */
    struct MsgPort     *dol_Task;        /* ptr to handler task */
    BPTR                dol_Lock;
    union {
        struct {
        BSTR    dol_Handler;    /* file name to load if seglist is null */
        LONG    dol_StackSize;  /* stacksize to use when starting process */
        LONG    dol_Priority;   /* task priority when starting process */
        ULONG   dol_Startup;    /* startup msg: FileSysStartupMsg for disks */
        BPTR    dol_SegList;    /* already loaded code for new task */
        BPTR    dol_GlobVec;    /* BCPL global vector to use when starting
                                 * a process. -1 indicates a C/Assembler
                                 * program. */
        } dol_handler;

        struct {
        struct DateStamp        dol_VolumeDate;  /* creation date */
        BPTR                    dol_LockList;    /* outstanding locks */
        LONG                    dol_DiskType;    /* 'DOS', etc */
        } dol_volume;

        struct {
        UBYTE   *dol_AssignName;     /* name for non-or-late-binding assign */
        struct AssignList *dol_List; /* for multi-directory assigns (regular) */
        } dol_assign;

    } dol_misc;

    BSTR                dol_Name;        /* bptr to bcpl name */
};
\end{verbatim}

and its members have the following purpose:
\smallskip{}

{\tt dol\_Next} is a \emph{BPTR} to the corresponding next entry in a
singly linked list of {\tt DosList} structures. However, this list
should not be walked manually, but instead {\tt FindDosEntry()}\key{FindDosEntry()} should be used for
iterating through this list.
\smallskip{}

{\tt dol\_Type} identifies the type of the entry, and by that also the
layout of the structure, i.e. which members of the unions are
used. The following types are defined in {\tt dos/dosextens.h}:

\begin{rkrmtable}{\emph{DosList} Entry Types} \label{table:doslisttypes}
{\tt \bf dol\_Type} & {\bf Description}\\ \hline \hline
{\tt DLT\_DEVICE} & A \emph{file system} or \emph{handler}, see~\ref{sec:deviceoverview}\\ \hline
{\tt DLT\_DIRECTORY} & A regular assign, see~\ref{sec:assignlist}\\ \hline
{\tt DLT\_VOLUME} & A volume, see~\ref{sec:volumelist}\\ \hline
{\tt DLT\_LATE} & A late binding assign, see~\ref{sec:assignlist}\\ \hline
{\tt DLT\_NONBINDING} & A non-binding assign, see~\ref{sec:assignlist}\\ \hline
\end{rkrmtable}

{\tt dol\_Task} is the \emph{MsgPort} of the handler to contact for
the particular \emph{handler}, \emph{assign} or \emph{volume}. It may
be {\tt NULL} if the \emph{handler} is not started, or a new handler
process is supposed to be started for each file opened. This is, for
example, the case for the console which requires a process for each
window it handles. \emph{File systems} usually provide their port here
such that the same process is used for all objects on the
volume. \emph{Volumes} keep here the \emph{MsgPort} of the \emph{file
  system} that operates the volume, but it to {\tt NULL} in case the
volume goes away, e.g. is ejected. For \emph{regular assigns}, this is
also the pointer to the \emph{MsgPort} of the \emph{file system} the
assign binds to; in case the assign is a \emph{multi-assign}, this is
the \emph{MsgPort} of the first directory bound to. All additional
ports are part of the {\tt AssignList}. For \emph{late assigns} this
member is initially {\tt NULL}, but will be filled in as soon as the
assign in bound to a particular directory, and then becomes the
pointer to the \emph{MsgPort} of the handler the assign is bound
to. Finally, for \emph{non-binding assigns} this member always stays
{\tt NULL}.
\smallskip{}

{\tt dol\_Lock} is only used for \emph{assigns}, and only if it is
bound to a particular directory. That is, the member remains
{\tt ZERO} for \emph{non-binding assigns} and is initially {\tt ZERO} for
\emph{late assigns}. For all other types, this member stays {\tt ZERO}.
\smallskip{}

{\tt dol\_Name} is a \emph{BPTR} to a \emph{BSTR} is the name under
which the \emph{handler}, \emph{volume} or \emph{assign} is
accessed. That is, this string corresponds to the path component
upfront the colon. As a courtesy to C and assembler functions,
AmigaDOS ensures that this string is {\tt NUL} terminated, i.e.
{\tt dol\_Name + 1} is a regular C string whose length is available in
{\tt dol\_Name[0]}.
\medskip{}

The members within {\tt dol\_handler} are used by \emph{handlers} and
\emph{file systems}, i.e. if {\tt dol\_Type} is {\tt DLT\_DEVICE}.
\smallskip{}

{\tt dol\_Handler} is a \emph{BPTR} to a \emph{BSTR} containing the
file name from which the \emph{handler} or \emph{file system} is
loaded from. It corresponds to the {\tt Handler}, {\tt FileSystem} and
{\tt EHandler} fields of the mount list. They all deposit the file
name here.
\smallskip{}

{\tt dol\_StackSize} specifies the size of the stack for creating the
\emph{handler} or \emph{file system} process. Interestingly, the unit
of the stack size depends on the {\tt dol\_GlobVec} entry. If {\tt
  dol\_GlobVec} is negative indicating a C or assembler handler, {\tt
  dol\_StackSize} is in bytes. Otherwise, that is, for BCPL handlers,
it is in 32-bit long words. This member corresponds to the {\tt
  Stacksize} entry of the mount list.
\smallskip{}

{\tt dol\_Priority} is priority of the handler process. Even though it
is a {\tt LONG}, it shall be a number between $-128$ and $127$ because
priorities of the exec task scheduler are {\tt BYTE}s. For all
practical purposes, the priority should be a value between $0$ and
$19$. It corresponds to the {\tt Priority} entry of the mount list.
\smallskip{}

{\tt dol\_Startup} is a handler-specific startup value that is used to
commumicate a configuration to the handler during startup. While this
value may be whatever the handler requires, the {\tt mount} command
either deposits here a small integer, or a pointer to the {\tt
  FileSysStartupMsg} structure defined in {\tt
  dos/filehandler.h}. Section~\ref{sec:handler} provides more details
on mounting handlers and how the startup mechanism
works. Unfortunately, it is hard to interpret {\tt dol\_Startup}
correctly, see~\ref{sec:startup}. One way to set this member is to set
{\tt Startup} in the mount list, see~\ref{sec:startup} for details.
\smallskip{}

{\tt dol\_SegList} is a \emph{BPTR} to the chained segment list of the
handler if it is loaded. For disk-based handlers, this member is
initially {\tt ZERO}. When a program attempts to access a file on the
handler, the \emph{dos.library} first checks whether this field is
{\tt ZERO}, and if so, attempts to find a segment, i.e. a binary, for
the handler. If the {\tt FORCELOAD} entry of the mount list is
non-zero, the {\tt mount} command already performs this activity. The
process of loading a handler depends on the nature of the handler and
explained in more detail in section~\ref{sec:startup}.
\smallskip{}

{\tt dol\_GlobVec} identifies the nature of the handler as AmigaDOS
supports (still) BPCL and C/assembler handlers and defines how access
to the \emph{dos list} is secured for handler loading and startup.
BCPL handlers use a somewhat more complex loading and linking
mechanism as the language-specific \emph{global vector} needs to be
populated. This is not required for C or assembler handlers where a
simpler mechanism is sufficient, more on this in
section~\ref{startup}. Another aspect of the startup process is how
the \emph{device list} is protected from conflicting accesses from
multiple processes. Two types of access protection are possible:
Exclusive access to the list, or shared access to the list. Exclusive
access protects the \emph{device list} from any changes while the
handler is loaded and until handler startup completed. This prevents
any other modification to the list, but also read access from any
other process to the list. Shared access allows read accesses to the
list while preventing exclusive access to it.
\smallskip{}

The value in {\tt dol\_GlobVec} corresponds to the {\tt GlobVec} entry
in the mount list. It shall be one of the values in
table~\ref{table:gvvalues}.

\begin{rkrmtable}{GlobVec Values} \label{table:gvvalues}
{\tt \bf dol\_Type} & {\bf Description}\\ \hline \hline
{\tt -1} & C/assembler handler, exclusive lock on the \emph{dos list} \\ \hline
{\tt -2} & C/assembler handler, shared lock on the \emph{dos list} \\ \hline
{\tt  0} & BCPL handler using system GV, exclusive lock on the \emph{dos list} \\ \hline
{\tt -3} & BCPL handler using system GV, shared lock on the \emph{dos list} \\ \hline
{\tt >0} & BPCL handler with custom GV, exclusive lock on the \emph{dos list} \\ \hline
\end{rkrmtable}

The values $0$, $-3$ and $>0$ all setup a BCPL handler, but differ in
the access type to the \emph{device list} and how the BCPL
\emph{global vector} is populated. This vector contains all global
objects and all globally reachable functions of a BCPL program,
including functions of the \emph{dos.library}. The values $0$ and $-3$
fill this vector with the system functions first, and then use the
BPCL binding mechanism to extend or override entries in this vector
with the values found in the loaded code. Any values $>0$ defines a
\emph{BPTR} to a custom vector which is used instead for initializing
the handler. This startup mechanism has never been used in AmigaDOS
and is not quite practical as this vector needs to be communicated into
the \emph{dos.library} somehow. For new code, BCPL linkage and binding
should not be used anymore.
\medskip{}

Members of the {\tt dol\_volume} structure are used if {\tt dol\_Type}
is {\tt DLT\_VOLUME}, identifying this entry as belonging to a known
specific data carrier.
\smallskip{}

{\tt dol\_VolumeDate} is the creation date of the volume. It is a
{\tt DateStamp}\ref{sec:datestamp} structure that is specified in
section~\ref{sec:datestamp}. It is used to uniquely identify the volume,
and to distinguish this volume from any other volume of the same name.
\smallskip{}

{\tt dol\_LockList} is a pointer to a singly-linked list of
\emph{locks} on the volume. This list is created by the
\emph{file system} when the volume is ejected, and contains all locks on this
volume. It is stored here to allow a similar file system to pick up
the locks once the volume is re-inserted, even if it is re-inserted
into another device. Note that the linkage is performed with
\emph{BPTR}s and the {\tt fl\_Link} member of the {\tt FileLock}
structure.
\smallskip{}

{\tt dol\_DiskType} is an identifier of the \emph{file system type}
that operated the volume and placed here such that an alternative
process of the same file system is able to pick up or refuse the locks
stored here for non-available volumes.
\medskip{}

Members of the {\tt dol\_assign} structure are used for all other
types, i.e. all types of \emph{assigns}.
\smallskip{}

{\tt dol\_AssignName} is pointer to the target name of the assign for
\emph{non-binding} and \emph{late assigns}. The \emph{dos.library}
uses this string to locate the target of the assign. For \emph{late
  assigns}, this member is used only on the first attempt to access
the assign at which {\tt dol\_Lock} is populated.
\smallskip{}

{\tt dol\_List} contains additional locks for \emph{multi-assigns} and
is only used if {\tt dol\_Type} is {\tt DLT\_DIRECTORY}. In such a
case, {\tt dol\_Lock} is the lock to the first directory of the
\emph{multi-assign}, while {\tt dol\_List} contains all following
\emph{locks} in a singly-linked list of {\tt AssignList} structures:

\mkey{AssignList}
\begin{verbatim}
struct AssignList {
        struct AssignList *al_Next;
        BPTR               al_Lock;
};
\end{verbatim}

{\tt al\_Next} points to the next \emph{lock} that is part of the
\emph{multi-assign} and {\tt al\_Lock} is the lock itself. This
structure is also defined in {\tt dos/dosextens.h}.

\section{Finding Handler or File System Ports}

The following functions find the \emph{MsgPort} of the \emph{handler}
or \emph{file system} that is responsible for a given object. The
functions search the \emph{device list}, check whether the handler is
already loaded or load it if necessary, then check whether the handler
is already running, and if not, launch an instance of it. If
\emph{multi-assigns} are involved, it can become necessary to contact
multiple \emph{file systems} to resolve the task and thus to iterate
through multiple potential \emph{file systems} to find the right one.

\subsection{Iterate through Devices Matching a Path}

The {\tt GetDeviceProc()}\mkey{GetDeviceProc()} find a handler, or the
next handler responsible for a given path. Once the handler has been
identified, or iteration through matching handlers is to be aborted,
{\tt FreeDeviceProc()} shall be called to release temporary resources.

\begin{verbatim}
devproc = GetDeviceProc(name, devproc)
D0                      D1     D2

struct DevProc *GetDeviceProc(STRPTR, struct DevProc *)
\end{verbatim}

This function takes a path in {\tt name} and either {\tt NULL} on the
first iteration or a {\tt DevProc} structure from a previous iteration
and returns either a {\tt DevProc} structure in case a matching
handler could be identified, or {\tt NULL} if no matching handler
could be found or all possible matches have been iterated over already
already.

\punchline{Give back what you got}{To release all temporary resources,
  the {\tt DevProc} structure returned by {\tt GetDeviceProc()} shall be
  either be released through {\tt FreeDeviceProc()} then aborting the scan,
  or used as first argument for {\tt GetDeviceProc()} to continue the
  iteration. The last call to this function will return {\tt NULL} and then
  also release all resources.}

The {\tt DevProc} structure, defined in {\tt dos/dosextens.h}
looks as follows:

\key{DevProc}
\begin{verbatim}
struct DevProc {
        struct MsgPort *dvp_Port;
        BPTR            dvp_Lock;
        ULONG           dvp_Flags;
        struct DosList *dvp_DevNode;    /* DON'T TOUCH OR USE! */
};
\end{verbatim}

{\tt dvp\_Port} is a pointer to a candidate \emph{MsgPort} that should
be tried to resolve {\tt name}.
\smallskip{}

If the matching handler is a \emph{file system}, then {\tt dvp\_Lock}
is a \emph{lock} of a directory. The path in {\tt name}is a path
relative to this directory. This \emph{lock} shall not be released,
but it may be copied with {\tt DupLock}\key{DupLock()}.
\smallskip{}

{\tt dvp\_Flags} identifies the nature of the found port. If the bit
{\tt DVPB\_ASSIGN} is set, i.e {\tt dvp\_Flags \& DVPF\_ASSIGN} is
non-zero, then the found match is part of a \emph{multi-assign} and
{\tt GetDeviceProc()} may be called again with the {\tt devproc}
argument just returned as second argument. This will return another
candidate for a path. {\tt DVPB\_UNLOCK} is another bit of the flags
but shall not be interpreted and is only used internally by the function.
\smallskip{}

The member {\tt dvp\_DevNode} shall not be touched or used and is
required internally by the function.
\medskip{}

If the function returns {\tt NULL}, then {\tt IoErr()}\key{IoErr()}
provides additional information on the failure. If the error code is
{\tt ERROR\_NO\_MORE\_ENTRIES}, then the last directory of a
\emph{multi-assign} has been reached. If the error code is {\tt
  ERROR\_DEVICE\_NOT\_MOUNTED}, then no matching device could be
found. Other errors may be returned, e.g. if the function could not
allocate sufficient memory for its operation.
\smallskip{}

Unfortunately, the function does not set {\tt IoErr()} consistently if
{\tt GetDeviceProc()} is called again on an existing {\tt DevProc}
structure as second argument with {\tt DVPB\_ASSIGN} cleared. {\tt
  IoErr()} remains then unaltered and it is therefore advisable to
clear it upfront.
\smallskip{}

The function also returns {\tt NULL} if {\tt name} corresponds to the
{\tt NIL:} pseudo-device and then sets {\tt IoErr()} to {\tt
  ERROR\_DEVICE\_NOT\_MOUNTED}. This is not fully correct, and callers
need to be aware of this defect.
\smallskip{}

Also, {\tt GetDeviceProc} does not handle the path ``{\tt *}'' at all,
even though it indicates the current console and the
\emph{Console-Handler} is responsible for it.  This case also needs to
be detected by the caller, and in such a case,
{\tt GetConsoleTask()}\key{GetConsoleTask()} delivers the correct port.

\punchline{Does not like all paths}{The {\tt GetDeviceProc()} function
  unfortunately does not handle all device specifiers correctly, and some
  special cases need to be filtered out by the caller. Namely ``{\tt *}''
  indicating the current console, and {\tt NIL:} for the {\tt NIL}
  pseudo-device are not handled here.}
 
\subsection{Releasing DevProc Information}

The {\tt FreeDeviceProc()}\mkey{FreeDeviceProc()} function releases a
{\tt DevProc}\key{DevProc} structure previously returned by {\tt
  GetDeviceProc()}\key{GetDeviceProc()} and releases all temporary
resources allocated by this function. It shall be called as soon as
the {\tt DevProc} structure is no longer needed.

\begin{verbatim}
FreeDeviceProc(devproc)
                 D1

void FreeDeviceProc(struct DevProc *)
\end{verbatim}

This function releases the {\tt DevProc} structure and all its
resources from an iteration through one or multiple
{\tt GetDeviceProc()} calls. If {\tt GetDeviceProc()} returned {\tt NULL}
itself it had already released such resources itself and no further
activity is necessary.
\smallskip{}

The {\tt dvp\_Port} or {\tt dvp\_Lock} within the {\tt DevProc}
structure shall not be used after releasing it with {\tt
  FreeDeviceProc()}. If a \emph{lock} is needed afterwards, a copy of
{\tt dvp\_Lock} shall be made with {\tt DupLock()}\key{DupLock()}. If
the port of the \emph{handler} or \emph{file system} is needed
afterwards, a resource of this handler shall be obtained, e.g. by
opening a file or obtaining a lock on it. Both the {\tt
  FileHandle}\key{FileHandle} and the {\tt FileLock}\key{FileLock}
structures contain a pointer to the port of the corresponding handler.
\smallskip{}

It is safe to call {\tt FreeDeviceProc()} with a {\tt NULL} argument;
this performs no activity.
\smallskip{}

This function does not set {\tt IoErr()} consistently and no
particular value may be assumed. It may or may not alter its value.

\subsection{Legacy Handler Port Access}

The {\tt DeviceProc()}\mkey{DeviceProc()} function is a legacy variant
of {\tt GetDeviceProc()}\key{GetDeviceProc()} that should not be used
anymore. It is not able to reliably provide locks to \emph{assigns}
and will not work through all directories of a \emph{multi-assign}.

\begin{verbatim}
process = DeviceProc( name )
 D0                    D1

struct MsgPort *DeviceProc (STRPTR)
\end{verbatim}

This function returns a pointer to a port of a \emph{handler} or
\emph{file system}able to handle the path {\tt name}. It returns
{\tt NULL} on error in which case it sets {\tt IoErr()}\key{IoErr()}.
\smallskip{}

If the passed in {\tt name} is part of an \emph{assign}, the handler
port of the directory the assign binds to is returned, and {\tt
  IoErr()} is set to the \emph{lock} of the assign. Unfortunately, one
cannot safely make use of this \emph{lock} as the \emph{device list}
may be altered any time, including the time between the return from
this function and its first use by the caller. Thus, {\tt
  GetDeviceProc()} shall be used instead which locks resources such as
the \emph{device list}; they are released through {\tt
  FreeDeviceProc()}.

\punchline{Obsolete and not fully functional}{{\tt DeviceProc()} function
  does not operate properly on \emph{multi-assigns} where it only provides
  the port and \emph{lock} to the first directory participating in the
  assign. It also returns {\tt NULL} for \emph{non-binding assigns} as there
  is no way to release a temporary lock obtained on the target of the
  \emph{assign}. Same as {\tt GetDeviceProc()}, it does not properly handle
  {\tt NIL:} and ``{\tt *}''.}

\subsection{Obtaining the Current Console Handler}

The {\tt GetConsoleTask()}\mkey{GetConsoleTask()} function returns the
\emph{MsgPort} of the handler responsible for the console of the
calling process, that is, the process that takes care of the file name
``{\tt *}'' or paths relative to {\tt CONSOLE:}.

\begin{verbatim}
port = GetConsoleTask()
 D0

struct MsgPort *GetConsoleTask(void)
\end{verbatim}

This function returns a port to the handler of the console of the
calling process, or {\tt NULL} in case there is no console associated
to the caller. The latter holds for example for programs started from
the workbench. It does not alter {\tt IoErr()}.

\subsection{Obtaining the Default File System}

The {\tt GetFileSysTask()}\mkey{GetFileSysTask()} function returns the
\emph{MsgPort} of the default \emph{file system} of the caller. The
default \emph{file system} is used as fall-back if a
\emph{file system} is required for a path relative to the {\tt ZERO} lock, and
the path itself does not contain an indication of the responsible
handler, i.e. is a relative path itself.
\smallskip{}

The default \emph{file system} is typically the boot file system, or
the file system of the {\tt SYS:} \emph{assign}, though it can be
changed with {\tt SetFileSysTask()} at any point.

\begin{verbatim}
port = GetFileSysTask()
 D0

struct MsgPort *GetFileSysTask(void)
\end{verbatim}

This function returns the port of the default file system of this
task. It does not alter {\tt IoErr()}. Note that {\tt SYS:} itself is
an \emph{assign} and paths starting with {\tt SYS:} do therefore not
require resolution through this function, though the default
\emph{file system} and the file system handling {\tt SYS:} are
typically identical. However, as the former is returned by {\tt
  GetFileSysTask()} and the latter is part of the \emph{device list}
\emph{assign}, they can be different.

\section{Iterating and Accessing the Device List}

While {\tt GetDeviceProc()}\key{GetDeviceProc()} uses the \emph{device
  list} to locate a particular \emph{MsgPort} and \emph{Lock}, all
other members of the {\tt DosList} structure remain unavailable. For
them, the \emph{device list} containing these structures need to be
scanned manually. The \emph{dos.library} provides functions to grant
access, search and release access to this list.

\subsection{Gaining Access to the Device List}

The {\tt LockDosList()}\mkey{LockDosList()} function requests shared
or exclusive access to a subset of entries of the \emph{device list}
containing all \emph{handlers}, \emph{volumes} and \emph{assigns} and
blocks until access is granted. It requires as input multiple sets
that specify which parts of the list to access:

\begin{verbatim}
dlist = LockDosList(flags)
 D0                   D1

struct DosList *LockDosList(ULONG)
\end{verbatim}

This function grants access to a subset of entries of the \emph{device
  list} indicated by {\tt flags}, and returns an opaque handle through
which elements of the list can be accessed. For this, see
{\tt FindDosEntry()}\key{FindDosEntry()}.
\smallskip{}

The {\tt flags} value shall be combination of the following values, all
defined in {\tt dos/dosextens.h}:

\begin{rkrmtable}{LockDosList Flags} \label{table:lockdoslistflags}
{\tt Flags} & {\bf Description}\\ \hline \hline
{\tt LDF\_DEVICES} & Access \emph{handlers} and \emph{file system} entries, see~\ref{sec:deviceoverview} \\ \hline
{\tt LDF\_VOLUMES} & Access \emph{volumnes}, see~\ref{sec:volumelist} \\ \hline
{\tt LDF\_ASSIGNS} & Access \emph{assigns}, see~\ref{sec:assignlist} \\ \hline \hline
{\tt LDF\_ENTRY}  & Lock access to a {\tt DosList} entries \\ \hline
{\tt LDF\_DELETE} & Lock \emph{device list} for deletion \\ \hline \hline
{\tt LDF\_READ} & Shared access to the \emph{device list} \\ \hline
{\tt LDF\_WRITE} & Exclusive access to the \emph{device list} \\ \hline
\end{rkrmtable}

At least {\tt LDF\_READ} or {\tt LDF\_WRITE} shall be included in the
flags, they shall not be set both. The three first flags may also be
combined to access multiple types.
\smallskip{}

{\tt LDF\_ENTRY} and {\tt LDF\_DELETE} are additional flags that
moderate access to entries of the \emph{device list}. If {\tt
  LDF\_ENTRY} is set, then exclusive access to the selected entries is
requested and entries shall not be altered or removed. The {\tt
  LDF\_ENTRY} flag shall not be combined with {\tt LDF\_READ}. If
{\tt LDF\_DELETE} is set, then access is granted for removing entries
from the list.
\medskip{}

The result code {\tt dlist} is \emph{not} a pointer to a {\tt DosList}
structure, but only a handle that may be passed into {\tt
  FindDosEntry()}\key{FindDosEntry()} or {\tt
  NextDosEntry()}\key{NextDosEntry}. If {\tt dlist} is {\tt NULL},
then locking failed because the combination of {\tt flags} passed in
was invalid.
\smallskip{}

This function does not alter {\tt IoErr()}.

\subsection{Requesting Access to the Device List}

The {\tt AttemptLockDosList()}\mkey{AttemptLockDosList()} requests
access to the \emph{device list} or a subset of its entries, and, in
case it cannot gain access, returns {\tt NULL}. Unlike
{\tt LockDosList()}\key{LockDosList()}, it does not block.

\begin{verbatim}
dlist = AttemptLockDosList(flags)
D0                          D1

struct DosList *AttemptLockDosList(ULONG)
\end{verbatim}

The {\tt flags} argument specifies which elements of the \emph{device
  list} are requested for access, and which type of access is
required. The flags are a combination of the flags listed in
table~\ref{table:lockdoslistflags}, and the semantics of the flags are
exactly as specified for {\tt LockDosList()}\key{LockDosList()}, see
there for details.
\smallskip{}

The result code is either a (non-{\tt NULL}) handle that may be passed
into {\tt FindDosEntry()}\key{FindDosEntry()} or {\tt
  NextDosEntry()}\key{NextDosEntry()} in case access could be granted,
or {\tt NULL}. In the latter case, the list is either currently locked
and access cannot be granted without blocking, or flags are
invalid. These two cases of failure cannot be distinguished
unfortunately.
\smallskip{}

This function does not alter {\tt IoErr()}.

\subsection{Release Access to the Device List}

The {\tt UnLockDosList()}\mkey{UnLockDosList()} function releases
access to the \emph{device list} once obtained through {\tt
  LockDosList()}\key{LockDosList()}.

\begin{verbatim}
UnLockDosList(flags)
                D1

void UnLockDosList(ULONG)
\end{verbatim}

This function releases access to the \emph{device list} again. The
{\tt flags} argument shall be identical to the {\tt flags}argument
provided to {\tt LockDosList()}\key{LockDosList()}.

\subsection{Iterate through the Device List}

The {\tt NextDosEntry()}\mkey{NextDosEntry()} iterates to the next
entry in the \emph{device list} given the current entry or the handle
returned by {\tt LockDosList()}.

\begin{verbatim}
newdlist = NextDosEntry(dlist,flags)
 D0                       D1    D2

struct DosList *NextDosEntry(struct DosList *,ULONG)
\end{verbatim}

This function returns the next {\tt DosList}\key{DosList} structure of the
\emph{device list} which shall have been locked with {\tt LockDosList()}. The
{\tt dlist} argument shall be either the return code of a previous
{\tt NextDosEntry()} or {\tt FindDosEntry()}\key{FindDosEntry()}
call, or the handle returned by {\tt LockDosEntry()}.
\smallskip{}

The {\tt flags} argument shall be a subset of the {\tt flags} argument
into {\tt LockDosList()} and specifies the type of {\tt DosList}
structures that shall be found. Only the first 3 elements of
Table~\ref{table:lockdoslistflags} are relevant here, all other flags
are ignored but may be included.
\smallskip{}

The {\tt newdlist} result is either a pointer to a {\tt DosList}
structure of the requested type, or {\tt NULL} if the end of list has
been reached. This function does not alter {\tt IoErr()}.

\subsection{Find a Device List Entry by Name}

The {\tt FindDosEntry()}\mkey{FindDosEntry()} function finds a
{\tt DosList}\key{DosList} structure of a particular type and particular
name, from a particular entry on, or the handle returned by
{\tt LockDosList()}\key{LockDosList()}.

\begin{verbatim}
newdlist = FindDosEntry(dlist,name,flags)
D0                       D1    D2   D3

struct DosList *FindDosEntry(struct DosList *,STRPTR,ULONG)
\end{verbatim}

This function scans through the \emph{device list} starting at the
entry {\tt dlist}, or the handle returned by {\tt LockDosList()}, and
returns the next {\tt DosList} structure that is of the type indicated
by {\tt flags} and has the name {\tt name}.
\smallskip{}

The {\tt flags} shall be a subset of the {\tt flags} argument passed
into {\tt LockDosList()}. Only the first 3 elements of
Table~\ref{table:lockdoslistflags} are relevant here, all other flags
are ignored but may be included.
\smallskip{}

The {\tt name} argument is the (case-insensitive) name of the {\tt
  assign}, {\tt handler}, {\tt file system} or {\tt volume} the
function should look for. The name \emph{shall not} include the colon
('{\tt :}') that separates the name from the remaining components of a
path, see section~\ref{sec:paths}. It may be {\tt NULL} in which case
every entry of the requested type matches.
\smallskip{}

The returned {\tt newdlist} is a pointer to a {\tt DosList} structure
that matches the name (if provided) and flags passed in, or {\tt NULL}
in case no match could be found and the entire list has been
scanned. Note that the returned {\tt DosList} may be identical to the
{\tt dlist} passed in if it already fits the requirements. Thus
potentially, {\tt NextDosList()} may be called upfront to scan from
the subsequent entry.
\smallskip{}

Passing {\tt NULL} as {\tt dlist} is safe and returns {\tt NULL},
i.e. the end of the list. Note that the (pseudo-) devices from
tables~\ref{table:specialdevices} and \ref{table:specialassigns} are
not part of the \emph{device list}, i.e. {\tt NIL}, {\tt CONSOLE},
{\tt *} and {\tt PROGDIR} cannot be found and are special cases of
{\tt GetDeviceProc()}\key{GetDeviceProc()}.
\smallskip{}

This function does not alter {\tt IoErr()}.

\section{Adding or Removing Entries to the Device List}

The \emph{dos.library} provides two service functions to add or remove
{\tt DosList} structures from the \emph{device list}. They secure the
\emph{dos.library} internal state from inconsistencies as other
processes may attempt to access the \emph{device list} simultaneously,
and they also ensure proper linkage of the structures.
\smallskip{}

\punchline{Locking the device list in file systems}{There is one particular
  race condition \emph{file system} authors should be aware of. When opening
  a file, or obtaining a \emph{lock}, the \emph{dos.library} calls through
  {\tt GetDeviceProc()} to identify a \emph{handler} responsible for the
  requested path. As {\tt GetDeviceProc()}\key{LockDosList()} requires
  access to the \emph{device list}, it will secure access to it through {\tt
    LockDosList()}\key{LockDosList()}, then possibly start up the
  \emph{handler}, and then unlock the list. Thus, at the time the handler is
  initiated, it may find the \emph{device list} unaccessible. Attempting to
  lock it would result in a \emph{deadlock} situation as the
  \emph{dos.library} waits for the \emph{handler} to reply its startup
  packet, and the \emph{handler} waits for the \emph{dos.library} to grant
  access to the \emph{device list}.  The following sections provide
  workarounds how to avoid this situation, see also
  section~\ref{sec:handler} for details on the \emph{handler} and \emph{file
  system} startup mechanism.}

\subsection{Adding an Entry to the Device List}

The {\tt AddDosEntry()}\mkey{AddDosEntry()} adds an initialized {\tt
  DosList} structure to the \emph{device list}.

\begin{verbatim}
success = AddDosEntry(dlist)
D0                     D1

LONG AddDosEntry(struct DosList *)
\end{verbatim}

This function takes an initialized {\tt DosList} entry pointed to by
{\tt dlist} and attempts to add it to the \emph{device list}. For
this, it requests write access to the list, i.e. locking of the
\emph{device list} through the caller is not necessary.
The {\tt DosList} may be either created manually, by
{\tt MakeDosEntry()}\key{MakeDosEntry()} of the \emph{dos.library}
or by {\tt MakeDosNode()}\key{MakeDosNode()} of the \emph{expansion.library}.
While there the structure is called a {\tt DeviceNode}, it is still a
particular incarnation of a {\tt DosList} and may be savely used here.
\medskip{}

\emph{Assigns} \emph{shall not} be added to the \emph{device list}
through this function, but rather through
{\tt AssignLock()}\key{AssignLock()}, {\tt AssignPath()}\key{AssignPath()}
or {\tt AssignLate()}\key{AssignLate()}, depending on the nature of the
required assign.  This avoids memory management problems when
releasing or changing assigns.
\smallskip{}

Particular care needs to be taken if this function is called from
within a \emph{handler} or \emph{file system}, e.g. to add a
\emph{volume} representing an inserted medium. As the list may be
locked by the \emph{dos.library} to secure the list from modifications
within a {\tt GetDeviceProc()}\key{GetDeviceProc()} function, a
deadlock may result where \emph{file system} and \emph{dos.library}
mutually block access. To prevent this from happening handlers should
check upfront whether the \emph{device list} is available for
modifications by {\tt AttemptLockDosList()}\key{AttemptLockDosList()}, e.g.

\begin{verbatim}
if (AttemptLockDosList(LDF_VOLUMES|LDF_WRITE)) {
  rc = AddDosEntry(volumenode);
  UnLockDosList(LDF_VOLUMES|LDF_WRITE);
}
\end{verbatim}
when adding a {\tt DosList} entry of type {\tt DLT\_VOLUME}. If
attempting to get write access failed, the handler should check for
incoming requests, handle them, and attempt adding the entry later.
\smallskip{}

The function fails if an entry is to be added and an entry of the same
name, regardless its type, is already present on the list. The only
exception is that the list may contain two \emph{volumes} of the same
name, provided provided their creation date {\tt dol\_VolumeDate}
differs, see section~\ref{sec:devicelist}.
\smallskip{}

If successful, the function returns non-zero, but then does not alter
{\tt IoErr()}. The {\tt DosList} is then enqueued in the
\emph{dos.library} database and it and its members shall then no
longer be altered or released by the caller. On failure, the function
returns $0$ and {\tt IoErr()} is set to {\tt ERROR\_OBJECT\_EXISTS}.

\subsection{Removing an Entry from the Device List}

The {\tt RemDosEntry()}\mkey{RemDosEntry()} removes a {\tt DosList}
entry from the \emph{device list}, making it unacessible for AmigaDOS.

\begin{verbatim}
success = RemDosEntry(dlist)
D0                     D1

BOOL RemDosEntry(struct DosList *)
\end{verbatim}

This function attempts to find the {\tt DosList} structure pointed to
by {\tt dlist} in the \emph{device list} and, if present, removes
it. Unlike what some other documentation says, this funciton locks the
\emph{device list} properly before attempting to remove an entry,
locking it upfront is not necessary.
\smallskip{}

The function does \emph{not} attempt to release the memory allocated
for the {\tt DosList} passed in, or any of its members, it just
removes the {\tt DosList} from the \emph{device list}. While
\emph{file systems} may know how they allocated the {\tt DosList}
structures represening their \emph{volumes} and hence should be aware
how to release the memory taken by them, there is no good solution on
how to recycle memory for {\tt DosList} structures representing
\emph{handlers}, \emph{file systems} or \emph{assigns}. Some manual
footwork is currently required, see also {\tt
  FreeDosNode()}\key{FreeDosNode()}. In particular, as entries
representing \emph{handlers} and \emph{file systems} may have been
created in multiple ways, their memory cannot be safely recycled.
\smallskip{}

Particular care needs to be taken if this function is called from
within a \emph{handler} or \emph{file system}, e.g. to remove a
\emph{volume} representing a removed medium. As the list may be
locked by the \emph{dos.library} to secure the list from modifications
within a {\tt GetDeviceProc()}\key{GetDeviceProc()} function, a
deadlock may result where \emph{file system} and \emph{dos.library}
mutually block access. To prevent this from happening handlers should
check upfront whether the \emph{device list} is available for
modifications by {\tt AttemptLockDosList()}\key{AttemptLockDosList()}, e.g.

\begin{verbatim}
if (AttemptLockDosList(LDF_DELETE|LDF_ENTRY|LDF_WRITE)) {
  rc = RemDosEntry(volumenode);
  UnLockDosList(LDF_DELETE|LDF_ENTRY|LDF_WRITE);
}
\end{verbatim}
when removing a {\tt DosList} entry. If attempting to get write access
failed, the handler should check for incoming requests, handle them,
and attempt adding the entry later.
\smallskip{}

This function returns a success indicator; it returns non-zero if the
function succeeds, and $0$ in case it fails. The only reason for
failure is that {\tt dlist} is not a member of the
\emph{device list}. This function does not touch {\tt IoErr()}.

\section{Creating and Deleting Device List Entries}

AmigaOs offers multiple functions to create {\tt DosList}
structures. The {\tt MakeDosEntry()}\key{MakeDosEntry()} function is a
low-level function that allocates a {\tt DosList} but only performs
minimal intialization of the structure. For \emph{assigns}, the
functions {\tt AssignLock()}\key{AssignLock()}, {\tt
  AssignLate()}\key{AssignLate()} or {\tt
  AssignPath()}\key{AssignPath()} provide better alternatives as they
include complete initialization of the {\tt DosList}. To initialize
{\tt DosList}s for \emph{handlers} and \emph{file systems}, the
\emph{expansion.library} function {\tt
  MakeDosNode()}\key{MakeDosNode()}.
\smallskip{}

Releasing {\tt DosList}s along with all its resources is unfortunately
much harder. For \emph{assigns}, the algorithm in
section~\ref{sec:releaseentry} provides a workable function based on
{\tt FreeDosEntry()}\key{FreeDosEntry()}.
\smallskip{}

{\tt DosLists} representing \emph{Volumes} are build and released by
\emph{file systems}; it depends on them which resources need to be
released along with the {\tt DosList} structure itself. While it is
recommended that \emph{file systems} should go through
{\tt MakeDosEntry()}\key{MakeDosEntry()} and {\tt FreeDosEntry()}\key{FreeDosEntry()}, it
is not a requirement.
\smallskip{}

Releasing a {\tt DosList} representing a \emph{handler} or \emph{file
  system} is currently not possible in a completely robust way. It is
suggested just to unlink such nodes if absolutely necessary, but
tolerate the memory leak.

\subsection{Creating a Device List Entry}

The {\tt MakeDosEntry()}\mkey{MakeDosEntry()} creates an empty
{\tt DosList} structure of the given type, and makes all elementary
initializations. It does not accquire any additional resources, and neither
inserts it into the \emph{device list}.
\smallskip{}

If an \emph{assign} is to be created, the {\tt AssignLock()}\key{AssignLock()},
{\tt AssignLate()}\key{AssignLate()} or {\tt AssignPath()}\key{AssignPath()}
functions are better alternatives and should be preferred as they perform
a more sophisticated initialization.

\begin{verbatim}
newdlist = MakeDosEntry(name, type)
D0                       D1    D2

struct DosList *MakeDosEntry(STRPTR, LONG)
\end{verbatim}

This function allocates a {\tt DosList} structure and initializes its
{\tt dol\_Type} to {\tt type}. The {\tt type} argument shall be one of
the values from table~\ref{table:doslisttypes}. The function also
makes a copy of {\tt name} and initializes the {\tt dol\_Name} to a
\emph{BSTR} copy of {\tt name}, which is a {\tt NUL} terminated C string.
\smallskip{}

Note that this function performs only minimal initialization of the
{\tt DosList} structure. All other members except {\tt dol\_Type} and
{\tt dol\_Name} are initialized to $0$.
\smallskip{}

This function either returns the allocated structure, or {\tt NULL}
for failure. In the latter case, {\tt IoErr()} is set to {\tt
  ERROR\_NO\_FREE\_STORE}. On success, {\tt IoErr()} remains
unaltered.

\subsection{Releasing a Device List Entry} \label{sec:releaseentry}

The {\tt FreeDosEntry()}\mkey{FreeDosEntry()} function releases
a {\tt DosList} structure allocated by {\tt MakeDosEntry()}\key{MakeDosEntry()}.
The {\tt DosList} shall be already removed from the \emph{device list} by
\emph{RemDosEntry()}. While this call releases the memory holding the
name of the entry, and also the {\tt DosList} structure itself, it
does not release any other resources. They shall be released by the
caller of this function. Furthermore, this function shall not be
called if the {\tt DosList} structure was allocated by any other means
than {\tt MakeDosEntry()}.

\begin{verbatim}
FreeDosEntry(dlist)
               D1

void FreeDosEntry(struct DosList *)
\end{verbatim}

This function releases the {\tt DosList} structure pointed to by
{\tt dlist} and its name, but only these two resources, and no other
resources.
\smallskip{}

If {\tt dol\_Type} is {\tt DLT\_DEVICE}, corresponding to
\emph{handlers} or \emph{file systems}, this function should better
not be called at all as the means of how the {\tt DosList} was
allocated is unclear. In such a case, a memory leak is the least
dangerous side effect.
\smallskip{}

If {\tt dol\_Type} is {\tt DLT\_DIRECTORY} or {\tt DLT\_LATE}, then
{\tt dol\_Lock} should be unlocked. If {\tt dol\_List} is non-{\tt
  NULL}, then each entry of the {\tt AssignList} structure shall be
released, along with the lock kept within. For {\tt DLT\_LATE} and
{\tt DLT\_NONBINDING}, the {\tt dol\_AssignName} function shall also
be released. The following code segment releases all resources for
\emph{assigns}:

\begin{verbatim}
    struct AssignList *al,*next;
    UnLock(dol->dol_Lock);
    al = dol->dol_misc.dol_assign.dol_List;
    while(al) {
        next = al->al_Next;
        UnLock(al->al_Lock);
        FreeVec(al);
        al = next;
    }
    FreeVec(dol->dol_misc.dol_assign.dol_AssignName);
    FreeDosEntry(dol);
\end{verbatim}

The above code reflects the way how resources were originally
allocated by the \emph{dos.library}.
\smallskip{}

If the type is {\tt DLT\_VOLUME}, it is up to the \emph{file system}
to release any resources it allocated along with the {\tt DosList}. It is file
system dependent which resources can or should be released. {\tt
  DosList} entries of this type should only be touched by the
\emph{file system} that created them.
\smallskip{}

This function cannot fail, and it does not touch {\tt IoErr()}.

%%% The section on assigns goes here and is missing.
\chapter{Processes} \label{sec:process}

\emph{Processes}\mkey{Process} are extensions of exec \emph{tasks}, and as
such scheduled by exec. The most important extensions are that processes
include a message port in the form of a \emph{MsgPort} structure for
inter-process communication to \emph{handlers}, a current directory to
resolve relative paths, and the last input/output error as returned by the
{\tt IoErr()}\key{IoErr()} function.
\smallskip{}

\emph{Processes} are represented by the {\tt Process} structure documented
in {\tt dos/dosextens.h}. It reads as follows:
\begin{verbatim}
struct Process {
    struct  Task    pr_Task;
    struct  MsgPort pr_MsgPort;
    WORD    pr_Pad;
    BPTR    pr_SegList;
    LONG    pr_StackSize;
    APTR    pr_GlobVec;
    LONG    pr_TaskNum;
    BPTR    pr_StackBase;
    LONG    pr_Result2;
    BPTR    pr_CurrentDir;
    BPTR    pr_CIS;
    BPTR    pr_COS;
    APTR    pr_ConsoleTask;
    APTR    pr_FileSystemTask;
    BPTR    pr_CLI;
    APTR    pr_ReturnAddr;
    APTR    pr_PktWait;	
    APTR    pr_WindowPtr;

    /* following definitions are new with 2.0 */
    BPTR    pr_HomeDir;
    LONG    pr_Flags;
    void    (*pr_ExitCode)();
    LONG    pr_ExitData;
    UBYTE   *pr_Arguments;
    struct MinList pr_LocalVars;
    ULONG   pr_ShellPrivate;
    BPTR    pr_CES;
};  /* Process */
\end{verbatim}

The members of this structure are as follows:
\smallskip{}

{\tt pr\_Task} is the exec task structure defined in {\tt exec/tasks.h}. It
is required by the exec scheduler. The only difference between an exec
{\tt Task} and a {\tt Process} is that {\tt pr\_Task.tc\_Node.ln\_Type} is set
to {\tt NT\_PROCESS} instead to {\tt NT\_TASK}. 
\smallskip{}

{\tt pr\_MsgPort} is a message port structure as defined in {\tt
  exec/ports.h}. This port is used by many functions of the
\emph{dos.library} to communicate with \emph{handlers} and \emph{file
systems}. Details of the communication protocol are given in
section~\ref{sec:handlers}.
\smallskip{}

{\tt pr\_Pad} is unused and only included in the structure to ensure that
all following members are aligned to 32-bit boundaries.
\smallskip{}

{\tt pr\_SegList} contains an array of \emph{segments} containing AmigaDOS
functions. The first entry in this array is a 32-bit integer indicating the
number of valid elements, the remaining entries are \emph{BPTRs} to segments
of AmigaDOS and the loaded binary. Some entries may be {\tt ZERO} indicating
that the corresponding entry is currently not used. Segments are explained
in more detail in section~\ref{sec:segments}. Typically, entries $1$ and $2$
are system segments containing AmigaDOS functions, entry $3$ is used for the
loaded binary, and entry $4$ the segment of the shell. This, however, only
reflects the current usage of segments, and later versions of AmigaDOS may
populate this vector differently. The segments contained in this vector are
used by the AmigaDOS runtime binder to build the \emph{Global Vector} of
processes using BCPL linkage. As BPCL is phased out, this vector is of no
particular importance today anymore, and can be ignored for almost all
purposes. The only exception is the \emph{Shell} which shall prepare this
vector to ensure that commands written in BCPL function properly. More on
this in section~\ref{sec:shell}.

{\tt pr\_StackSize} is the size of the process stack in bytes. It is always
a multiple of 4 bytes long.
\smallskip{}

{\tt pr\_GlobVec} is another BCPL legacy. It contains the \emph{Global
Vector} of the process. For binaries using the BCPL linkages, this is a
custom-build array of global data and function entry points from {\tt
  pr\_SegList}. For C and assembler binaries, the \emph{Global Vector} is
the system shared vector; it contains \emph{dos.library} global data
required by some of its functions, such as base pointers to system
libraries. As no particular advantage can be taken from this vector
(anymore) as all functions available in it are also available as
\emph{dos.library} entry points, it should be left alone.
\smallskip{}

{\tt pr\_TaskNum} is an integer allocated by the system for processes that
execute a shell, or are binaries that have been launched by the shell. The
number here corresponds to the integer printed by the {\tt Status}
command. Note that AmigaDOS does not use task numbers consistenty,
i.e. processes that are started from the workbench or have been created by
some other means are not identified by a task number. In such a case, this
member remains~$0$.
\smallskip{}

{\tt pr\_StackBase} is a \emph{BPTR} to the address of the lower end of the
stack, i.e. the end of the C or assembler stack. As the BCPL stack grows in
opposide direction, it is the start of the BCPL stack. While it is
initialized, it is not used by the \emph{dos.library} at all.
\smallskip{}

{\tt pr\_Result2} is the secondary result code set by many functions of the
\emph{dos.library}. The value stored here is delivered by {\tt IoErr()}\key{IoErr()}.
\smallskip{}

{\tt pr\_CurrentDir} is the \emph{lock} representing the current directory
of the process. All relative paths are resolved from this \emph{lock},
i.e. they are relative to {\tt pr\_CurrentDir}. If this member is {\tt ZERO},
the currrent directory is the root directory of the file system stored in
{\tt pr\_FileSystemTask}. As the latter is (unless altered) the file system
of the boot volume, this is usually identical to the directory identified by
the {\tt SYS} assign.
\smallskip{}

{\tt pr\_CIS} is \emph{file handle} of the standard input stream of the
process. It is also returned by {\tt Input()}\key{Input()}. It can be {\tt
  ZERO} in case the process does not have a standard input stream. This is
\emph{not} equivalent to a {\tt NIL:} input handle --- in fact, any attempt
to read from a non-existing input stream will crash. Processes started from
the workbench do not have an input stream, unless one is installed here with
{\tt SelectInput()}\key{SelectInput()}.
\smallskip{}

{\tt pr\_COS} is the \emph{file handle} of the standard output stream of the
process. It is also returned by the {\tt Output()}\key{Output()} function of
the \emph{dos.library}. It can be {\tt ZERO} in case the process does not
have a standard output stream, which is not equivalent to a {\tt NIL:} file
handle. Any attempt to output to {\tt ZERO} will crash the system. Processes
started from the workbench do not have an output stream, unless one is
installed with {\tt SelectOutput()}\key{SelectOutput()}.
\smallskip{}

{\tt pr\_ConsoleTask} is the \emph{MsgPort} of the console within which this
process is run, if such a console exists. This \emph{handler} is contacted
when opening ``{\tt *}'' or a path relative to {\tt CONSOLE:}. Processes
started from the workbench do not have a console, unless one is installed
with {\tt SetConsoleTask()}\key{SetConsoleTask()}.
\smallskip{}

{\tt pr\_FilesSystemTask} is the \emph{MsgPort} of the file system that is
contacted in case a relative path is to be resolved relative to the {\tt
  ZERO} lock. This member is initialized to the \emph{MsgPort} of the file
system the system was booted from, but can be changed by {\tt
  SetFileSysTask()}\key{SetFileSysTask()}. This member is also returned by
{\tt GetFileSysTask()}\key{SetFileSysTask()}.
\smallskip{}

{\tt pr\_CLI} is a \emph{BPTR} to the {\tt CommandLineInterface}
\key{CommandLineInterface} structure containing information on the Shell
this process is running in. If this process is not part of a Shell, this
member is {\tt ZERO}. This is for example the case for programs started from
the workbench, or \emph{handler} or \emph{file system}.
\smallskip{}

{\tt pr\_ReturnAddr} is another \emph{BCPL} legacy and should not be used by
new implementations. It is the value to which the stack pointer is restored
by the {\tt Exit()}\key{Exit()} function. The cleanup function called by
{\tt Exit()} is found at {\tt BADDR(pr\_ReturnAddr - 1)} because the BCPL
stack shrinks towards lower addresses. AmigaDOS stores here the default
process shutdown code which releases system resources accquired by AmigaDOS
when creating the process, which is the same function that is called if the
process ``falls off its final function with {\tt RTS}''. This cleanup
process does not, however, release any other resources obtained by user
code. In principle, custom startup code could deposit here a stack pointer
to release all resources by the user, though this mechanism is rarely used.
For C, the {\tt exit()} function provided by the C library of the compiler
should be used instead.
\smallskip{}

{\tt pr\_PktWait} is a function that is called when waiting for
inter-process communication, in particular when waiting for a returning
packet set out to a handler. If this is {\tt NULL}, the system default
function is used.  The signature of this function is
\begin{verbatim}
msg = (*pr_PktWait)(void)
D0

struct MsgPort *(*pr_PktWait)(void)
\end{verbatim}

that is, no particular arguments are delivered, the process must be obtained
from {\tt exec}, and the message received shall be delivered back into
register {\tt D0}. The returned pointer shall not be {\tt NULL}, rather,
this function shall block until a message has been received. For details,
see the {\tt DoPkt()}\key{DoPkt()} function and section~\ref{sec:handlers}.
\smallskip{}

{\tt pr\_WindowPtr} is, unlike what the name suggests, a pointer to an
\emph{intuition} {\tt Screen} structure, see {\tt intuition/screens.h}, on
which error requesters will appear. If this is {\tt NULL}, error requesters
appear on the workbench screen, and if this is set to {\tt (APTR)(-1L)},
error requesters will be suppressed at all, and the implied reponse to them
is to cancel the operation. This error requester is specified in more detail
with the {\tt ErrorReport()}\key{ErrorReport()} function.
\smallskip{}

{\tt pr\_HomeDir} is the \emph{lock} to the directory containing the binary
that is currently executed as this process, if such a directory exists. It
is {\tt ZERO} if the binary is resident. This \emph{lock} is filled in by
the Shell or the Workbench when loading and starting a process. It is used
to resolve paths relative to the {\tt PROGDIR} pseudo-assign, see
section~\ref{table:specialassigns}. If this lock is {\tt ZERO}, any attempt
to resolve a path within {\tt PROGDIR:} will create a request to inserted a
volume {\tt PROGDIR:}, which is probably not a very useful reaction of
AmigaDOS.
\smallskip{}

{\tt pr\_Flags} are system-use only flags that shall not be used or
interpreted. They are used by the system process shutdown code to identify
which resources need to be released, but future systems may find additional
uses for this member.
\smallskip{}

{\tt pr\_ExitCode()} is a pointer to a function that is called by AmigaDOS
as part of the process shutdown code, and as such quite more useful that
{\tt pr\_ReturnAddr}. Thie function prototype is as follows:

\begin{verbatim}
returncode = ExitFunc(rc,exitdata)
D0                    D0 D1

LONG ExitFunc(LONG,LONG)
\end{verbatim}

The value of {\tt rc} is the return code process, i.e. the value left in
register {\tt D0} when the code drops off the final {\tt RTS}, and {\tt
  exitdata} is taken from {\tt pr\_ExitData}. The {\tt returncode} is a
modified version of the process return code that is, however, ignored.
\smallskip{}

{\tt pr\_ExitData} is used as argument for the {\tt pr\_ExitCode()}
function, see above.
\smallskip{}

{\tt pr\_Arguments} is a pointer to the command line arguments of the
process if it corresponds to a command started from the Shell. This is a
{\tt NUL} terminated string. This argument string can also be found in
register {\tt A0} for programs started from the Shell, or in the buffer of
{\tt pr\_CIS}. The {\tt ReadArgs()}\key{ReadArgs()} function takes it from
the latter source, and not from {\tt pr\_Arguments}. Otherwise, this member
remains {\tt NULL}.
\smallskip{}

{\tt pr\_LocalVars} is a {\tt MinList} structure, as defined in {\tt
  exec/lists.h}, that contains local variables specific to the shell within
which the process is executed, if any. The structure of such variables is
defined in {\tt dos/var.h}. This structure is specified in
section~\ref{sec:shell}.
\smallskip{}

{\tt pr\_ShellPrivate} is reserved for the Shell and its value shall not be
used, modified or interpreted. It is currently unused, but can be used by
future releases.
\smallskip{}

{\tt pr\_CES} is the \emph{file handle} to be used for error output. This
stream gues usually to the console the process runs in, if such a console
exists. This handle can be changed by {\tt SelectError()}\key{SelectError()}.
If {\tt pr\_CES} is {\tt NULL}, processes should fall back to {\tt pr\_COS}
for printing errors. Preferably, processes should use the
{ErrorOutput()}\key{ErrorOutput()} function to obtain an error stream,
though.

\section{Creating Processes}

AmigaDOS provides several functions to create functions: {\tt
  CreateNewProc()}\key{CreateNewProc()} is the revised and most flexible
function for launching a process, taking many parameters in the form of a
tag list. The legacy function{\tt CreateProc()}\key{CreateProc()} supports
less options, but available under all Os versions. Shells as created by the
{\tt System()}\key{System()} function implicitly also create processes, but
are not discussed here, but in section~\ref{sec:shell}. Therefore, {\tt
  System()} shares a couple of options with {\tt CreateNewProc()}.
\smallskip{}

There is surprisingly not a single function to delete processes. Processes
die whenever their execution drops off at the end of the {\tt main()}
function, or whenever execution reaches the final {\tt RTS} instruction of
the main program function. The {\tt Exit()}\key{Exit()} function also
terminates a process, but shall be called from within the process, and is
typically not suitable as it does not release resouces accquired by the
program itself, but only those allocated by the system itself.

\subsection{Creating a New Process from a TagList}

The {\tt CreateNewProc()}\mkey{CreateNewProc()} function takes a {\tt
  TagItem} array as defined in {\tt utility/tagitem.h} and launches a new
process from this list. The tags this function takes are defined in {\tt
  dos/dostags.h}.

\begin{verbatim}
process = CreateNewProc(tags)
D0                       D1

struct Process *CreateNewProc(struct TagItem *)

process = CreateNewProcTagList(tags)
D0                              D1

struct Process *CreateNewProcTagList(struct TagItem *)

process = CreateNewProcTags(Tag1, ...)

struct Process *CreateNewProcTags(ULONG, ...)
\end{verbatim}

The above functions are all equivalent, just the calling conventions are
different. For {\tt CreateNewProcTags()}, the {\tt TagList} is created by
the compiler on the stack and a pointer is then implicitly passed into the
function. The following tags are recognized by the function:
\smallskip{}

{\tt NP\_Seglist} takes a \emph{BPTR} to a segment list as returned by {\tt
  LoadSeg()} and launches the process at the first byte of the first segment
of the list.
\smallskip{}

{\tt NP\_FreeSeglist} is a boolean indicator that defines whether the
segment provided to {\tt NP\_Seglist} is released when the process
terminates. Unlike what the official documentation claims, the default value
of this tag is {\tt DOSFALSE}, i.e. the segment is \emph{not} released.
\smallskip{}

{\tt NP\_Entry} is mutually exclusive to {\tt NP\_Seglist} and defines an
absolute address (and not a segment) as entry point of the process to be
created. If this tag is provided, then {\tt NP\_FreeSeglist} shall
\emph{not} be set a non-zero value. Either {\tt NP\_Entry} or {\tt
  NP\_Seglist} shall be included.
\smallskip{}
     
{\tt NP\_Input} sets the input file handle, i.e. {\tt pr\_CIS} of the
process to be created. This tag takes a \emph{BPTR} to a \emph{file
handle}. The default is \emph{not} to set the input file handle, e.g. to
leave it {\tt ZERO}.
\smallskip{}

{\tt NP\_CloseInput} selects whether the input file handle, if provided,
will be closed when the process terminates. If non-zero, the input file handle
will be closed, otherwise it remains opened. The default is to close the
input file handle.
\smallskip{}

{\tt NP\_Output} sets the output file handle, i.e. {\tt pr\_COS} of the
process to be created. This tag takes a \emph{BPTR} to a \emph{file
handle}. The default is to leave the output at {\tt ZERO}.
\smallskip{}

{\tt NP\_CloseOutput} selects whether the output file handle, if
provided, will be closed when the process terminates. If non-zero, the
output file handle will be closed, otherwise it remains open. The default is
to close the output file handle.
\smallskip{}

{\tt NP\_Error} sets the error file handle, i.e. {\tt pr\_CES} of the
process to be created. This tag also takes a \emph{BPTR} to a \emph{file
handle}. The default is to leave the error output handle at {\tt ZERO}.
\smallskip{}

{\tt NP\_CloseError} selects whether the rror file handle, if provided, will
be closed when the process terminates. If non-zero, the error file handle
will be closed, otherwise it remains open. The default is \emph{not} to
close the error file handle. This (different) default is to ensure backwards
compatibility.
\smallskip{}

{\tt NP\_CurrentDir} sets the current directory of the process to be
created. The argument is a \emph{Lock}. The default is to duplicate the
current directory of the caller with {\tt DupLock()}\key{DupLock()} if the
caller is a process, or leave the current directory at {\tt ZERO}.
The current directory of the process, i.e. {\tt pr\_CurrentDir}, is released
when the process terminates, unless {\tt NP\_CurrentDir} is set to {\tt ZERO}. 
\smallskip{}

{\tt NP\_StackSize} sets the stack size of the process to be created in
bytes. The default is a stack size of $4000$ bytes.
\smallskip{}

{\tt NP\_Name} is a pointer to a {\tt NUL} terminated string to which the
task name of the process to be created is set. This string is copied before
the process is launched, and the copy is released automatically when the
process terminates. The default process name is ``{\tt New Process}''.
\smallskip{}

{\tt NP\_Priority} sets the priority of the process to be created. The tag
value shall be an integer in the range~$-128$ to~$127$, though useful values
are in the range of~$0$ to~$20$. The default is~$0$.
\smallskip{}

{\tt NP\_ConsoleTask} specifies a pointer to a \emph{MsgPort} to the handler
that is responsible for the console of the process to be created. That is,
if the created process opens ``{\tt *}'' or a path relative to {\tt
  CONSOLE:}, it will use the specified handler. The default is to use the
console handler if the caller is a process, or {\tt NULL} if the caller is 
only a task.
\smallskip{}

While not explicitly available as a tag, the default file system of the created
process, i.e. {\tt pr\_FileSystemTask}, is set to the default file system of
the calling process if the caller is a process, or otherwise use the default
file system from the \emph{dos.library}. This file system is contacted to
resolve paths relative to the {\tt ZERO} lock.
\smallskip{}

{\tt NP\_WindowPtr} specifies a pointer to a {\tt Screen} on which error
requesters will be displayed, $0$ to display requesters on the workbench, or
$-1$ to suppress error requesters. It will be installed in the
{\tt pr\_WindowPtr} of the process to be created. The default is to copy the
pointer from the calling process, or {\tt NULL} in case the caller is only a
task.
\smallskip{}

{\tt NP\_HomeDir} sets the {\tt pr\_HomeDir} \emph{lock} which is used to
resolve paths relative to the {\tt PROGDIR:} pseudo-assign. The default is
to copy {\tt pr\_HomeDir} of the calling process, or {\tt ZERO} in case the
caller is a task. This \emph{lock} is released when the process terminates,
i.e. the \emph{lock} provided as argument here remains available to the
caller, and shall be released by the caller in one way or another.
\smallskip{}

{\tt NP\_CopyVars} determines if the local shell variables in
{\tt pr\_LocalVars} of the calling process are copied into the variables of the
process to be created. If set to non-zero, a copy of the variables of the
calling process are made, otherwise the new process does not receive any
shell variables by itself. The latter also happens if the caller is a task
and not a process. The variables are automatically released when the new
process terminates.
\smallskip{}

{\tt NP\_Cli} determines whether the new process will receive a new shell
environment in the form of a {\tt CommandLineInterface} structure. If
non-zero, a new CLI structure will be created and a \emph{BPTR} to this
structure will be filled into the {\tt pr\_CLI} member of the process to be
created. The new shell environment will be a copy of the shell environment
of the caller if one is present, or a shell environment initialized with all
defaults. This means that the prompt, the path, and the command name will be
copied over. If~$0$, no such environment will be created. The latter is also the
default.
\smallskip{}

{\tt NP\_Path} provides a chained list of \emph{locks} within which commands
are searched. This is the same list the {\tt PATH} command adjusts, see
section~\ref{sec:shell} for details on this structure. This tag only applies
if {\tt NP\_Cli} is non-zero to create a shell environment. This chained
list is \emph{not} copied, and will be released when the created process
terminates; hence, the locks provided here are \emph{no longer} available to
the caller if {\tt CreateNewProc()} succeeds. If {\tt CreateNewProc()}
fails, the entire lock list remains a property of the caller and thus needs
to be potentially released there. The default, if this tag is not provided, is
to copy the paths of the caller if the calling process has a non-zero
{\tt pr\_CLI} structure.
\smallskip{}

{\tt NP\_CommandName} provides the name of the command being executed within
the shell environment if {\tt NP\_Cli} indicates that one is to be
created. The default is to copy the command name of the shell environment of
the calling process if one exists, or to leave the command name empty if
none is provided. The command name is copied into the shell environment of
the process being created and thus remains available to the caller. More on
the shell environment is found in section~\ref{sec:shell}.
\smallskip{}

{\tt NP\_Arguments} provides command line arguments for the process to be
created. This is a {\tt NUL} terminated string that is copied into the
process to be created, and will also be released there. If provided, the
arguments are copied in {\tt pr\_Arguments} of the process to be created,
and will also be loaded into registers {\tt A0} and its length into
{\tt D0}. If {\tt NP\_Arguments} are non-zero, a non-{\tt ZERO} {\tt
  NP\_Input} file handle shall also provided. This is because the arguments
are also copied into the buffer associated to the input \emph{file handle}
to make them available to {\tt ReadArgs()}\key{ReadArgs()}, or any other
function that performs buffered read from {\tt pr\_CIS}, see
section~\ref{sec:bufferedio} for details.
\smallskip{}

{\tt NP\_ExitCode} determines a pointer to function that is called when the
created process terminates. This pointer is filled into {\tt
  pr\_ExitCode}. See section~\ref{sec:process} for the description and the
signature of this function.
\smallskip{}

{\tt NP\_ExitData} provides an argument that will be passed into the {\tt
  NP\_ExitCode} function in register {\tt D1} when the process terminates.
\smallskip{}

While the official documentation also mentiones the tags {\tt
  NP\_NotifyOnDeath} and {\tt NP\_Synchronous}, these tags are currently
ignored and do not perform any function.
\medskip{}

The {\tt CreateNewProc()} function returns on success a pointer to the
{\tt Process} structure just created. At this stage, the process has already
been launched and, depending on its priority, may already be running. On
failure, the function returns {\tt NULL}. Unfortunately, it does not set
{\tt IoErr()} consistently on failure.


%%% For later.

The {\tt DateStamp} structure reads as follows:

\begin{verbatim}
struct DateStamp {
   LONG  ds_Days;             /* Number of days since Jan. 1, 1978 */
   LONG  ds_Minute;           /* Number of minutes past midnight */
   LONG  ds_Tick;             /* Number of ticks past minute */
};
\end{verbatim}

{\tt ds\_Days} counts the number of days since January $1^{\mbox{{\tiny st}}}$ 1978.

{\tt ds\_Minute} counts the number of minutes past midnight, i.e. the start of the day.

{\tt ds\_Tick} counts the ticks since the start of the minute. A tick
is $1/50^{\mbox{\tiny th}}$ of a second, regardless whether the
machine is a PAL or NTSC system. This constant is also defined as
{\tt TICKS\_PER\_SECOND} in {\tt dos/dos.h}.




\begin{thebibliography}{99}
\bibitem{Bantam} Commodore-Amiga Inc:
{\sl AmigaDOS Manual, $3^{\mbox{\tiny rd}}$ Edition}
Random House Information Group (1991)  
\bibitem{mc030} Motorola MC68030UM/AD Rev. 2:
{\sl MC68030 Enhanced 32-Bit Microprocessor User's Manual, 3rd ed.}
Prentice Hall, Englewood Cliffs, N.J. 07632 (1990)
\bibitem{mc040} Motorola MC68040UM/AD Rev. 1:
{\sl MC68040 Microprocessor User's Manual, revised ed.}
Motorola (1992,1993)
\bibitem{mc060} Motorola MC68060UM/AD Rev. 1:
{\sl MC68060 Microprocessor User's Manual.}
Motorola (1994)
\bibitem{mcfam} Motorola MC68000PM/AD Rev. 1:
{\sl Programmer's Reference Manual.}
Motorola (1992)
\bibitem{yuchen} Yu-Cheng Liu:
{\sl The M68000 Microprocessor Family.}
Prentice-Hall Intl., Inc. (1991)
\bibitem{rkrmlib} Dan Baker (Ed.):
{\sl Amiga ROM Kernal Reference Manual: Libraries. 3rd. ed.}
Addison-Wesley Publishing Company (1992)
\bibitem{rkrmdev} Dan Baker (Ed.):
{\sl Amiga ROM Kernal Reference Manual: Devices. 3rd. ed.}
Addison-Wesley Publishing Company (1992)
\bibitem{rkrmiia} Dan Baker (Ed.):
{\sl Amiga ROM Kernal Reference Manual: Includes and Autodocs. 3rd. ed.}
Addison-Wesley Publishing Company (1991)
\bibitem{guru} Ralph Babel:
{\sl The Amiga Guru Book.}
Ralph Babel, Taunusstein (1993)
\end{thebibliography}
%\begin{theindex}
\printindex
%\end{theindex}
\end{document}


